%%%%%%%%%%%%%%%%%%%%%%%%%%%%%%%%%%%%%%%%%%%%%%%%%%%%%%%%%%%%%%%%%%%%%%%%%%%%
% RCS INFORMATION:
%
%       $RCSfile: ug_spaceball_display.tex,v $
%       $Author: johns $        $Locker:  $                $State: Exp $
%       $Revision: 1.7 $      $Date: 2014/12/29 03:37:44 $
%
%%%%%%%%%%%%%%%%%%%%%%%%%%%%%%%%%%%%%%%%%%%%%%%%%%%%%%%%%%%%%%%%%%%%%%%%%%%%
% DESCRIPTION:
%  The controls available from the OpenGL Display window
%
%%%%%%%%%%%%%%%%%%%%%%%%%%%%%%%%%%%%%%%%%%%%%%%%%%%%%%%%%%%%%%%%%%%%%%%%%%%%

\section{Using the Spaceball in the Graphics Window}
\label{ug:spaceball}
\index{spaceball!using}
\index{SpaceNavigator!using}

\VMD\ provides optional support for SpaceNavigator, Magellan, and
Spaceball six-degree-of-freedom input devices.  
The Spaceball may be used to rotate, translate, and scale molecules,
using up to 6 control axes simultaneously (3 axes in translation, 
3 in rotation).  The Spaceball can be used independently and
simultaneously with the mouse.  With the spaceball in one hand and the 
mouse in the other, a user can perform complex picking and identification
operations more efficiently, since the mouse can be left in pick mode
(for example) while the Spaceball is used to perform rotations, 
translations, and scaling operations with the other hand. 

\index{spaceball!modes}
\index{SpaceNavigator!modes}
The Spaceball can be run in one of several modes within \VMD. 
The Spaceball interface currently provides two methods of rotation 
and translation, and a scaling mode.  The Spaceball interface
currently uses {\em Button 1} (known as {\em Function 1} in the SpaceWare
driver) to reset the view, and {\em Button 2} to cycle through the
available Spaceball interface modes.

\subsection{Spaceball Driver}
\index{spaceball!driver}
\index{SpaceNavigator!driver}
\VMD\ interfaces to the Spaceball in one of two 
ways; either by communicating directly with the Spaceball using 
built-in serial interface software, or vendor provided drivers.
\index{spaceball!driver!serial}
\index{environment variables!VMDSPACEBALLPORT}
Unix and Mac OS X versions of \VMD\ use the built-in serial Spaceball driver.
At startup, \VMD\ checks for the existence of an environment
variable {\em VMDSPACEBALLPORT}.  This environment variable must be
set to the Unix device name of the serial port to which the Spaceball
is attached.  The serial port device permissions must be set to allow
the \VMD\ user to open the device for reading and writing.  In typical
usage, this usually requires performing a {\tt chmod 666 /dev/somettyname} 
on the appropriate device as root.
One restriction with the use of the built-in Spaceball driver is that
only one \VMD\ process may safely use the Spaceball at a time.  If multiple
\VMD\ sessions are started on the same machine and all are set to open
the Spaceball, it will behave very erratically.

\index{spaceball!driver!windowing system}
\index{SpaceNavigator!driver!windowing system}
The Linux and Windows version of \VMD\ can use open source (e.g. spacenavd)
or vendor-provided (SpaceWare) driver to communicate with 
SpaceNavigator, Magellan, or Spaceball devices via windowing system events.  
The window system drivers operate somewhat differently from the 
serial driver built into VMD.  
The window system driver software runs as a separate process
from \VMD\ and must be started and fully operational before \VMD\ is run.
At startup time \VMD\ attempts to open the windowing system driver 
interface, displaying the success or failure of initialization as it occurs, 
with applicable diagnostic information.  The windowing system
driver provides detailed control over the sensitivity and configuration 
of the Spaceball, Magellan, or SpaceNavigator device.  In order to use
the Spaceball function keys with \VMD\, the windowing system driver must be set
to send button events as {\em Function 1} and {\em Function 2} at a minimum.
Once set, it should be possible to cycle through the various \VMD\ Spaceball 
operational modes as described below.

