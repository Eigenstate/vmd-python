%%%%%%%%%%%%%%%%%%%%%%%%%%%%%%%%%%%%%%%%%%%%%%%%%%%%%%%%%%%%%%%%%%%%%%%%%%%%
% RCS INFORMATION:
%
%       $RCSfile: ug_mol_colors.tex,v $
%       $Author: johns $        $Locker:  $                $State: Exp $
%       $Revision: 1.32 $      $Date: 2014/12/29 02:50:13 $
%
%%%%%%%%%%%%%%%%%%%%%%%%%%%%%%%%%%%%%%%%%%%%%%%%%%%%%%%%%%%%%%%%%%%%%%%%%%%%
% DESCRIPTION:
%  molecule drawing methods
%
%%%%%%%%%%%%%%%%%%%%%%%%%%%%%%%%%%%%%%%%%%%%%%%%%%%%%%%%%%%%%%%%%%%%%%%%%%%%

\section{Coloring Methods}
\label{ug:topic:coloring}
\index{coloring!methods}

\VMD\ maintains a database of the colors used for the molecules and
other graphical objects which are visible in the display window.  It
keeps track of
\begin{itemize}
  \item color name definitions - its RGB value;
  \item mappings from a color category to color name - so residue name
MET is colored yellow
  \item the current color scale - red to white to blue, and several related parameters
\end{itemize}

\index{color!names}
\index{color!id}
There are 1057 colors available in VMD, with color ids ranging from 0 to
1056.  The first 33 are,
in order: blue, red, gray, orange, yellow, tan, silver, green, white,
pink, cyan, purple, lime, mauve, ochre, iceblue, black,
yellow2, yellow3, green2, green3, cyan2, cyan3, blue2, blue3, 
violet, violet2, magenta, magenta2, red2, red3, orange2, and orange3.

The next group of 1024 colors (from 33 to 1056) are colors used in
the color map, These can be set to one of
several ranges with the {\sf Color} window or the {\tt color} text command:
\index{window!color}
\index{color!window}
\index{color!command}
red$\rightarrow$green$\rightarrow$blue,
red$\rightarrow$white$\rightarrow$blue, or black$\rightarrow$white,
etc.  There are no names for the specific colors.  The color map will
be discussed in more detail in a section to follow.


\subsection{Color categories}
\label{ug:topic:color:categories}
\index{color!category}

\VMD\ maintains a database of the colors used for the molecules and the
other graphical objects in the display window.  The database consists
of several color {\em categories}; each color
category contains a list of names, and each name is assigned a color.
For example, there is a Resname color category, and within this
category there are many names; one for each of the available residue
names.  Some of these are {\tt ALA}, {\tt CYS}, and {\tt PRO}.  Each
name can be assigned a color from a list of 33 available colors called
the \index{color map}{\it color map}.  The RGB value of each color can
be modified directly in the
\hyperref{{\sf Color} window}{{\sf Color} window [\S}{]}{ug:ui:window:color}.
To color items in a gradation manner,
there are additional 1024 colors used in the
\hyperref{color scale}{color scale [\S}{]}{ug:topic:coloring:scale}.

The different color categories in \VMD\ are listed in table
\ref{table:ug:colorcats}.  The {\sf Color} window\index{color!window}
\index{window!color}
can be used to change the assignment
of colors to the names in each of these categories.  For example, to
change the color used to draw Arginine residues when molecules are
colored by residue, you would use the {\sf Color} window, select the `Resname'
category, select the `Arg' name there, and then pick the color to use
for Arginine's from the list of colors next to the names.

\begin{table}[htb]
  \hspace{0.4in}
  \begin{tabular}{|l|l|} \hline
    \multicolumn{1}{|c}{Category} &
        \multicolumn{1}{|c|}{Contents} \\ \hline\hline
    Display	& Color of background, gradient, depth cueing, text \\
    Axes	& The components of the axes \\
    Name	& The available atom names (color by Name) \\
    Type	& The available atom types (color by Type) \\
    Element     & Atomic elements  (color by Element), with ''X'' for unknown \\
    Resname	& The residue names (color by ResName) \\
    Restype	& The residue types (color by ResType) \\
    Chain	& The one-character chain identifier. \\
    Segname	& The segment names (color by SegName)  \\
    Conformation & The available conformation codes (color by Conformation) \\
    Molecule	& The names assigned to each molecule (color by Molecule) \\
    Highlight	& The protein, nucleic, and non-backbone colors \\
    Structure   & The secondary structure type (helix, sheet, coil) (color by Structure) \\
    Surface     & The surface types \\
    Labels	& The different labels (atoms, bonds, etc.) \\
    Stage	& The colors for the checkboard stage \\
    \hline
  \end{tabular}
  \caption{Color categories used in \VMD.}
  \label{table:ug:colorcats}
\index{color!category}
\end{table}


\subsection{Coloring Methods}
\label{ug:topic:coloring:methods}
\index{coloring!methods}

As described in chapter \ref{ug:topic:drawing}, each representation
for a molecule has a specific {\em coloring method}.  The coloring
method determines how the color for each atom in the representation
(view) is determined.  These different methods use the colors assigned
to the names in the categories listed above, and use those names to
color the atoms.  Molecular drawing methods which also draw the bonds
between atoms will always color each half of the bond separately,
using the color of the nearest atom for each half.  
Table~\ref{table:ug:colormethods} lists the different coloring methods
available.  The description for each method explains the source of
the information used to determine the color.

\begin{table}[htb]
%  \hspace{0.4in}
  \begin{tabular}{|l|l|} \hline
    \multicolumn{1}{|c}{Method} &
        \multicolumn{1}{|c|}{Description} \\ \hline\hline
    Name	& Atom name, using the Name category \\
    Type	& Atom type, using the Type category \\
    Element     & Atomic element, using the Element category \\
    ResName	& Residue name, using the Resname category \\
    ResType	& Residue type, using the Restype category \\
    ResID	& Residue identifier, using the resid mod 16 for the color \\
    Chain	& The one-character chain identifier,
			using the Chain category \\
    SegName	& Segment name, using the Segname category \\
    Conformation & Conformation, e.g. PDB alternate location identifier \\
    Molecule	& Molecule all one color, using the Molecule category \\
    Structure	& Helix, sheet, and coils are colored differently \\
    ColorID	& Use a user-specified color index (from 0 to 15) \\
    Beta	& Color scale based on beta value of the PDB file \\
    Occupancy	& Color scale based on the occupancy field of the PDB file \\
    Mass	& Color scale based on the atomic mass \\
    Charge	& Color scale based on the atomic charge \\
    Pos		& Color scale based on radial distance from the molecule center \\
    PosX, PosY, PosZ & Color scale based on axial distance from the molecule center \\
    User, User2, User3, User4  & Color scale assigned by per-atom values 
                for each timestep \\
    PhysicalTime, Timestep & Color scale based on the physical (simulation) time or \\
                & timestep index associated with the displayed trajectory frame \\
    Velocity    & Color scale based on the per-atom velocity value \\
                & associated with the displayed trajectory frame \\
    Fragment    & Color scale based on the VMD fragment index \\
    Index	& Color scale based on the VMD atom index \\
    Backbone	& Backbone atoms green, everything else is blue \\ 
    Throb	& Color scale animated by the current wall clock time \\ 
    Volume      & Surfaces are colored by the linked volumetric data set \\
  \hline
  \end{tabular}
  \caption{Molecular coloring methods.}
  \label{table:ug:colormethods}
\index{coloring!methods}
\end{table}


\subsection{Coloring by color categories}
\label{ug:topic:coloring:categories}
\index{color!category}
\index{coloring!by category}

The default method is to color by the atom name.  The way it works is
that there is a color category called `Name' which contains a list of
all the atom names (e.g., CA, N, O5', and H) that have been loaded into
\VMD.  Each name is assigned one of the 16 main colors (e.g., cyan, blue, red,
and white).  When the drawing representation needs a color for a
specific atom, it looks in the appropriate color category and finds
that CA is colored cyan, N is blue, and so on.

Most of the coloring methods are based on color categories, so
coloring by `ResName' colors each residue name differently, `SegName'
colors each segment differently, and so on.  The mapping between a
given item in a color category and a color can be changed using the
\hyperref{{\sf Color} window}{{\sf Color} window [\S}{]}{ug:ui:window:color}.
\index{color!window}
\index{window!color}
This allows users to make atoms with the name CA be black and the
residue CYS be yellow.  Some attention was given to making the colors
reasonable, so that oxygens are red, nitrogens blue, sulphur and
cysteines yellow, etc.

\subsection{Color scale}
\label{ug:topic:coloring:scale}
\index{color!scale}
\index{coloring!by color scale}

Several of the coloring methods, including `Beta', `Charge', and
`Occupancy', describe a range of floating point values rather than a
set of names.  These are colored via the {\em color scale}, which is a
list of 1024 smoothly changing colors.  There are many color gradations
available.  All of them consist of transformations of three colors.
For instance, ``RGB'' colors the smallest value red, values near the
middle of the scale are green, and the largest values are blue.
Colors in-between are linear mixes of the two colors.  The list of 
available gradations is given below.


\begin{table}[htb]
  \hspace{0.7in}
  \begin{tabular}{|l|l|} \hline
    \multicolumn{1}{|c}{Method} &
        \multicolumn{1}{|c|}{Description} \\ \hline\hline
    RWB         & small=red,   middle=white, large=blue \\
    BWR         & small=blue,  middle=white, large=red \\
    RGryB       & small=red,   middle=gray,  large=blue \\
    BGryR       & small=blue,  middle=gray,  large=red \\
    RGB         & small=red,   middle=green, large=blue \\
    BGR         & small=blue,  middle=green, large=red \\
    RWG         & small=red,   middle=white, large=green \\
    GWR         & small=green, middle=white, large=red \\
    GWB         & small=green, middle=white, large=blue \\
    BWG         & small=blue,  middle=white, large=green \\
    BlkW        & small=black, large=white \\
    WBlk        & small=white, large=black \\ \hline
  \end{tabular}
  \caption{Available Color Scale Gradations.}
  \label{table:ug:gradmethods}
\index{color!scale}
\index{coloring!by color scale}
\end{table}


The minimum of the range of values is linearly scaled and shifted to
start at 0 and end at 1.  Assume the color scale is RGB.  For a given
value of x in the scale range [0..1], the RGB value is found first
from a linear scaling based on the midpoint.
If x $=$ 0, R is 1 (for maximum red).  This continues linearly until x
$=$ midpoint, at which point, R is 0 and stays 0.
The green component is 0 at both x $=$ 0 and x $=$ 1 and is 1 at the
midpoint.  Linear scaling occurs in between.
The blue component is 0 for x $<=$ midpoint, and 1 for x $=$ 1.
\begin{rawhtml}
<CENTER>
\end{rawhtml}
\myfigure{ug_color_scale}{Example showing red/green/blue gradients summed to produce the color scale.}
{fig:ug:color_scale}
\begin{rawhtml}
</CENTER>
\end{rawhtml}

An additional term, ``min'', is added to each of the component terms
before they are merged.  This shifts the final colors more towards
white or black.  Min can take on values from -1 to 1.

\begin{rawhtml}
<CENTER>
\end{rawhtml}
\myfigure{ug_color_scale_min}{The shift to the red component of the RGB 
scale caused by the value of ``min''.}
{fig:ug:color_scale_min}
\begin{rawhtml}
</CENTER>
\end{rawhtml}

There is only one color scale used at a time so it is impossible to
display objects colored by multiple different color scales.

\subsection{Materials}
\label{ug:topic:coloring:materials}

\VMD\ allows users to apply a materials property to the
molecular models they create.  The material determines such things
as how transparent an object is, or how shiny, or how large the specular 
reflections are.   
Making objects semi-transparent is a potentially powerful means of
viewing multiple layers of the molecule simultaneously.  Imagine a
protein on the surface of, and extending part way into, a membrane.
One way to visualize the extent of the penetration is to represent the
lipids as `Bonds' and make them transparent.  That will show the
membrane without completely obstructing the view of the protein.

\VMD\ maintains a database of materials which can be applied to any
representation in the system, much like the database for colors.  There
are two default materials, "Opaque" and "Transparent", which cannot be
modified.  Each material is defined by five settings, as follows: 

\begin{itemize}
\item Ambient: The ambient coefficient describes how strongly the 
            material reflects ambient light.  Ambient light
            provides a uniform illumination of objects with a
            background lighting of the object color.  The ambient light
            factor is generally used to moderate the effects of 
            shadows from direct lighting, making shadows less dark than
            they otherwise would be.
\item Diffuse: Diffuse reflections are independent of the viewing
            direction, but depend on the direction of the light source
            with respect to the surface of the displayed object.
\item Specular: The specular coefficient describes the intensity of
            specular highlights.  The higher the specular value, 
            the brighter the resulting highlights.
\item Shininess: The shininess coefficient describes the breadth of
            the angle of specular reflection. The smaller the number the
            broader the angle and the rougher objects appear.  The larger
            the value of shininess, the narrower the angle of specular
            reflection, and the smoother the surface.
            Default corresponds to a Phong exponent of 40.
\item Mirror: The mirror coefficient describes the mirror
            reflectivity of a surface.  When the scene is rendered using
            ray tracing, surfaces with mirror reflectivity will show 
            reflections much like a mirror-polished metal surface.
\item Opacity: The opacity coefficient describes how opaque the surface is;
            1 is solid, 0 is transparent. By default, transparent objects 
	    are drawn with Opacity set to 0.3.
\item Outline: The outline coefficient controls shading of sillhouette edges,
            darkening the visible edges of surfaces where they are nearly 
            perpendicular to the camera view direction.  The outline parameter
            scales the degree of edge shading (darkening).
\item OutlineWidth: The outlinewidth coefficient controls the angular width of
            nearly perpendicular sillhouette edge that is shaded.
\end{itemize}
For details regarding these material properties, consult an elementary
graphics book such as Foley \& Van Dam (Computer Graphics).  

