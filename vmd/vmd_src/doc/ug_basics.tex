%%%%%%%%%%%%%%%%%%%%%%%%%%%%%%%%%%%%%%%%%%%%%%%%%%%%%%%%%%%%%%%%%%%%%%%%%%%%
% RCS INFORMATION:
%
%       $RCSfile: ug_basics.tex,v $
%       $Author: eamon $        $Locker:  $                $State: Exp $
%       $Revision: 1.14 $      $Date: 2003/06/06 19:03:38 $
%
%%%%%%%%%%%%%%%%%%%%%%%%%%%%%%%%%%%%%%%%%%%%%%%%%%%%%%%%%%%%%%%%%%%%%%%%%%%%
% DESCRIPTION:
%  How to use the mouse, the basic menus, the various Forms options
%
%%%%%%%%%%%%%%%%%%%%%%%%%%%%%%%%%%%%%%%%%%%%%%%%%%%%%%%%%%%%%%%%%%%%%%%%%%%%

\chapter{User Interface Components}
\label{chapter:ug:uis}

\VMD\ provides several methods for the user to control and
interact with the molecular display.  The primary methods are by using the
mouse, either in the graphics window or in the different graphical user 
interface (GUI) {\em forms} provided by the program.  
In addition to the mouse, \VMD\ also supports a number of more advanced 
input devices such as the Spaceball, Magellan, and Phantom, 
which provide the ability to manipulate molecules with six degrees of freedom.
Some devices such as the Phantom can also provide haptic (sense of touch) 
force feedback.
\VMD\ also provides a text console interface for executing built-in commands 
or running scripts.  
This chapter describes how to use the mouse-based user interfaces, 
and some of the advanced input devices supported in VMD. 
The the text and scripting interface is described fully in 
chapter \ref{chapter:ug:text}.

% all about the Graphics Display
%%%%%%%%%%%%%%%%%%%%%%%%%%%%%%%%%%%%%%%%%%%%%%%%%%%%%%%%%%%%%%%%%%%%%%%%%%%%
% RCS INFORMATION:
%
%       $RCSfile: ug_mouse_display.tex,v $
%       $Author: johns $        $Locker:  $                $State: Exp $
%       $Revision: 1.30 $      $Date: 2012/01/10 19:30:06 $
%
%%%%%%%%%%%%%%%%%%%%%%%%%%%%%%%%%%%%%%%%%%%%%%%%%%%%%%%%%%%%%%%%%%%%%%%%%%%%
% DESCRIPTION:
%  The controls available from the OpenGL Display window
%
%%%%%%%%%%%%%%%%%%%%%%%%%%%%%%%%%%%%%%%%%%%%%%%%%%%%%%%%%%%%%%%%%%%%%%%%%%%%

\section{Using the Mouse in the Graphics Window}
\label{ug:ui:disp}
\index{mouse!using}

The graphics window is labeled {\sf VMD OpenGL Display} and contains a
view of the molecules and other objects which make up the scene.  
When the mouse is in the graphics display window, it may be used to perform 
the following actions such as:
\begin{itemize}
  \item Rotate, translate, or scale the displayed molecules
  \item Select, or `pick' atoms or other objects in order to 
        move them, or label them 
  \item Translate and rotate a set of atoms
  \item Apply a force (acceleration) to a set of atoms
  \item Move the lights
\end{itemize}
User-defined keyboard accelerators, or {\em hot keys}, are also available
when the mouse is in the graphics display window.  
These keys are bound to \VMD\ text commands, which are executed when the key
is pressed.  \VMD\ has many built-in default hot key commands (see 
Tables~\ref{table:ug:mouse:control}, \ref{table:ug:rotations},
\ref{table:ug:menushortcuts} and \ref{table:ug:animationhotkeys}). 
Users can add new hot keys, overriding default settings if desired.  

%%%%%%%%%%%%%%%%%%%%%%%%%%%%%%%%%%%%%%%%%%

\subsection{Mouse Modes }
\label{ug:ui:disp:modes}
\index{mouse!modes}

The mouse is in one of several {\em modes} at any time; 
the current mouse mode determines the effect of pressing
and releasing mouse buttons or the mouse wheel while the mouse
is in the graphics window.
Each mouse mode, except the lights mode (see below), 
sets the mouse cursor to a characteristic shape.  
The mouse mode is selected via the Mouse menu.

The available mouse modes are as follows:
\begin{itemize}

\item {\bf Rotate Mode}
\hspace{0.2in} (hot key 'r')
\\
\label{ug:ui:disp:rotate}
\index{rotation!using mouse}
When the mouse is in rotate mode, holding the left mouse button down 
and moving the mouse rotates the molecules about axes parallel to the
screen, in a `virtual trackball' behavior.  To get a rotation around
the axes coming out of the screen (the `z' axis), hold the middle
button down and move the mouse left or right.

You can leave \index{rotation!continuous}molecules rotating
without continuously moving the mouse.  Start the molecule moving with
the mouse, as above, then release the mouse button before you stop
moving the mouse.  With some practice it becomes easy to impart a
slight spin on the molecule, or whirl it about madly.  To stop the
rotation, either press and hold the left mouse button down until the
molecule stops moving, or select `Stop Rotation' in the Mouse menu.
Also, pressing the rotation hot key {\tt r} or any of the other mouse 
mode hot keys causes \index{rotation!stop}rotation to stop.

\item {\bf Translate Mode} 
\hspace{0.2in} (hot key 't')
\\
\label{ug:ui:disp:trans}
\index{translation!using mouse}
When the mouse is in translate mode, holding the left button
down allows you to move the molecules parallel to the screen plane 
(left, right, up, and down).  To move
the molecule towards or away from you, hold the middle button down and
move the mouse right or left, respectively.

\item {\bf Scale Mode} 
\hspace{0.2in} (hot key 's')
\\
\label{ug:ui:disp:scale}
\index{scaling!using mouse}
Pressing either the left or middle button down and moving to the right 
enlarges the molecules, and moving the mouse left shrinks them.  
The difference is that the middle button scales faster than the left button.
Scaling can also be accomplished with the mouse wheel (irrespective
of the current mode setting) on computers equipped with an appropriate
mouse.

\item {\bf Move Light} 
\\
\label{ug:ui:disp:lights}
\index{light!controlling with mouse}
\index{image!lighting controls}
\VMD\ provides four directional lights to illuminate the molecular scene.  
The lights provide diffuse lighting and specular highlights 
and help the user perceive surface shape in rendered objects.
You can use the mouse to rotate each of the light source directions
to a new position.
If the light isn't on, moving it will not affect the displayed image.
To turn a light on or off, use the {\sf Lights} item within the 
{\sf Mouse} menu.

\item {\bf Add/Remove Bonds} \\
\label{ug:ui:disp:addbonds}
\index{bonds!add or remove}
\index{mouse!add or remove bonds}
When the mouse is in add/remove bonds mode, clicking on atoms in a 
molecule will add a bond between those atoms if one is not already present,
or remove the bond between those atoms if there is already a bond.  The
two atoms must belong to the same molecule.  

\end{itemize}


%%%%%%%%%%%%%%%%%%%%%%%%%%%%%%%%%%%%%%%%%%

\subsection{Pick Modes}
\label{ug:ui:disp:pick}
\index{picking!modes}
\index{picking!atoms}
\index{picking!bonds}
\index{picking!angles}
\index{picking!dihedrals}
\index{labels!picking with mouse}
Mouse picking can be used to turn on or off various types
of labels, to query for information about an object, or to move items
around on the screen.  You can label an atom (and display the atom
name), or you can label geometric values such as the
distance between two atoms (a {\em bond} label), an angle between three
atoms (an {\em angle} label), or the dihedral angle formed by four
atoms (a {\em dihedral} label).  This is done by setting the mouse
into the proper picking mode and then selecting the relevant atoms
with the mouse.  Picking modes are selected from the {\sf Mouse} menu.

The available pick mode actions are:
\begin{itemize}

\item {\bf Center}\index{picking!center}
\hspace{0.2in} (hot key 'c')
\\ 
This mode is used to change the point about which a molecule
rotates when the molecule is rotated.  To cause a molecule to rotate
about a specific atom, select this mode and then click on that atom.
The rotation point may be restored to its default position (the center
of volume of the molecule) by executing the `Reset View' option from
the Mouse menu.  

\item {\bf Query}\index{picking!query} 
\hspace{0.2in} (hot key '0')
\\
Clicking on an item will print out the name of the item (e.g. the 
atom name) to the text console window.

\item {\bf Label $\rightarrow$ Atom}\index{picking!atoms} 
\hspace{0.2in} (hot key '1')
\\
Clicking on an atom will toggle on/off a label for the atom.  

\item {\bf Label $\rightarrow$ Bond}\index{picking!bonds} 
\hspace{0.2in} (hot key '2')
\\
Clicking on two atoms in a row will toggle on/off a bond distance
label between the two atoms (a dotted line with the distance printed
at the midpoint).  

\item {\bf Label $\rightarrow$ Angle}\index{picking!angles} 
\hspace{0.2in} (hot key '3')
\\
Clicking on three atoms in a row will toggle on/off a label showing
the angle formed by the three atoms. 

\item {\bf Label $\rightarrow$ Dihedral}\index{picking!dihedrals} 
\hspace{0.2in} (hot key '4')
\\
Clicking on four atoms in a row toggles on/off a label showing the
dihedral angle formed by the four atoms.  

\item {\bf Move $\rightarrow$ Atom}\index{picking!move atom} 
\hspace{0.2in} (hot key '5')
\\
In this mode, the position of an atom can be changed by clicking on 
the desired atom, and dragging with the mouse while the button is still
pressed.  This will change the atom coordinates.  

\item {\bf Move $\rightarrow$ Residue}\index{picking!move residue} 
\hspace{0.2in} (hot key '6')
\\
This mode may be used to move all the atoms in a selected residue at
the same time.  Select an atom in a residue, and move it to a new
position while keeping the mouse button pressed.  All the atoms in the
same residue as the selected one will be moved the same amount.  
Holding down the <shift> key and the left mouse button while moving the mouse
will rotate the atoms in the residue about the selected atom.  If the middle
mouse button is held down instead, the atoms in the residue will rotate about
a line drawn through the picked atom and parallel to a line coming directly
out of the screen.  This behavior is similar to the usual Rotate mode, except
that coordinates of atoms are changed. 

\item {\bf Move $\rightarrow$ Fragment}\index{picking!move fragment} 
\hspace{0.2in} (hot key '7')
\\
A {\em fragment} is a set of atoms all connected by a series of
covalent bonds.  This mode acts just like MoveResidue, except that the
atoms which are moved are all in the selected fragment rather than in
the selected residue.  This will change the atom coordinates.
Holding down the <shift> key and the left mouse button while moving the mouse
will rotate the atoms in the fragment about the selected atom.  If the middle
mouse button is held down instead, the atoms in the fragment will rotate about
a line drawn through the picked atom and parallel to a line coming directly
out of the screen.  This behavior is similar to the usual Rotate mode, except
that coordinates of atoms are changed. 

\item {\bf Move $\rightarrow$ Molecule}\index{picking!move molecule} 
\hspace{0.2in} (hot key '8')
\\
This mode may be used to move all the atoms in a selected molecule at
the same time.  Select an atom in a molecule, and move it to a new
position while keeping the mouse button pressed.  All the atoms in the
same molecule as the selected one will be moved the same amount.  
Holding down the <shift> key and the left mouse button while moving the mouse
will rotate the atoms in the molecule about the selected atom.  If the middle
mouse button is held down instead, the atoms in the molecule will rotate about
a line drawn through the picked atom and parallel to a line coming directly
out of the screen.  This behavior is similar to the usual Rotate mode, except
that coordinates of atoms are changed. 

\item {\bf Move $\rightarrow$ Rep}\index{picking!move highlighted rep} 
\hspace{0.2in} (hot key '9')
\\
This mode may be used to move all the atoms in a selected representation at
the same time.  You select a representation by clicking on one of the reps
in the browser window of the Graphics window.  In order to move the atoms in 
this rep, the atom you pick with the mouse must be selected by that rep.

When you have clicked on an atom in the rep, move the mouse to a new position
while  keeping the mouse button pressed.  All the atoms selected by the 
highlighted rep will be moved the same amount.  
Holding down the <shift> key and the left mouse button while moving the mouse
will rotate the atoms in the rep about the selected atom.  If the middle
mouse button is held down instead, the atoms in the rep will rotate about
a line drawn through the picked atom and parallel to a line coming directly
out of the screen.  This behavior is similar to the usual Rotate mode, except
that coordinates of atoms are changed. 

\end{itemize}

%%%%%%%%%%%%%%%%%%%%%%%%%%%%%%%%%%%%%%%%%%%%%%%%%%%%%%%%%%%%%%%%%%%%%%
%%%%%%%%%%%%%%%%%%%%%%%%%%%%%%%%%%%%%%%%%%%%%%%%%%%%%%%%%%%%%%%%%%%%%%
%%%%%%%%%%%%%%%%%%%%%%%%%%%%%%%%%%%%%%%%%%%%%%%%%%%%%%%%%%%%%%%%%%%%%%

\subsection{Hot Keys}
\label{ug:ui:hotkeys}
\index{hot keys}

When the mouse is in the graphics window, many commands are
accessible via programmable hot keys.  Hot keys allow you to do things
like change mouse modes or advance the animation by a frame by simply
pressing a key.  
There are a number of predefined hot keys,
as listed in tables \ref{table:ug:mouse:control},
\ref{table:ug:rotations}, \ref{table:ug:menushortcuts},
and \ref{table:ug:animationhotkeys}.
They can be printed out with the command {\tt user print keys}.
The commands listed are the text commands which are executed
when the hot key is pressed; these text commands are explained in
section \ref{ug:section:text}.  

\index{hot keys!customizing}
To add or modify a hot key, use the command
{\tt user add key {\it key} {\it command}}.
The {\em key} parameter must be a single character.  
If {\em command} contains more than one word, it must be enclosed in 
braces so that the subsequent command words are not ignored.  
When that key is pressed while the mouse cursor is in the 
graphics display window, the associated command will be executed.
\index{VMD!customizing}
Once you have a set of commands which are particularly
useful and familiar for you, you will want these
hot key commands automatically available every time you run \VMD.  
This can be done by placing the commands to add these items
in your {\tt .vmdrc} file, which is a file containing \VMD\ text
commands that is executed every time
\VMD\ starts up. 
\index{{\tt .vmdrc}} 
\index{startup files}
\index{files!startup}
The basic method for setting up this file is described
in section \ref{ug:section:vmdrc}.  
Once you have such a file, put the {\tt user add} commands in it.


\begin{table}[htb]
  \hspace{1in}
  \begin{tabular}{|c|l|l|} \hline
    Hot Key & \multicolumn{1}{|c}{Command} &
		\multicolumn{1}{|c|}{Purpose}\\ \hline\hline
	{\tt r, R}	& {\tt mouse mode 0 0} 	& enter rotate mode; stop rotation \\
	{\tt t, T}	& {\tt mouse mode 1 0} 	& enter translate mode \\
	{\tt s, S}	& {\tt mouse mode 2 0} 	& enter scaling mode \\
	{\tt 0}		& {\tt mouse mode 4 0}	& query item \\
	{\tt c}		& {\tt mouse mode 4 1}	& assign rotation center \\
	{\tt 1}		& {\tt mouse mode 4 2} 	& pick atom \\
	{\tt 2} 	& {\tt mouse mode 4 3}	& pick bond (2 atoms) \\
	{\tt 3}		& {\tt mouse mode 4 4}	& pick angle (3 atoms) \\
	{\tt 4}		& {\tt mouse mode 4 5} 	& pick dihedral (4 atoms) \\
	{\tt 5}		& {\tt mouse mode 4 6}	& move atom \\
	{\tt 6}		& {\tt mouse mode 4 7} 	& move residue \\
	{\tt 7}		& {\tt mouse mode 4 8} 	& move fragment \\
	{\tt 8}		& {\tt mouse mode 4 9}	& move molecule \\
	{\tt 9}		& {\tt mouse mode 4 13}	& move highlighted rep \\
	{\tt \%}	& {\tt mouse mode 4 10} & apply force on atom \\
	{\tt ${}^\wedge$}	& {\tt mouse mode 4 11}	& apply force on residue \\
	{\tt \&}	& {\tt mouse mode 4 12} & apply force on fragment \\ \hline
  \end{tabular}
  \caption{Mouse control hot keys.}
  \label{table:ug:mouse:control}
\index{hot keys!mouse control}
\index{mouse!modes}
\index{picking!hot keys}
\end{table}


\begin{table}[htb]
  \hspace{1in}
  \begin{tabular}{|c|l|l|} \hline
%    Hot Key & \multicolumn{1}{|c}{Command} &
%		\multicolumn{1}{|c|}{Purpose}\\ \hline\hline
    Hot Key & Command & Purpose\\ \hline\hline
	{\tt x}		& {\tt rock x by 1 -1} 	& spin about x axis \\
	{\tt X}		& {\tt rock x by 1 70} 	& rock about x axis \\
	{\tt y}		& {\tt rock y by 1 -1} 	& spin about y axis \\
	{\tt Y}		& {\tt rock y by 1 70} 	& rock about y axis \\
	{\tt z}		& {\tt rock z by 1 -1} 	& spin about z axis \\
	{\tt Z}		& {\tt rock z by 1 70} 	& rock about z axis \\
	{\tt j, Cntl-n}	& {\tt rotate x by 2}	& rotate $2^{\circ}$ about x \\
	{\tt k, Cntl-p}	& {\tt rotate x by -2}	& rotate $-2^{\circ}$ about x \\
	{\tt l, Cntl-f}	& {\tt rotate y by 2}	& rotate $2^{\circ}$ about y \\
	{\tt h, Cntl-b}	& {\tt rotate y by -2}	& rotate $-2^{\circ}$ about y \\
	{\tt g}		& {\tt rotate z by 2}	& rotate $2^{\circ}$ about z \\
	{\tt G}		& {\tt rotate z by -2}	& rotate $-2^{\circ}$ about z \\
	{\tt Cntl-a}	& {\tt scale by 1.1}	& enlarge 10 percent \\
	{\tt Cntl-z}	& {\tt scale by 0.9}	& shrink 10 percent \\ \hline
  \end{tabular}
  \caption{Rotation \& scaling hot keys.}
  \label{table:ug:rotations}
\index{hot keys!rotation and scaling}
\index{rotation!hot keys}
\end{table}


\begin{table}[htb]
  \hspace{1in}
  \begin{tabular}{|c|l|l|} \hline
%    Hot Key & \multicolumn{1}{|c}{Command} &
%		\multicolumn{1}{|c|}{Purpose}\\ \hline\hline
    Hot Key & Command &
		Purpose\\ \hline\hline
{\tt Alt-M}	& {\tt menu main off;menu main on}	& Show main menu  \\
{\tt Alt-f}	& {\tt menu files off;menu files on}	& Show files menu \\
{\tt Alt-g}	& {\tt menu graphics off;menu graphics on} & Show graphics menu \\
{\tt Alt-l}	& {\tt menu labels off;menu labels on}	& Show labels menu \\
{\tt Alt-r}	& {\tt menu render off;menu render on} 	& Show render menu \\
{\tt Alt-d}	& {\tt menu display off;menu display on}& Show display menu \\
{\tt Alt-c}	& {\tt menu color off;menu color on} 	& Show color menu \\
{\tt Cntl-r}	& {\tt display resetview}		& Reset display \\
{\tt Alt-q}	& {\tt quit confirm}			& Quit VMD with confirmation \\
{\tt Alt-Q}	& {\tt quit}				& Quit VMD \\
{\tt Alt-h}	& {\tt hyperref invert}			& Invert hyper text mode (NOT help) \\ \hline
  \end{tabular}
  \caption{Menu control hot keys.}
  \label{table:ug:menushortcuts}
\index{hot keys!menu control}
\index{window!hot keys}
\end{table}






\begin{table}[htb]
  \hspace{1in}
  \begin{tabular}{|c|l|l|} \hline
%    Hot Key & \multicolumn{1}{|c}{Command} &
%		\multicolumn{1}{|c|}{Purpose}\\ \hline\hline
    Hot Key & Command &
		Purpose\\ \hline\hline
	{\tt +,f,F}	& {\tt animate next}	& move to next frame\\
	{\tt -,b,B}	& {\tt animate prev}	& move to previous frame \\
	{\tt .,>}	& {\tt animate forward} & play animation forward \\
	{\tt ,}		& {\tt animate reverse} & play animation reverse \\
	{\tt <}		& {\tt animate reverse}	& play animation reverse \\
	{\tt /, ?}	& {\tt animate pause}	& stop animation \\ \hline
  \end{tabular}
  \caption{Animation hot keys.}
  \label{table:ug:animationhotkeys}
\index{hot keys!animation control}
\index{animation!hot keys}
\end{table}


\newpage



% Spaceball interface
%%%%%%%%%%%%%%%%%%%%%%%%%%%%%%%%%%%%%%%%%%%%%%%%%%%%%%%%%%%%%%%%%%%%%%%%%%%%
% RCS INFORMATION:
%
%       $RCSfile: ug_spaceball_display.tex,v $
%       $Author: johns $        $Locker:  $                $State: Exp $
%       $Revision: 1.7 $      $Date: 2014/12/29 03:37:44 $
%
%%%%%%%%%%%%%%%%%%%%%%%%%%%%%%%%%%%%%%%%%%%%%%%%%%%%%%%%%%%%%%%%%%%%%%%%%%%%
% DESCRIPTION:
%  The controls available from the OpenGL Display window
%
%%%%%%%%%%%%%%%%%%%%%%%%%%%%%%%%%%%%%%%%%%%%%%%%%%%%%%%%%%%%%%%%%%%%%%%%%%%%

\section{Using the Spaceball in the Graphics Window}
\label{ug:spaceball}
\index{spaceball!using}
\index{SpaceNavigator!using}

\VMD\ provides optional support for SpaceNavigator, Magellan, and
Spaceball six-degree-of-freedom input devices.  
The Spaceball may be used to rotate, translate, and scale molecules,
using up to 6 control axes simultaneously (3 axes in translation, 
3 in rotation).  The Spaceball can be used independently and
simultaneously with the mouse.  With the spaceball in one hand and the 
mouse in the other, a user can perform complex picking and identification
operations more efficiently, since the mouse can be left in pick mode
(for example) while the Spaceball is used to perform rotations, 
translations, and scaling operations with the other hand. 

\index{spaceball!modes}
\index{SpaceNavigator!modes}
The Spaceball can be run in one of several modes within \VMD. 
The Spaceball interface currently provides two methods of rotation 
and translation, and a scaling mode.  The Spaceball interface
currently uses {\em Button 1} (known as {\em Function 1} in the SpaceWare
driver) to reset the view, and {\em Button 2} to cycle through the
available Spaceball interface modes.

\subsection{Spaceball Driver}
\index{spaceball!driver}
\index{SpaceNavigator!driver}
\VMD\ interfaces to the Spaceball in one of two 
ways; either by communicating directly with the Spaceball using 
built-in serial interface software, or vendor provided drivers.
\index{spaceball!driver!serial}
\index{environment variables!VMDSPACEBALLPORT}
Unix and Mac OS X versions of \VMD\ use the built-in serial Spaceball driver.
At startup, \VMD\ checks for the existence of an environment
variable {\em VMDSPACEBALLPORT}.  This environment variable must be
set to the Unix device name of the serial port to which the Spaceball
is attached.  The serial port device permissions must be set to allow
the \VMD\ user to open the device for reading and writing.  In typical
usage, this usually requires performing a {\tt chmod 666 /dev/somettyname} 
on the appropriate device as root.
One restriction with the use of the built-in Spaceball driver is that
only one \VMD\ process may safely use the Spaceball at a time.  If multiple
\VMD\ sessions are started on the same machine and all are set to open
the Spaceball, it will behave very erratically.

\index{spaceball!driver!windowing system}
\index{SpaceNavigator!driver!windowing system}
The Linux and Windows version of \VMD\ can use open source (e.g. spacenavd)
or vendor-provided (SpaceWare) driver to communicate with 
SpaceNavigator, Magellan, or Spaceball devices via windowing system events.  
The window system drivers operate somewhat differently from the 
serial driver built into VMD.  
The window system driver software runs as a separate process
from \VMD\ and must be started and fully operational before \VMD\ is run.
At startup time \VMD\ attempts to open the windowing system driver 
interface, displaying the success or failure of initialization as it occurs, 
with applicable diagnostic information.  The windowing system
driver provides detailed control over the sensitivity and configuration 
of the Spaceball, Magellan, or SpaceNavigator device.  In order to use
the Spaceball function keys with \VMD\, the windowing system driver must be set
to send button events as {\em Function 1} and {\em Function 2} at a minimum.
Once set, it should be possible to cycle through the various \VMD\ Spaceball 
operational modes as described below.



% Joystick interface
%%%%%%%%%%%%%%%%%%%%%%%%%%%%%%%%%%%%%%%%%%%%%%%%%%%%%%%%%%%%%%%%%%%%%%%%%%%%
% RCS INFORMATION:
%
%       $RCSfile: ug_joystick_display.tex,v $
%       $Author: justin $        $Locker:  $                $State: Exp $
%       $Revision: 1.2 $      $Date: 2001/12/19 18:23:16 $
%
%%%%%%%%%%%%%%%%%%%%%%%%%%%%%%%%%%%%%%%%%%%%%%%%%%%%%%%%%%%%%%%%%%%%%%%%%%%%
% DESCRIPTION:
%  The controls available from the OpenGL Display window
%
%%%%%%%%%%%%%%%%%%%%%%%%%%%%%%%%%%%%%%%%%%%%%%%%%%%%%%%%%%%%%%%%%%%%%%%%%%%%

\section{Using the Joystick in the Graphics Window}
\label{ug:joystick}
\index{joystick!using}

The Windows version of \VMD\ provides support for the 
Windows joystick driver, and will enumerate all available joystick
devices at startup time.  The joystick interface employed in VMD 
is quite simple, allowing the use of three control axes to translate,
rotate, and scale the molecule.  The joystick interface assumes a 
device with at least two buttons.  The first joystick button resets
the view in the display window, and the second button cycles through 
each of the available joystick modes.  When \VMD\ first attaches to
each of the joysticks, they are initially disabled so that miscalibrated
joysticks do not adversely affect the \VMD\ session.  Each joystick is
initially enabled by pressing its second button to switch modes.  All
joysticks are independently controlled such that multiple joysticks can
control different control axes, and multiple users could interact 
with the program with separate controls.

\newpage



% using Forms in VMD
%%%%%%%%%%%%%%%%%%%%%%%%%%%%%%%%%%%%%%%%%%%%%%%%%%%%%%%%%%%%%%%%%%%%%%%%%%%%
% RCS INFORMATION:
%
%       $RCSfile: ug_forms.tex,v $
%       $Author: johns $        $Locker:  $                $State: Exp $
%       $Revision: 1.19 $      $Date: 2012/01/10 19:32:56 $
%
%%%%%%%%%%%%%%%%%%%%%%%%%%%%%%%%%%%%%%%%%%%%%%%%%%%%%%%%%%%%%%%%%%%%%%%%%%%%
% DESCRIPTION:
%  loads the descriptions of the different forms
%
%%%%%%%%%%%%%%%%%%%%%%%%%%%%%%%%%%%%%%%%%%%%%%%%%%%%%%%%%%%%%%%%%%%%%%%%%%%%


\section{Description of each \VMD\ window}
\label{ug:ui:windows}

\VMD\ uses several different GUI windows, each designed to control a specific
aspect of the molecular display (e.g., to control the appearance of the
graphics display window, or to change the colors of displayed objects). The
following sections give a brief description of the windows available in \VMD; 
the remaining chapters in this manual describe the actions which these 
windows make available in greater detail.

%%% explain the menus, in glorious detail

%%% these are now all part of the main window in one way or another
%%%%%%%%%%%%%%%%%%%%%%%%%%%%%%%%%%%%%%%%%%%%%%%%%%%%%%%%%%%%%%%%%%%%%%%%%%%%
% RCS INFORMATION:
%
%       $RCSfile: ug_form_main.tex,v $
%       $Author: johns $        $Locker:  $                $State: Exp $
%       $Revision: 1.40 $      $Date: 2012/01/10 19:30:05 $
%
%%%%%%%%%%%%%%%%%%%%%%%%%%%%%%%%%%%%%%%%%%%%%%%%%%%%%%%%%%%%%%%%%%%%%%%%%%%%
% DESCRIPTION:
%  Main window
%
%%%%%%%%%%%%%%%%%%%%%%%%%%%%%%%%%%%%%%%%%%%%%%%%%%%%%%%%%%%%%%%%%%%%%%%%%%%%


\subsection{Main Window}
\label{ug:ui:window:main}
\index{window!main}

\begin{rawhtml}
<CENTER>
\end{rawhtml}
\myfigure{ug_main}{The Main window}{fig:ug:main}
\begin{rawhtml}
</CENTER>
\end{rawhtml}

The {\sf Main} window is the main way to access other windows,
load and save files, control trajectory playback,
change various global program settings, access help, and to quit
the program.  
Many of these actions can also be performed with the menu shortcut keys
described in Table~\ref{table:ug:menushortcuts}.

The {\sf Quit} menu item exits \VMD.  This will bring up
another window which verifies that you do indeed wish to exit.  
Press {\sf Yes} to quit, or {\sf No} to return to \VMD.
\index{quit}


\subsubsection{Help}
 \label{ug:ui:disp:help}
 \index{help}
 The {\sf Help} menu items each start a web browser to display on-line \VMD\
 help documents.  The browser is designated by the environment variable
 \hyperref{{\tt VMDHTMLVIEWER}}{{\tt VMDHTMLVIEWER} [\S~}{]}
 {ug:exec_env:variables}. Selecting a help item multiple times may start
 multiple browsers.  The default web browser is Mozilla for Unix systems, and
 the built-in Explorer shell for Windows systems.  The menu contains items for the VMD Quick Help
 page, as well as the current User's Guide, FAQ, and links to various helpful
 information and programs.  


\subsection{Main Window Molecule List browser}
\label{ug:ui:window:mol}
\label{ug:ui:window:mol:top}
\index{molecule!list}
The {\sf Main} window shows the global status of the loaded molecules.  Any
number of molecules may be displayed by \VMD\ simultaneously.  Each molecule
can separately be hidden from view or fixed in place (e.g., prevented from
being affected by mouse rotation commands).  The window contains controls to
change the status of the molecules individually or in groups.

The browser displays information about each molecule.  A unique integer ID
is assigned to each molecule by \VMD\ when it is loaded.  The {\sf Molecule}
is the file name which contained the topology information.  {\sf Atoms}
shows the number of atoms in the molecule, and \index{frames}{\sf Frames}
gives the number of \timesteps associated with the file.

Next to each molecule is a set of status flags, which indicate the current
{\sf Status} of each molecule.  Each molecule has the following
characteristics, which can be {\em on} or {\em off}:\index{molecule!status}
\begin{itemize}

  \item {\bf Top (T)}\index{molecule!top} \\
{\em Top} indicates the
default molecule used in the text commands when nothing is specified
for the {\tt mol} text command.  It is also used in some forms (like
Graphics and Animate) to determine certain values.  There can be only
one top molecule at a time.

  \item {\bf Active (A)}\index{molecule!active} \\
Several commands and actions in \VMD\ operate on many molecules.
These commands, unless specifically specified otherwise, will do their
action for all the {\em active} molecules.  The primary use for this
control is to prevent some molecules from being animated.  Inactive
molecules will not animate when the play button is pressed.

  \item {\bf Drawn (D)}\index{molecule!drawn} \\
If a molecule is \index{drawn}{\em Drawn} then it is being displayed
in the graphics display window.  This is useful for
temporarily hiding a molecule from view without deleting it.

  \item {\bf Fixed (F)}\index{molecule!fixed} \\
{\em  Fixed} molecules do not undergo rotation,
translation, or scaling.  Note that while it may seem that one
molecule has been moved relative to another, the difference is only
apparent.  The internal coordinates do not change when a standard rotation
is applied by using, for example, the mouse.  It is possible, however, to
change the coordinates of atoms in a molecule, using the text command
interface, and by using the atom move picking modes.

\end{itemize}


\subsubsection{Changing the Molecule's Status}
\index{molecule!status!changing}

The status of a given molecule can be changed by selecting the molecule in
the browser and double-clicking the appropriate flag. Only one molecule can
be top at any one time, so the previous top molecule will change status when
another is toggled.

\subsubsection{Saving Trajectory Frames}
\label{ug:ui:window:edit:write}
\index{animation!write}
\index{file types!output}
\index{files!writing}
\index{trajectory!write}

Using the {\sf Save Coordinates\ldots} menu item, you can write trajectory frames to 
a file in one of several file formats including PDB, DCD, Amber CRD, etc.
This feature may be used to write out a new trajectory in a single file 
after assembling many frames from different sources 
(such as PDB CRD, DCD or Gromacs files, or even from a remote
simulation).  You can also use this, in combination with the 
molecule file browser as a way to make PDB files from a DCD/CRD trajectory.

You can either save the entire stored trajectory, or a slice of
the data by using the
\hyperref{{\sf Amount} chooser}{{\sf Amount} chooser [\S~}{]}{ug:ui:window:edit:amount}.
Then select the appropriate output file type in the {\sf File Type}
chooser, and press the {\sf Save} button in the bottom right corner.
This brings up the file browser, which you can use to enter the new
filename.  Once you press the {\sf Save}
button in the browser, the file will be written without further
confirmation. See the section on the
\hyperref{\tt atomselect writexxx}{{\tt atomselect writexxx} [\S~}{]}{ug:ui:text:atomselect:writexxx}
command for information on how to write atom coordinates for an atom selection
in a PDB file.

\subsubsection{Deleting Trajectory Frames}
\label{ug:ui:window:main:delete:frames}
\index{animation!delete}
\index{frame!delete}

You can delete frames from memory through a dialog box. To bring it up, start by
 selecting a molecule and choosing the 
{\sf Delete Frames\ldots} from the {\sf Molecule} menu, or by double-clicking 
on the {\sf Frames} column for that molecule in the Molecule Browser.
On this is done, choose the
range of frames you wish to delete with the {\sf First} and {\sf Last} controls, 
and then press the {\sf Delete} button.  There is
no confirmation of deletions.

The {\sf Stride} control allows you to keep some frames in the range using the 
specified interval. For example, if your range contains 10 frames labeled 0 
through 9, and you use a stride of 4, the frames numbered 0, 4 and 8 will be kept.
A stride of 0 (zero) implies that all frames will be deleted.


\subsubsection{Deleting a Molecule}
\label{ug:ui:window:main:delete:molecule}
\index{molecule!deleting}

The {\sf Delete Molecule} menu item deletes all the selected molecules.  There
is no prompt verifying the deletion, so take some care.  If a deleted molecule
was the top molecule, a new top molecule will be set from the remaining
structures.

\subsubsection{GUI Shortcuts}
\label{ug:ui:window:main:guishortcuts}
\index{window!main}

There are a few useful mouse-based shortcuts that can be used in the Molecule List browser. Here is a list:

\begin{itemize}
\item Double-clicking on a molecule's name brings up the Rename Molecule dialog box.

\item Double-clicking on a molecule's number of frames brings up the Delete Frames dialog box.

\item Triple-clicking on the T (top) in front of a molecule focusses on that molecule by making it the only molecule to be displayed (D) and active (A). Furthermore, the view is reset and the molecule gets selected in the Representations window.
\end{itemize}
 

\subsection{Main Window Animation Controls}
\label{ug:ui:window:animate}
\index{animate!window controls}

\begin{rawhtml}
<CENTER>
\end{rawhtml}
\myfigure{ug_animate}{The Main window animation controls}{fig:ug:animate}
\begin{rawhtml}
</CENTER>
\end{rawhtml}

Each molecule in \VMD\ can contain multiple sets of atomic coordinates, 
which may be animated to show its motion over time.  
The coordinate sets can come from a molecular dynamics simulation, 
or simply multiple versions of the same molecular structure.  
The Main window contains controls for animated playback of these trajectories.
The controls contains several buttons which act like the
buttons on a VCR or DVD player.  The buttons provide a way to play
the trajectory, step forward, stop, go to a specific frame, and go to 
the beginning or end.
The status and frame counters shown in the animation control reflects 
the state of the {\em top} molecule.  
Commands entered via this control, however, affect all 
\index{molecule!active}
\hyperref{active molecules}{active molecules [\S }{]}{ug:ui:window:mol:top},
not just the top molecule, allowing concurrent animation of multiple molecules. 

\subsubsection{Animation Speed}
\index{animation!speed}
\index{animation!step}
The rate of playback can be controlled in two ways.  The
{\sf Step} control changes the animation step size.  By default, the frame
step is 1, so each step of the playback increases (or decreases) the
animation frame number by one.  If the frame step is 5 then the animation
proceeds five times faster because only a fifth of the frames are shown.
The {\sf Speed} slider at the bottom of the window also affects the
playback speed.  Internally, this controls how many screen updates are
needed between each step.  By default, the slider is at the far right
indicating that one step is performed for each screen redraw.  Moving the
slider to the left increases the minimum time required between updates.

\subsubsection{Jumping to Specific Frames}
\index{animation!jump}
The start and end buttons are used to simplify the
comparison between the initial and final structures.
The start button resets the current animation to the first frame, 
and end jumps to the last frame.  
If you need to jump to a specific frame, enter the
frame number in the frame counter text area next to the start button and 
press enter.
One thing to bear in mind is that the frame number starts at 0, so to jump
to the 5th frame, you must actually enter 4 here.
The animation controls are all relative to the 
\hyperref{top molecule}{top molecule [\S }{]}{ug:ui:window:mol:top}.
\index{molecule!top}

\subsubsection{Looping Styles}
\index{animation!style!loop}
\index{animation!style!once}
\index{animation!style!rock}
When the animation is playing forward and reaches the end of
the data available for the top molecule, one of three possible actions
takes place, as specified in the style chooser.  The default is
`Loop', which will reset the active molecules to the first frame and
continue playing forward.  `Once' will stop the animation when it
reaches the last frame, and \index{rock}`Rock' reverses the direction of
animation.  The actions are symmetrical when the animation is playing
in reverse.


%%%%%%%%%%%%%%%%%%%%%%%%%%%%%%%%%%%%%%%%%%%%%%%%%%%%%%%%%%%%%%%%%%%%%%%%%%%%
% RCS INFORMATION:
%
%       $RCSfile: ug_form_file.tex,v $
%       $Author: johns $        $Locker:  $                $State: Exp $
%       $Revision: 1.23 $      $Date: 2012/01/10 19:30:04 $
%
%%%%%%%%%%%%%%%%%%%%%%%%%%%%%%%%%%%%%%%%%%%%%%%%%%%%%%%%%%%%%%%%%%%%%%%%%%%%
% DESCRIPTION:
%  window for reading in files
%
%%%%%%%%%%%%%%%%%%%%%%%%%%%%%%%%%%%%%%%%%%%%%%%%%%%%%%%%%%%%%%%%%%%%%%%%%%%%


\subsection{Molecule File Browser Window}
\label{ug:ui:window:files}
\index{window!molecule file browser}
\index{files!reading}

\begin{rawhtml}
<CENTER>
\end{rawhtml}
\myfigure{ug_files}{The Molecule File Browser window}{fig:ug:files}
\begin{rawhtml}
</CENTER>
\end{rawhtml}

The {\sf Files} window is used to load a file from disk into a new or 
existing \VMD\  molecule. It can be brought up by choosing {\sf New 
Molecule\ldots} from the {\sf File} menu, or by hilighting a molecule in 
the \hyperref{{\sf Main} window}{{\sf Main} window}{}{ug:ui:window:main} and 
choosing the {\sf Load Data Into Molecule\ldots} menu item. 
Once the window appears, select the file you want by using the file 
browser or by typing the filename into the text entry area.
By default \VMD\ will try to guess the type of file you are loading
by matching the filename extension with one of the file reader plugins 
in the file type list (the available file types are described 
in Chapter \ref{ug:topic:filetypes}).
If \VMD\ is unable to guess the appropriate file type or guesses
incorrectly, you must select it from the list manually.

You can control into which \VMD\ molecule you want to load your data by 
selecting it from the {\sf Load files for:} popup menu at the top of the 
window.
If the file being loaded is intended for a new molecule, select {\sf New 
Molecule} instead. If the file being loaded
contains additional coordinate frames, electron density map, or other
ancillary data for an existing molecule, choose the appropriate molecule from
the selection list at the top of the window.  
If the file being loaded contains trajectory \timesteps, you have the 
option of loading a subset of the trajectory skipping ranges or strides
of \timesteps rather than the whole thing.  You can also select for \VMD\
to load all \timesteps before continuing on, or to load them in the background
so that you may continue to interact with the menus and windows while it loads
additional \timesteps.  
If the file being loaded contains multiple volumetric data, you may 
select which data sets you would like to load.

Once you have selected the file to be loaded, the appropriate file type,
and the way it will be loaded, press the {\sf Load} button and \VMD\ will
being loading the selected file.  Any informational messages, 
errors or warnings which occur while loading the file will appear 
in the text window.


\subsubsection{Reading Trajectory Frames}
\label{ug:ui:window:edit:read}
\index{animation!read}
\index{animation!appending}
\index{files!reading}
\index{file types!input}
\index{trajectory!read}
\index{AMBER!files}
\index{CHARMM!files}
\index{NAMD!files}
\index{Gromacs!files}
\index{XPLOR!files}
\VMD\ can read in new coordinate sets from one of several file
formats such as PDB, CRD, DCD, or Gromacs files.
The new coordinate sets are appended to the end of the
stored \timesteps for the selected molecule.  
Loading coordinate data is like loading any other file, select it
with the file browser make sure the file type is set correctly for
the file being loaded, and then press the {\sf Load} button.

By default, VMD will load all of the \timesteps contained in a 
coordinate or trajectory file. 

Sometimes you may not want to read in a whole coordinate or
trajectory file.  For example, you may only want the last frame, 
or every tenth frame.  You can do this by changing the options in the 
{\sf \Timesteps} control of Files window.
\label{ug:ui:window:edit:amount}
\index{animation!\timesteps}
The {\sf \Timesteps} controls consist of three numeric input fields 
labeled {\sf First}, {\sf Last}, and {\sf Stride}.  These make it
possible to use a subset of the frames, starting at frame {\sf First}
and selecting every {\sf Stride} frames until the {\sf Last} is reached.
For instance, to select every fifth frame between frames 14 and 98,
set:

\begin{itemize}
  \item{{\sf First} to 14}
  \item{{\sf Last} to 98}
  \item{{\sf Stride} to 5}
\end{itemize}

(Remember that frame numbers in \VMD\ start at 0, so frame 0 is the
first frame.)  The value `-1' is a special number; setting {\sf First}
to -1 is the same as starting at the first frame, {\sf Last} = -1 is
the same as ending at the last frame, and {\sf Stride} = -1 is the same
as taking one step.



\subsection{Mouse Menu}
\label{ug:ui:window:mouse}
\index{window!mouse menu}

\begin{rawhtml}
<CENTER>
\end{rawhtml}
% \myfigure{ug_mouse}{Main Window Mouse Menu}{fig:ug:mouse}
\begin{rawhtml}
</CENTER>
\end{rawhtml}

The {\sf Mouse} menu indicates and controls the behavior of the mouse
when the mouse moves and clicks within the graphics window.  Mouse clicks
and drags can affect VMD in one of two ways.  It can change the {\em view}
of the scene, either by rotating, translating, or scaling.  It can also
{\em pick} objects in the scene, causing some further action to be taken.
These behaviors are all reflected in the state of the Mouse menu.  
            
Below, we describe the main parts of the Mouse menu.

\subsubsection{Mouse modes}
\index{mouse!mouse mode}

The top three menu items select whether the mouse will 
rotate, translate, or scale the scene
when the user clicks and drags with the left mouse button.  

\subsubsection{Pick modes}
\index{mouse!object menus}

These modes, located right below the mouse modes in the Mouse menu, 
control how the mouse affects objects in the scene (as opposed
to how the mouse changes the {\em view} of these objects).  Note that
any time you choose a new pick mode, the current mouse mode changes to "Rotate".  

\begin{itemize}
\item {\bf Center} changes how VMD rotates and scales the scene.
To get a feel for how this works, select "Center" from the Mouse menu, then
click on an atom in the scene.  If you now rotate the scene by clicking
and dragging with the left mouse button, the scene should rotate about the
picked atom.  If you change the view mode to "Scale" using the "View Mode"
pulldown menu, the scene will expand while keeping the picked atom in view.
The picked atom will remain the center atom until a new atom is selected as
"Center", the "Reset View" button is pressed, or a new molecule is loaded.

\item {\bf Query} prints information about the item (e.g. the atom name) on the
text console window.

\item {\bf Label} adds labels to atoms in the scene.  Labels
include atoms, bonds, angles, and dihedrals.  These labels require, 
respectively, one, two, three, and four atoms to be picked. For the latter 
three label types, the numerical value of the geometric label is displayed, 
along with a stippled line connecting the picked atoms.  The units for 
"Bonds" corresponds to whatever units the coordinate file is written in. 
"Angles" and "Dihedrals" are measure in degrees.  

Labels can then be manipulated through the {\sf Labels} window.
 
\index{mouse!move}
\index{mouse!move!atom}
\index{mouse!move!residue}
\index{mouse!move!fragment}
\index{mouse!move!molecule}
\index{mouse!move!highlighted rep}
\item{\bf Move} changes the actual coordinates of atoms in the scene.  Note
that this is different from simply changing the view.  Clicking on one of
the buttons in the Mode Mode menu selects what group of atoms to move.  "Atom"
moves only the selected atom.  "Residue" moves all atoms in the same residue
(e.g., amino acid or nucleotide) as the selected atom.  "Fragment" moves all 
atoms connected by a bond to the picked atom.  "Molecule" moves
every atom in the molecular structure.  "Highlighted Rep" is the most 
flexible; it moves all atoms in the highlighted representation in the
browser window of the Graphics window.  

Atoms are moved by clicking and dragging with the left mouse button.  If
the {\sf shift} key is held while the mouse is moved, the affected atoms are
{\em rotated} about the selected atom.   Rotating atoms with the left button
rotates about the x or y axis of the screen; rotating with the middle or
right button rotates about an axis perpendicular to the screen.

Note that there is currently no way to undo Move operations, so the 
atom coordinates should first be saved to a file. 

\item{\bf Force} applies a force to selected atoms in a running simualtion. 
These forces will
be visible only if an IMD connection has been established.  Clicking and
dragging with the left mouse button will apply a force to the selected Atom,
Residue, or Fragment, as in Move Mode.  Clicking with the middle or right
button will cancel the force on the selected atoms.

\item{\bf Move Light} allows the lights to be positioned around the scene.  
Individual lights are turned on or off in the Display window.  Selecting
one of the lights in the Move Light menu rotates the selected light about
the origin. The Move Light Mode can also be cancelled by changing into any 
other pick mode or mouse mode.

\item {\bf Add/Remove Bonds} adds a bond between two clicked atoms if there is
not one present, and removes the bond otherwise. Both atoms must belong to the
same molecule.

\end{itemize}
 

 


%%%%%%%%%%%%%%%%%%%%%%%%%%%%%%%%%%%%%%%%%%%%%%%%%%%%%%%%%%%%%%%%%%%%%%%%%%%%
% RCS INFORMATION:
%
%       $RCSfile: ug_form_display.tex,v $
%       $Author: johns $        $Locker:  $                $State: Exp $
%       $Revision: 1.30 $      $Date: 2014/12/29 03:13:32 $
%
%%%%%%%%%%%%%%%%%%%%%%%%%%%%%%%%%%%%%%%%%%%%%%%%%%%%%%%%%%%%%%%%%%%%%%%%%%%%
% DESCRIPTION:
%  the Display window 
%
%%%%%%%%%%%%%%%%%%%%%%%%%%%%%%%%%%%%%%%%%%%%%%%%%%%%%%%%%%%%%%%%%%%%%%%%%%%%


\subsection{Display Menu and Display Settings Window}
\label{ug:ui:window:display}
\index{window!display}
\index{display!window}

\begin{rawhtml}
<CENTER>
\end{rawhtml}
\myfigure{ug_display}{The Display menu}{fig:ug:display}
\begin{rawhtml}
</CENTER>
\end{rawhtml}

The {\sf Display} menu controls many of the characteristics of the graphics
display window.  The characteristics which may be modified include:
\begin{itemize}
  \item{\bf Reset View}\index{reset view} --
This menu item can be used to force \VMD\ to reset the scene back to the
default viewing orientation and scale as is done when a molecule is 
first loaded.
  
  \item{\bf Stop Rotation}\index{stop rotation} --
This menu item stops autorotation of the scene.  The scene can be 
autorotated by quickly dragging the mouse while briefly depressing 
and releaseing the mouse button, leaving the scene spinning until it
is stopped either by this menu item or by further mouse interactions. 

  \item{\bf Perspective}\index{perspective view} --
The view of the scene can be {\sf Perspective} or {\sf Orthographic}. 
\index{orthographic view}
In the perspective view (the default), objects which are far away are
smaller than those near by.  In the orthographic view, both objects
appear at the same scale.  Note that several of the supported external
rendering programs do not support orthographic rendering.  As such, it may
be necessary to ``fake it'' by translating the scene far away from the 
camera, and apply a zoom factor.  This has the effect of significantly
reducing the perspective, while not truly an orthographic view.

  \item {\bf Antialiasing}\index{antialiasing} --
Turns antialiasing on or off.  Antialiasing helps smooth out
the jagged appearance of displayed geometry resulting from the inherently 
discrete pixels on the display device.  The antialiasing feature is 
only available on platforms which support full-screen antialiasing,
sometimes known as ``multisample antialiasing''.  Several SGI and Sun
graphics systems fully support this feature.  
On platforms lacking the multisample
capability, there may be alternate ways to perform full-screen antialiasing
by selecting an option in the display driver setup.  Windows machines most
commonly place these controls in the display driver configuration panel.

  \item {\bf Depth Cueing}\index{depth cueing} --
Turns depth cueing on or off.  Depth cueing causes distant objects to 
blend into the background color, in order to aid in 3-D depth perception.
The depth cueing settings controlled in the {Display Settings} window.
The {\bf Cue Mode} parameter controls which type of fog equation is
used.  The {\bf Linear} depth cueing mode provides a simple depth
gradient with a defined starting point and endpoint.  The 
{\bf Exp} and {\bf Exp2} depth cueing modes take a density parameter,
and generally blend into the background color much more sharply than
the linear depth cueing mode.
Scaling up the molecule will increase the amount of depth cueing
effect that is visible, since it will occupy a larger depth range.
Scaling the molecule size down decreases the depth cueing effect.  
Translating the molecule into and out of the screen will cause it
to blend into and out of the background color.

  \item {\bf Culling}\index{backface culling} --
Turns backface culling on or off.  This feature is primarily used
to accelerate rendering performance on software based implementations
of OpenGL, such as Mesa.  Backface culling actually reduces performance
on some hardware renderers, so you'll have to use your own best judgement
on whether or not it is helpful to use on your specific computer system.

  \item {\bf FPS}\index{frames per second indicator} --
This option enables or disables on-the-fly display of the achieved
\VMD\ rendering frame rate.  The frame rate is displayed in the upper
right hand corner of the graphics window when it is enabled.

  \item {\bf Lights}\index{light!toggle} --
\index{image!lighting controls}
The graphics display window can use up to four separate light sources
to add a realistic effect to displayed graphical objects.
The {\sf Lights On} browser turns these light sources on or off.  If the
number is highlighted, the light is on, and clicking on it turns the
light off.  See 
section \ref{ug:ui:disp:lights}
for more discussion regarding lights.

  \item {\bf Axes}\index{axes} --
A set of XYZ axes may be displayed at any one of five places on
the screen (each of the corners or the center) or turned off.  This is
controlled by the {\sf Axes} chooser.

  \item {\bf Background}\index{display!backgroundgradient} --
The display background can either be set to a uniform color over
the entire scene, or a vertical gradient can be set with a linearly
changing color from the top of the viewport to the bottom.

  \item {\bf Stage}\index{stage} --
The {\sf Stage} browser controls the stage, which is a checkerboard plane
that can be located in any one of six places or turned off.

  \item {\bf Stereo, Eye Sep, and Focal Length}\index{stereo!parameters} --
\label{ug:ui:window:stereo}
These controls are found in the {\sf Display Settings} window.
These controls set the stereo mode and parameters; stereo is discussed
fully in
chapter \ref{ug:topic:stereo}.
The {\sf Stereo} chooser changes the stereo mode, while the {\sf Eye Sep}
and {\sf Focal Length} controls change the eye separation distance and the
focal length, respectively.
The {\sf Stereo Eye Swap} control optionally reverses the left/right eyes
when displaying on projectors or other devices that for one reason or
another don't preserve the correct left/right eye assignments.

  \item {\bf Cachemode}\index{cachemode} --
The {\sf Cachemode} toggle controls whether or not \VMD\ uses
a display list caching mechanism to accelerate rendering of 
static geometry.  This feature can be extremely beneficial for
achieving good interactive display performance on tiled display walls,
and for remote display over a network.  Caching cannot be performed
while animating trajectories, so the performance benefit is only possible
interactive rotation and zooming of static molecular structures.

  \item {\bf Rendermode}\index{rendermode} --
The {\sf Rendermode} chooser controls which low-level rendering method
\VMD\ uses.  The {\bf Normal} rendering mode is the default VMD rendering
algorithm based on standard fixed-function OpenGL.  The {\bf GLSL} 
rendering mode uses OpenGL Programmable Shading Language to implement
real-time ray tracing of spheres, alpha-blended transparency,
and high-quality per-pixel lighting for all geometry.  On machines
with high performance graphics boards supporting programmable shading,
the {\bf GLSL} rendering mode provides quality on par with many of
the external software renderers supported by \VMD\, but at interactive 
display rates.

\item {\bf Clipping Planes (Near Clip and Far Clip)}\index{clipping planes} --
These controls are found in the {\sf Display Settings} window.
Only those parts of the scene between the near and far clipping planes 
are drawn.  
The display clipping planes also set the depth cueing start and endpoints.
Objects at the near clipping plane are distinct and crisp, 
objects at the far clipping plane will be blended into the background.
Clipping planes positions are changed with the {\sf Near Clip} 
and {\sf Far Clip} controls.  It is not possible for the near
clip to be farther away than the far clip.  When using stereo, it may
be useful to set the near clip plane much lower than the default value.
This makes the geometry ``pop out of the screen'' a bit more, 
and can be used for greater dramatic effect.

\item {\bf Screen Height (Hgt) and Distance (Dist)} --
\label{ug:ui:window:screen_size}
\index{screen parameters}
These controls are found in the {\sf Display Settings} window.
The screen height, along with the screen distance, defines the geometry
and position of the display screen relative to the viewer.  The screen
height is the vertical size of the display screen, in `world' coordinates.
Each molecule is initially scaled and translated to fit within a 2 x 2 x 2
box centered at the origin; so the screen height helps determine how large
the molecule appears initially to the viewer.  

The screen distance parameter determines the distance, in `world' coordinates, 
from the origin to the display screen.  If this is zero, the origin of the
coordinate system in which molecules (and all other graphical objects) are
drawn coincides with the center of the display.  If distance is negative
the origin is located between the viewer and the screen, if it is 
positive, the screen is closer to the viewer than the origin.
A negative value puts any stereo image in front of the screen, aiding the
three-dimensional effect; a positive value results in a stereo image that is
behind the screen, a less dramatic effect (but easier to see, for some
people) stereo effect.

\begin{rawhtml}
<CENTER>
\end{rawhtml}
\myhugefigure{screen_params}{Relationship between screen height, distance to origin, and the viewer}{Relationship between screen height
(SCRHEIGHT), screen distance to origin (SCRDIST), and the viewer}
{fig:ug:screen}
\begin{rawhtml}
</CENTER>
\end{rawhtml}

Figure \ref{fig:ug:screen} describes the relationship between the screen
height, the screen distance, and the world coordinate space.


\item {\bf Shadows}\index{Shadows} --
The shadows control enables and disables direct lighting 
shadowing when using the built-in Tachyon CPU or GPU renderers 
or when exporting the VMD molecular scene to external renderers 
that implement shadowing algorithms.
The simple direct lighting model implemented in most renderers yields
shadows that are completely dark, producing a somewhat harsh lighting
quality akin to what would be expected in a desert under full sunlight
with no clouds.  

\item {\bf Ambient Occlusion}\index{Ambient Occlusion} --
The ambient occlusion (AO) lighting control enables the use of 
so-called ambient occlusion indirect lighting when using the built-in
Tachyon CPU or GPU renderers, or external renderers that implement AO.
Images with much higher quality shading can be produced by augmenting 
direct lighting with ambient occlusion or broad angle or indirect 
lighting techniques.
Ambient occlusion lighting emulates the broad lighting effects 
similar to what would be experienced on a cloudy or overcast day with
omnidirectional light arriving on all surfaces.
The AO ambient coefficient controls the strength of the omnidirectional
lighting components.  The AO direct coefficient scales the direct 
lighting contribution associated with the directional and 
positional lights.

\item {\bf Depth of Field (DoF)}\index{Depth of Field}\index{DoF} --
The depth of field (DoF) control enables or disables emulation of 
depth of field focal blur effects associated with fast focal ratio
camera optics and close focus distances.  The depth of field 
implementation provided by the built-in Tachyon ray tracer and most
other renderers yields a plane of perfect focus at a specified distance
from the camera.  The degree of focal blurring with increasing distance
from the plane of perfect focus depends on both the simulated f/stop
and the distance between the plane of perfect focus and the camera.


\end{itemize}






%%% things that are in their own windows 
%%%%%%%%%%%%%%%%%%%%%%%%%%%%%%%%%%%%%%%%%%%%%%%%%%%%%%%%%%%%%%%%%%%%%%%%%%%%
% RCS INFORMATION:
%
%       $RCSfile: ug_form_graphics.tex,v $
%       $Author: johns $        $Locker:  $                $State: Exp $
%       $Revision: 1.40 $      $Date: 2012/01/10 19:30:05 $
%
%%%%%%%%%%%%%%%%%%%%%%%%%%%%%%%%%%%%%%%%%%%%%%%%%%%%%%%%%%%%%%%%%%%%%%%%%%%%
% DESCRIPTION:
%    The graphics form
%
%%%%%%%%%%%%%%%%%%%%%%%%%%%%%%%%%%%%%%%%%%%%%%%%%%%%%%%%%%%%%%%%%%%%%%%%%%%%


\subsection{Graphics Window}
\label{ug:ui:window:graphics}
\index{window!graphics}
\index{graphics!window}

The {\sf Graphical Representations} or ``Graphics'' window controls how 
molecules are drawn.  Molecules are represented by {\it reps}, 
which are defined by four main parameters: the
\hyperref{selection}{selection [\S~}{]}{ug:topic:selections}, the
\hyperref{drawing method}{drawing method [\S~}{]}{ug:topic:drawing}, the 
\hyperref{coloring method}{coloring method [\S~}{]}{ug:topic:coloring},
and the \hyperref{material}{material [\S~}{]}{ug:topic:coloring:materials}.
\index{representation!style}
\index{drawing!method} 
\index{coloring!methods}
\index{material!methods}
\index{selection} 
The selection determines which part of the
molecule is drawn, the drawing method defines which graphical
representation is used, the coloring method gives the the color
of each part of the representation, and the material determines
the effects of lighting, shading, and transparency on the representation.

\subsubsection{Draw Style Tab}
\begin{rawhtml}
<CENTER>
\end{rawhtml}
\myfigure{ug_graphics_draw_style}{The Graphics window (in Draw Style mode)}
{fig:ug:graphics:drawstyle}
\begin{rawhtml}
</CENTER>
\end{rawhtml}

Select a molecule for editing using the `Selected Molecule' chooser
at the top of the window.  The browser below this chooser lists the
reps available for the molecule.  Each line of the browser summarizes 
information about the drawing method, the coloring method, and the
selection.  Below this browser, choosers and a text input filed reflect 
the current state of the rep, and provide controls for changing the
properties of the rep.  Each drawing method has specific controls 
which will appear when it is selected.
\label{ug:ui:window:graphics:controls}
\label{ug:ui:window:graphics:material}
\index{transparency}
When the `ColorID' coloring method is selected,
a text entry box is shown allowing you to specify the index of a color
to use for the selection, which may be a number from 0 to 16.
\index{color!id}

\paragraph{Changing a rep.}
\index{representation!changing}
To change a representation, select it in the representation browser.  
The atom selection for that rep will
appear in the {\sf Selected Atoms} text area and the controls will update
to reflect the current settings.  Changing the settings will immediately
affect the displayed representation if the 
{\sf Apply Changes Automatically} check box is selected.
When it is disabled updates will only occur when the {\sf Apply} button 
is pressed.  
Changing the drawing method brings up method-specific controls and defaults.
If you go back to the previous draw style,
\VMD\ restores any changes that you may have made to the settings.  
Pressing the {\sf Default} button will restore the default settings.
The display will be updated after every change.  

\paragraph{Adding a rep.}
\index{representation!add new}
To add a new \index{representation!adding}representation of the molecule, 
enter the selection into the {\sf Atom Selection} text area (or keep what is
there) and press {\sf Create Rep}.  This adds the representation to the
currently selected molecule.

\paragraph{Deleting a rep.}
\index{representation!deleting}
To delete a \index{representation!deleting}representation, 
select the representation in the browser and press the {\sf Delete} button.  
Bear in mind that this does not delete the molecule, it only deletes 
one of its graphical representations.

\paragraph{Hiding a rep.}
\index{representation!hiding}
To hide a rep, double-click its entry in the browser.
The text will turn pink to indicate that the rep is hidden.
Turn the rep back on by double-clicking again on the same line.
Hidden reps will not recalculate their geometry if the
animation frame changes until the rep is turned back on.

\subsubsection{Selections Tab}
\index{atom!name lists}

\begin{rawhtml}
<CENTER>
\end{rawhtml}
\myfigure{ug_graphics_selection}{The Graphics window (in Selections mode)}
{fig:ug:graphics:selection}
\begin{rawhtml}
</CENTER>
\end{rawhtml}

The {\sf Selections} tab\index{atom!name lists} 
provides access to
browsers which display the lists of atom names, residue names, and so
forth for the selected molecule.
When the {\sf Selections} tab is pressed, several browsers
appear in place of the drawing and coloring method controls.  
These are used to list the available keywords, macros, and values for use in 
selecting atoms for the associated representation.
The top browser lists singlewords and macros such as {\tt all}, {\tt water},
and {\tt hydrophobic}. 
The botton left browser contains a list of the keywords and functions
understood by the
\index{atom!selection!keywords}
\index{selection!keywords}
\hyperref{selection command}{selection command [\S}{]}{ug:topic:selections}.
If a keyword is selected which can take on a value (for instance, {\tt
name} and {\tt index}), then the possible names will be displayed in
the bottom-rightmost browser.  The functions can be
identified by the {\tt (} to the right of the name.  After selecting a
keyword, the right browser will display all the names associated with the
keyword.  For example, selecting {\sf resname} in the left browser will show
all the three-letter residue names known for the selected molecule.

Clicking on a field in the value browser will add it to the selection
text field.  {\it Double} clicking a keyword field adds the keyword
to the text field.  Press {\sf Apply} to actually change the atom selection
for the current rep.  Press {\sf Reset} to restore the atom selection to
its original value.

\index{atom!selection!macros}
The {\sf Selections} tab also shows the atom selection macros that have
been defined.  These macros let you define a commonly used atom selection
as a single word so that it can be inserted into a rep more
conveniently.  Atom selection macros can currently be defined only through the 
\hyperref{Tcl}{Tcl [\S}{]}{ug:ui:text:atomselect} or
\hyperref{Python}{Python [\S}{]}{ug:ui:text:python:atomselect}
text interfaces; see these sections for details.


\subsubsection{Trajectory Tab}

\paragraph{Selection and Color auto-update.} 
\index{representation!auto-update} 
When an atom selection such as {\tt water within 3 of protein} is made, the
atoms in the selection are computed for the current animation frame.  When the
animation frame changes, the selection is not normally recalculated; thus the
displayed atoms may not correspond to those that would be selected if the atom
selection were performed for the new animation frame.  If the {\sf Update
Selection Every Frame} checkbox is highlighted by clicking on the checkbox,
then the atom selection for the current rep will be recalculated every time the
animation frame changes.  Similarly, if the {\sf Update Color Every Frame}
checkbox is activated, the color will be recalculated for every frame. 

\paragraph{Color Scale Data Range.}
Several of the coloring methods available in {\sf Draw Style} tab
operate over data fields that have no specifically implied range of 
values.  It is often useful to highlight a very specific range of data
values, in order to accomplish this the color scale range can be manually
set to a specific starting and ending values, overriding the default 
behavior which is to autoscale from the minimum value to the maximum value.
This feature is particularly useful when displaying trajectories, since
the range of values of interest may be quite different from the autoscaled 
range for a single frame or all frames. 

\paragraph{Draw Multiple Frames.}
\index{trajectory!draw multiple frames}
Draw multiple trajectory frames or coordinate sets simultaneously.
This setting allows the user to select one or more ranges of frames
to display simultaneously.  The frame specification takes one of the
following forms {\bf now}, {\it frame\_number}, {\it start:end}, or
{\it start:step:end}.  

\paragraph{Trajectory Smoothing.}
\index{trajectory!smoothing}
\index{animation!smoothing}
The {\sf Trajectory Smoothing Window Size} is used to control the application
of a per-representation windowed-averaging smoothing function.  This simple
smoothing feature can be used to eliminate much of the thermal noise inherent
in a molecular dynamics trajectory so that one can more easily see structural
changes occuring over a wider time scale.  The window size parameter controls
how many \timesteps are averaged together to produce the coordinates which
are actually displayed.  One important consideration
when using the trajectory smoothing feature is that \VMD\ does not take
periodic boundary conditions into consideration when smoothing trajectory
coordinates, so any atoms which wrap around within the span of the window will
cause erratic motions in the displayed representation.  This can be avoided by
unwrapping trajectory coordinates prior to loading into \VMD\ or by using atom
selections to eliminate atoms which wrap around.


\subsubsection{Periodic Tab}
\index{periodic image display}
\index{periodic boundary conditions}
\index{unit cell information}
The {\sf Periodic} tab controls the display of periodic images of a
molecule.  In order to display periodic images, a molecule must have
unit cell information set for {\tt a}, {\tt b}, {\tt c}, 
{\tt alpha}, {\tt beta}, and {\tt gamma}, which are discussed in 
section~\ref{ug:topic:molinfo}.  
When the proper unit cell information is present, the periodic display
feature can show periodic images of the unit cell by transforming and 
rendering additional copies of the structure.  The current implementation
of this feature doesn't provide for complex crystallographic symmetry 
operations.  Unit cells that can be replicated by translation along the 
three unit cell axes are the only ones supported presently.  The periodic
images to be drawn are selected by enabling images in one or more of 
the six faces of the unit cell.  The {\sf Self} image selects the 
untranslated unit cell itself, so that one my render a representation
consisting of only replica images.  This feature allows the unit cell
and its periodic images to be displayed using different materials, for
cases where it is desirable to draw more attention to the original unit
cell or to one ore more of the replicas.  The {\sf Number of Images}
counter controls how many replicas are made in each of the six directions.
Some file formats read by \VMD\ may not include unit cell information, 
in such cases you can use the scripting interface to set the unit cell 
information manually.  {PDB} files containing {CRYST1} records are an 
example of a file format that provides unit cell information.



%%%%%%%%%%%%%%%%%%%%%%%%%%%%%%%%%%%%%%%%%%%%%%%%%%%%%%%%%%%%%%%%%%%%%%%%%%%%
% RCS INFORMATION:
%
%       $RCSfile: ug_form_labels.tex,v $
%       $Author: johns $        $Locker:  $                $State: Exp $
%       $Revision: 1.17 $      $Date: 2012/01/10 19:30:05 $
%
%%%%%%%%%%%%%%%%%%%%%%%%%%%%%%%%%%%%%%%%%%%%%%%%%%%%%%%%%%%%%%%%%%%%%%%%%%%%
% DESCRIPTION:
%  labels form
%
%%%%%%%%%%%%%%%%%%%%%%%%%%%%%%%%%%%%%%%%%%%%%%%%%%%%%%%%%%%%%%%%%%%%%%%%%%%%


\subsection{Labels Window}
\label{ug:ui:window:labels}
\index{window!label}
\index{labels!window}

\begin{rawhtml}
<CENTER>
\end{rawhtml}
\myfigure{ug_labels}{The Labels window}{fig:ug:labels}
\begin{rawhtml}
</CENTER>
\end{rawhtml}

The Labels window is used to manipulate the labels which may be placed on
atoms, and the geometry monitors which may be placed between atoms.
Labels are selected with a mouse, as discussed in section 
\ref{ug:ui:disp:pick}.  Once selected, the {\sf Labels}
window can be used to turn different labels on or off or to delete them
entirely.  Also, labels displaying geometrical data such as bond
lengths may be graphically displayed using this window.


\subsubsection{Label categories}
\index{labels!categories}

The {\sf Category} chooser (in the upper left) is used to select which
category of labels to manipulate.  The different label categories include:
\begin{itemize}
  \item Atoms, which are shown as a text string next to the atom listing
the name and residue of the atom;
  \item Bonds, which are shown as dotted lines between the atoms with the
bond length displayed at the bond midpoint;\index{bonds!label}
  \item Angles, which are shown as dotted lines between the three atoms with
the angle displayed at the center of the defined triangle;\index{angles}
  \item Dihedrals, which are shown as dotted lines between the four atoms
with the dihedral angle (the angle between the planes formed by the first
three atoms and the last three atoms) shown at the midpoint of the 
torsional bond.
  \item Springs, which are shown as dotted lines between the atoms with the
bond length displayed at the bond midpoint;\index{bonds!label}
\end{itemize}
All the labels for the selected category which have been previously
added are displayed in the browser in the center of the window.  The
line itself contains from 1 to 4 atom names, depending on the
category; the atom names have the form
\begin{centering}
{\tt <residue name><residue id>:<atom name>}
\end{centering}
followed by either {\tt (on)} or {\tt (off)}.  The last word
indicates if the label is turned on or off.


\subsubsection{Modifying or deleting a label} 
\index{labels!hide}
\index{labels!show}
\index{labels!delete}

A label can be turned on or off without deleting it, by selecting the
label in the central browser and pressing the {\sf Hide} button.  To
turn it back on, select it again then press the {\sf Show} button.
Press the {\sf Delete} button to delete it.  This browser allows
multiple selections, which, for example, allows you to delete several
labels at once.  To select everything in the current category, press
{\sf Select All}; to unselect them, press {\sf Unselect All}.  If
nothing is selected, the action is applied to everything.  Thus, one
way to turn everything off is to press {\sf Unselect All} then press
{\sf Hide}.  (It may seem counterintuitive, but it was done this way
so all the labels could be deleted by just pressing {\sf Delete}.)

\subsubsection{Pick information}
\index{mouse!pick information}

The {\sf Picked Atom} tab displays
information about the last atom picked by the mouse.  This information
is also echoed to the vmd console.  The data in the will
remain in until a new label is selected by the mouse.
Information about the following fields is identified:

\begin{itemize}
\item Molecule - the name of the molecule referenced
\item XYZ - the position of the atom in 3D space
\item Resname - the type of the amino or nucleic acid to which this atom
belongs
\item ResID - the internal VMD ID number of the entire residue to which
the particular atom belongs. E.g., ResId for an atom of a protein is the
same as the residue number of that atom as listed in its PDB file.
\item Name - the name of the atom as it appeared in the coordinate file
\item Type - the type of the atom, as determined by an internal VMD match-up
of the given name to a likely atom type associated with that name
\item Chain - if the coordinate file contained data in the ``Chain'' field
for this atom, then that data is given here.
\item Segname - the name of the segment to which this atom belongs
\item Index - the internal VMD index used to identify the atom; this is
useful for specifying selection syntax to generate different
representation styles for particular atoms. For PDB files Index corresponds
to the atom number listed in the file minus 1 (so that the index starts with
0).
\item Value - the calculated length of bonds, angles, or 
      geometric measurements performed by the selected label
\end{itemize}


\subsubsection{Plotting a label's value}
\label{ug:ui:window:labels:plotting}
\index{labels!plotting}
\index{atoms!distance between!plotting}
\index{plot!geometry monitors}

If the label has a numeric value (such as a bond length geometry
monitor), it is easy to graph the change of the value over time (for
multiple frames in an animation).  The {\sf Graph} button calls a
Tcl script to plot the data for the selected labels.  You can create
your own script to handle label plotting simply by creating a Tcl proc
named {\tt vmd\_labelcb\_user}.  The proc should accept three arguments.
Have a look at the default scripts in the VMD scripts directory, found in
the VMD installation directory under {\tt scripts/vmd/graphlabels.tcl}.
If no supported graphing program is available, a dialog box will be
presented which will allow you to save the values of the labels to a file.

%%%%%%%%%%%%%%%%%%%%%%%%%%%%%%%%%%%%%%%%%%%%%%%%%%%%%%%%%%%%%%%%%%%%%%%%%%%%
% RCS INFORMATION:
%
%       $RCSfile: ug_form_color.tex,v $
%       $Author: johns $        $Locker:  $                $State: Exp $
%       $Revision: 1.17 $      $Date: 2012/01/10 19:30:04 $
%
%%%%%%%%%%%%%%%%%%%%%%%%%%%%%%%%%%%%%%%%%%%%%%%%%%%%%%%%%%%%%%%%%%%%%%%%%%%%
% DESCRIPTION:
% the Color window
%
%%%%%%%%%%%%%%%%%%%%%%%%%%%%%%%%%%%%%%%%%%%%%%%%%%%%%%%%%%%%%%%%%%%%%%%%%%%%


\subsection{Color Window}
\label{ug:ui:window:color}
\index{window!color}
\index{color!window}

\begin{rawhtml}
<CENTER>
\end{rawhtml}
\myfigure{ug_color}{The Color window}{fig:ug:color}
\begin{rawhtml}
</CENTER>
\end{rawhtml}

\VMD\ maintains a database of the colors used for the molecules and the
other graphical objects in the display window.  The database consists
of several color \index{color!category}{\em categories}; each color
category contains a list of names, and each name is assigned a color.
The assignment of colors to names can be changed with this window.
There are 16 colors, as well as black (the \VMD\ color
\index{color!map}{\em map}), and this window can also be used to modify
the definitions of these 17 colors.  For more about colors, see the
chapter on
\hyperref{Coloring}{Coloring [\S }{]}{ug:topic:coloring}.

To see the names associated with a color category, click on the
category in the {\sf Category} browser located on the left side of the
window.  Click on the name to see the color to which it is mapped.  To
change the mapping, click on a new color in the browser to the right
of the {\sf Category} browser.  For instance, to change the
\index{color!background}background to white, pick `Display' in the
left browser and `Background' in the center one.  The right browser
will indicate the current color (which is initially {\tt black} for
the background).  Scroll through the right browser and select {\tt
white} to change the background.


\subsubsection{Changing the RGB Value of a Color}
\index{color!redefinition}

The {\sf Color Definitions} tab at the bottom of the Color menu lets
you change the RGB definition of the 17 palette colors.  Select a color
to edit using the browser at the bottom left corner of the menu, then
slide the three sliders to set the amount of each red, green and blue
component.  {\sf Default} restores the original color definition, and
{\sf Grayscale} toggles whether or not the three sliders will move 
together as a unit.  Color definitions are immediately updated in the
graphics window, so you can see the result of your editing right away.

\subsubsection{Color Scale}
\label{ug:ui:window:color:scale}
\index{color!scale!changing}

Several of the coloring methods in the graphics window (e.g., Beta, 
Index, Position) are used to color
a range of values, as opposed to a list of names.  The actual
coloring is determined by the \index{color!scale} \hyperref{color
scale}{color scale [\S }{]}{ug:topic:coloring:scale}.

The color scale used to assign these colors is set in the {\sf Color Scale}
tab of the color menu.  Choose one of the ten color scales from the
chooser, and adjust the {\sf Offset} and {\sf Midpoint} sliders until
the color scale shown at the bottom of the tab is as desired.  



\subsection{Material Window}
\label{ug:ui:window:material}
\index{window!material}
\index{image!shading and material properties}

\begin{rawhtml}
<CENTER>
\end{rawhtml}
\myfigure{ug_material}{The Material Window}{fig:ug:window}
\begin{rawhtml}
</CENTER>
\end{rawhtml}

This window is used to create and modify material definitions.  The material
definitions created here will show up in the pulldown menu in the Graphics
window, allowing you to apply a material to a given representation.  

The upper left corner of the Materials window contains a browser listing all the
currently defined materials.  Below this browser is a set of five sliders which
indicate the current materials settings for the material highlighted in the
browser.  Highlighting a different material in the browser by clicking with the
mouse will update the settings of the sliders.  Conversely, moving the sliders
will change the definition of the the currently highlighted material in the
browser.  Pressing the "Default" button will restore either of the first two
materials, "Opaque" and "Transparent", to their original settings.

To create a new material, press the "Create New" button in the upper right
corner of the window.  A new material with a default name will be created and
displayed in the browser window.  This name can be changed at any time to
something more descriptive by typing in the input box to the right of the
material browser and pressing "enter" (note that the names of "Opaque" and
"Transparent" cannot be changed).  You can now edit the properties of this
material using the sliders at the bottom of the window.  All materials in the
materials browser, including those you create, will appear in the Material
pulldown menu in the Graphics window. 

To experiment with the material settings, first create a new material so
that you can edit its values.  Next, load any molecule, change its 
drawing method to VDW representation, and using the Material pulldown menu
in the Graphics window, change the representation's material to the 
material you just created.  Now, go back to the Materials window, highlight
the new material in the browser, and change some of the values in the sliders.
The effect of changing shininess should be especially dramatic.


%%%%%%%%%%%%%%%%%%%%%%%%%%%%%%%%%%%%%%%%%%%%%%%%%%%%%%%%%%%%%%%%%%%%%%%%%%%%
% RCS INFORMATION:
%
%       $RCSfile: ug_form_render.tex,v $
%       $Author: johns $        $Locker:  $                $State: Exp $
%       $Revision: 1.21 $      $Date: 2012/01/10 19:30:05 $
%
%%%%%%%%%%%%%%%%%%%%%%%%%%%%%%%%%%%%%%%%%%%%%%%%%%%%%%%%%%%%%%%%%%%%%%%%%%%%
% DESCRIPTION:
%  render form
%
%%%%%%%%%%%%%%%%%%%%%%%%%%%%%%%%%%%%%%%%%%%%%%%%%%%%%%%%%%%%%%%%%%%%%%%%%%%%


\subsection{Render Window}
\label{ug:ui:window:render}
\index{window!render}
\index{render!window}
\index{rendering}

The Render window is used to export the currently displayed graphics scene to
an image file or to a geometric scene description file suitable for use
by one of several external renderers, which can produce a final image.
The supported rendering packages are listed in table \ref{ug:table:render}.  
See Chapter \ref{ug:topic:rendering}
for detailed information on how rendering is performed using 
external programs, as well as information on 3-D printing and 
other uses of the exported scene description files.

\begin{rawhtml}
<CENTER>
\end{rawhtml}
\myfigure{ug_render}{The Render window}{fig:ug:render}
\begin{rawhtml}
</CENTER>
\end{rawhtml}

The rendering process works in two stages.  The first stage
exports the displayed \VMD\ scene to a text or image file
in the selected format.
The second (optional) stage renders the exported file,
potentially displaying the results when complete.  
The exported file is named in the {\sf Filename} field; 
a default name is given when a new format is selected, 
so it is best to hold off entering the filename until after
the file format is selected.  Another way to select the filename is
available by pressing the {\sf Browse}  button, which opens up a file
browser.
Pressing the {\sf Start Rendering} button writes the data file.  After
that, the {\sf Render Command} is executed.  The default command
should start the appropriate rendering program if it is available.

Some of the rendering commands have been set to call a display 
program on the rendered image when it is completed.
\VMD\ will wait for the display program to finish, which causes \VMD\ 
to freeze until the display program closes, 
so you may want to run the job in the background.  This can be done 
(on Unix) by enclosing the existing text with {\tt ()}'s and putting
an {\tt \&} at the end.  For example, the way to make the Raster3D render 
command run in the background is:
\index{rendering!in background process}
{\tt 
\begin{verbatim}
        (render < %s -sgi %s.rgb; ipaste %s.rgb)&
\end{verbatim}
}

%%%%%%%%%%%%%%%%%%%%%%%%%%%%%%%%%%%%%%%%%%%%%%%%%%%%%%%%%%%%%%%%%%%%%%%%%%%%
% RCS INFORMATION:
%
%       $RCSfile: ug_form_tool.tex,v $
%       $Author: johns $        $Locker:  $                $State: Exp $
%       $Revision: 1.12 $      $Date: 2012/01/10 19:30:06 $
%
%%%%%%%%%%%%%%%%%%%%%%%%%%%%%%%%%%%%%%%%%%%%%%%%%%%%%%%%%%%%%%%%%%%%%%%%%%%%
% DESCRIPTION:
%  tracker/ tool control
%
%%%%%%%%%%%%%%%%%%%%%%%%%%%%%%%%%%%%%%%%%%%%%%%%%%%%%%%%%%%%%%%%%%%%%%%%%%%%


\subsection{Tool Window}
\label{ug:ui:window:tool}
\index{window!tool}
\index{tool!window}
\begin{rawhtml}
<CENTER>
\end{rawhtml}
\myfigure{ug_tools}{The Tool window}{fig:ug:tool}
\begin{rawhtml}
</CENTER>
\end{rawhtml}

The {\sf Tool} window is used to set up external 3D pointers,
buttons, force-feedback devices, and the \VMD\ ``tools'' that they
control.  \VMD\ communicates with input devices through
CAVElib, FreeVR, or via Virtual Reality Peripheral Network (VRPN),
or with direct operating system interfaces.  
Since VRPN provides networked device abstraction, \VMD\ doesn't 
have to be running on the same computer that VRPN devices are 
attached to.  With VRPN, you may use buttons, trackers,
and also force-feedback (haptic) devices such as the PHANToM.
In the CAVE or FreeVR, \VMD\ recognizes two types of devices: buttons 
and trackers.  The built-in Spaceball driver can also be used to
control tools.

\subsubsection{Configuring input devices}
To use input devices with \VMD\ ``tools'', you need a 
{\it sensor configuration file},
\index{sensors}
\index{sensor configuration file} 
\index{tool!VRPN}
\index{tool!CAVE}
\index{tool!FreeVR}
\index{tool!Spaceball}
in your home directory called {\tt
.vmdsensors}\index{{\tt .vmdsensors}} (see the VMD Installer Guide).  
In this file, any number of devices
can be specified, using a {\sl universal sensor locator\/}
(USL)\index{USL}\index{universal sensor locator}.  
The format for a USL is as follows:
\\
{\it USL\/} -- {\it type{\tt ://}place{\tt /}name{\tt :}nums}
\begin{itemize}
  \item{\it type\/} -- the type of sensor
({\tt vrpntracker}, {\tt vrpnbuttons}, {\tt vrpnfeedback},
{\tt cavetracker}, {\tt cavebuttons}, or {\tt sballtracker})
  \item{\it place\/} -- the machine that controls it.  Devices that
    cannot yet be used on arbitrary computers over the network must
    have the keyword {\tt local} here to be compatible with future
    versions.
  \item{\it name\/} -- the name of the device within that machine.  If
    multiple devices can't currently exist, such as with the CAVE,
    then a standard name should be used, such as {\tt cave}, so that
    the same USL will make sense in the future, when multiple devices
    are allowed.
  \item{\it nums\/} -- a comma-separated list of numbers of devices
    belonging to that names (optional, defaults to zero).  Some
    devices demand only one number or a specific number but button
    devices should work correctly now.
\end{itemize}
The lines of a sensor configuration file come in four flavors:
\begin{itemize}
  \item {\it Comments\/} begin with {\tt \#} and are ignored.
  \item {\it Empty lines\/} are also ignored.
  \item {\it Device lines\/} have the form {\it {\tt device} name
    USL}, where {\it name\/} is the name that \VMD\ will use to refer
    to the device, and {\it USL\/} is the device's USL.
  \item {\it Options\/} tell \VMD\ how to use the most recently listed
    device.  Currently, there are four supported options:
\begin{itemize}
  \item ``{\tt scale} $x$'' scales the position of a tracker by a
    factor~$x$.
  \item ``{\tt offset} $x$ $y$ $z$'' adds a constant vector to the
    position of a tracker.
  \item ``{\tt rot right}|{\tt left} $A_{00}$ $A_{01}\ldots A_{33}$''
    multiplies the orientation matrix returned by a tracker on either
    the right or the left by the matrix~$A$.
  \item ``{\tt forcescale} $x$'' multiplies the force applied to a
    force-feedback device by the amount~$x$.
\end{itemize}
\end{itemize}

Here is a simple example, showing some of the things you can do with a
sensor configuration file, for a more complete example, please refer
to the .vmdsensors file that came with your \VMD\ distribution:
\begin{verbatim}
### Sensable PHANTOM via VRPN 
### http://www.sensable.com/
### The Phantom haptic device connected to the computer "odessa"
device phantomtracker   vrpntracker://odessa/Phantom0
scale 10
rot left 0 0 -1 0 1 0 1 0 0
device phantombuttons   vrpnbuttons://odessa/Phantom0
device phantomfeedback vrpnfeedback://odessa/Phantom0
\end{verbatim}

\subsubsection{Using Tools}
\label{ug:tools}
\index{tools} 
There are several different ``tools'',
each of which can be used with any of the input
devices\footnote{The tools have been designed to allow \VMD\ to
use haptic devices.  Most of the tools can give force-feedback
to the user, but none of them require haptic devices in order to operate.}:
\begin{itemize}
\item The {\bf Grab Tool} mimics a pair of tweezers, and can be
used to move molecules around on the screen without any keyboard or
mouse commands.  Pressing a button connects the 3d cursor to the
nearest molecule.  Then, moving or rotating the tracker will cause the
molecule to move or rotate around on the screen.

\item The {\bf Rotate Tool} is a tool for precisely rotating
molecules with haptic devices.  When a button is pressed and released,
the cursor is again connected to the molecule.  With this tool,
however, the center of the molecule is fixed, and the end of the
haptic pointer is forced to lie on the surface of a sphere about this
center.  Moving the device around the surface of the sphere rotates
the molecule, and another button click releases the molecule.  There
are detentes --- like the clicks commonly felt in a 2d dial --- on the
surface of the sphere, arranged so that the user can rotate the molecule to
precise 90-degree points.  If the user holds down the button for a
while initially, he can feel the sphere and the detentes, but do not
affect the molecule.  This ``preview mode'' allows the user to find a
good point from which to start the rotation.

\item The {\bf Joystick Tool} is the three-dimensional equivalent of a
Joystick, for haptic devices.  Pressing the button creates a virtual
``spring,'' holding the device to its current location.  If it is
pushed away from this point in some direction, the selected molecule
starts sliding in that direction, with a velocity that is proportional
to the displacement of the device.  The joystick tool shows how a
three dimensional input device can be used to supply relative
(differential) coordinates instead of absolute coordinates.  

\item The {\bf Tug Tool} is a tool that allows interaction with
running molecular dynamics simulations.  Pressing the button connects
the device with a simulated spring to the nearest atom, and pulling on
it adds a force to the simulation.  If a haptic device is being used,
the user will feel a force on his hand that is proportional to this
force.  In this way, the tug tool implements something like the
click-and-drag that is commonly used with windowing systems.

If an atom selection is assigned to the Tools,
the the Tug Tool will apply a force to 
all the atoms in the selection.  The force applied will be proportional to
the masses of the atoms in the selection, so that all atoms experience the
same acceleration.  When a Tool Selection has been assigned, the Tug Tool
will always affect that selection, even if the button is pressed far from
any atoms in the selection; this is intended to make it easier for the
user to apply forces only on those atoms he/she intends to steer. 

\item The {\bf Spring Tool} also allows interaction with running
molecular dynamics simulations.  It works like the Tug Tool {\sl
except\/} that when the button on the tracker is released near an
atom, the simulated spring is connected to it.  See
section~\ref{ug:ui:window:labels} for information on viewing and
modifying the list of active springs.

\item The {\bf Pinch Tool} is similar to the Tug Tool, except that
force is applied only along the axis defined by the orientation of the
tracker.

\item The {\bf Print Tool} is meant to be used as a debugging aid when
one first sets up VMD for use with VRPN, the CAVE, or other 3-D input
devices.  When enabled, this tool prints text messages to the VMD console
indicating the current position of the tool in question.  This tool is 
useful when calibrating the various transformation matrices that operate on 
tracker position and orientation data (whether in VMD or in VRPN, CAVElib, 
etc).

\end{itemize}
%%
%%
To add a new tool to a \VMD\ session, open the {\sf Tool} window and click
the {\bf Create Tool\/} button.  The tool's number and type are
displayed in the list to the left.  Devices can be added to the tool
by selecting them from the {\bf Add Device\/} menu, or removed with
the {\bf Delete Device\/} button.  Some of the options that can be
specified in the sensor configuration file can be edited in using the
controls below, and the tool's type can be changed with the {\bf
Type\/} menu.


\subsection{IMD Connect Simulation Window}
\label{ug:ui:window:sim}
\index{window!IMD simulation}

\VMD\ has the ability to work with a molecular dynamics program
running on another computer, to interact with and display the results 
of a simulation as they are calculated.
A major feature in \VMD\ is the ability to add
perturbative steering forces to a running simulation, which are
incorporated directly into the dynamics calculation;
we refer to this capability as Interactive Molecular Dynamics (IMD).  
\index{imd!requirements}
In order to run and IMD simulation it is necessary to have a molecular 
dynamics program that supports the IMD communication protocol.  
To date, two such programs exist; NAMD, developed at University of Illinois,
and Protomol, developed at Notre Dame.
The rest of the discussion in this chapter assumes you are using \NAMD.
See \htmladdnormallinkfoot{the \NAMD\ home page}
{http://www.ks.uiuc.edu/Research/namd/} for information on obtaining \NAMD.

\subsubsection{Interactive Molecular Dynamics}
\index{IMD}
\index{Interactive Molecular Dynamics}
IMD works by establishing a TCP connection between \VMD\ and the 
molecular dynamics simulation program.  NAMD, or whichever MD program is 
being used, acts as the server. 
In order to prepare \NAMD to accept \VMD's IMD connection request, 
\NAMD must be configured to listen for incoming connections on a network port.
Once NAMD has started up, may wait for the user to connect through 
that port.  When \VMD connects to \NAMD succesfully, the simulation commences.
Before connecting to the remote simulation, the \VMD user must first load a 
molecule corresponding to the system being simulated.  The structure file
should correspond to the same structure file used by \NAMD.
Once the molecule is loaded and NAMD has been started and is listening for a 
connection, you are ready to connect to the simulation and start receiving 
coordinates.  To establish a connection, open the Simulation window, enter 
the hostname
on which NAMD is running and the port on which NAMD is listening for incoming
connections, then press the Connect button to establish the connection.
If NAMD is running on several distributed nodes, VMD must connect to the root 
node on which NAMD initially started out.  

\subsubsection{IMD Using the Simulation window}
The {\sf Simulation} window allows you to control the behavior of a 
molecular dynamics
simulations which has been previously connected to through use of the
Remote window.  This window contains controls to change parameters for the
simulation and to affect how \VMD\ displays the results of the simulation.
The window also contains informative displays, which show the current status
of the simulation connection, and such things as the current energy,
temperature, and timestep of the molecular system being simulated.

At the top of the window are two entry fields and a button for establishing
a connection to a running MD simulation.  Enter both the hostname on which 
the simulation is running, and the port on which the simulation is listening,
then press the Connect button to establish the connection.  See the text 
console for possible error messages and status updates.
Below the connection display is a browser used to set some connection 
parameters.  These include:
\index{remote!connection!modifiable parameters}
\begin{itemize}
  \item {\bf Transfer Rate}: How often a timestep is transferred from the
remote simulation program to \VMD.  By default, this is 1, which means
every calculated timestep is sent.  If this is set to some value N,
then only every Nth step will send from the remote computer, 
thereby decreasing the amount of network processing and rendering 
that needs to be done.

  \item {\bf Keep Rate}: How often \VMD\ saves the timestep in its
animation list, instead of just discarding it after displaying it.  By
default, this is 0, which means that \VMD\ does not save any timesteps.
When this is 0, then when \VMD\ receives a new timestep it {\em replaces} the
last timestep in the animation list with the new timestep, instead of
{\em appending} it.  When it is set to some number N larger than 0, then
every Nth timestep received from the remote simulation will be appended to
the molecule. 
\end{itemize}

Parameters may be changed by entering text into the appropriate entry field
and pressing {\tt <return>}.
When a new value is entered, a command is sent to the remote simulation 
to change it.  There may be some delay between when the simulation gets 
commands, acts on them, and the results propagate back to \VMD.
Connection state is shown in the center of the window.
The simulation status text area displays energy values for the 
system being simulated (kinetic, electrostatic, etc.),
as well as the current timestep and the temperature.  
It is updated each time a new coordinate set (timestep) 
is received by \VMD.
\index{remote!connection!killing}
The {\sf Stop Sim} button will terminate the remote simulation, but
will not delete the molecule in \VMD.
\index{remote!connection!detaching}
The {\sf Detach Sim} button will sever the connection
between \VMD\ and \NAMD, 
but will allow the simulation to continue running.



%%% extensions we'll take the time to document in the main VMD manual
%%%%%%%%%%%%%%%%%%%%%%%%%%%%%%%%%%%%%%%%%%%%%%%%%%%%%%%%%%%%%%%%%%%%%%%%%%%%
% RCS INFORMATION:
%
%       $RCSfile: ug_form_sequence.tex,v $
%       $Author: johns $        $Locker:  $                $State: Exp $
%       $Revision: 1.14 $      $Date: 2012/01/10 19:30:05 $
%
%%%%%%%%%%%%%%%%%%%%%%%%%%%%%%%%%%%%%%%%%%%%%%%%%%%%%%%%%%%%%%%%%%%%%%%%%%%%
% DESCRIPTION:
%  sequence window
%
%%%%%%%%%%%%%%%%%%%%%%%%%%%%%%%%%%%%%%%%%%%%%%%%%%%%%%%%%%%%%%%%%%%%%%%%%%%%


\subsection{Sequence Window}
\label{ug:ui:window:sequence}
\index{window!sequence}
\index{sequence!window}
\index{sequence!dna}

\begin{rawhtml}
<CENTER>
\end{rawhtml}
\myfigure{ug_sequence}{The Sequence window}{fig:ug:sequence}
\begin{rawhtml}
</CENTER>
\end{rawhtml}

The Sequence window is used to list the residue sequences of proteins and the 
base sequences of nucleic acids, and to select residues/bases from the sequence
list for highlighting in the 3-D structure in the main \VMD\ window.   When
residues are selected in the main \VMD\ window, the corresponding residue  is
highlighted in the sequence list in this window.  Color-coded protein structure
information is displayed for amino-acid residues, and B-factor information is
displayed for all residues.  In this section, ``residues'' refers both to amino
acid residues in proteins, and to nucleotide bases with associated backbone in
DNA and RNA molecules. 


\subsubsection {Sequence information}
	\index{sequence}

	The  Sequence window contains a vertical listing of the residue sequence of a loaded molecule.  The {\sf Molecule} pop-up menu control chooses which molecule to display the sequence of, the current 'top' molecule is displayed the first time the Sequence window is opened.  The name and molecule number of the  sequence displayed is shown in the title frame of the Sequence window.
       
   For each residue displayed, the window lists: residue number, residue name/code, and chain letter.  If no chain is specified, chain letter is set to ``X''. To the right of this are two color coded columns, ``B value'' and ``struct''.  ``B-value'' shows the contents of the B-value (temperature factor) field. The ``struct'' field shows protein secondary structure; select {\sf Help:Structure Codes} from the window menu, or see Table~\ref{table:ug:structcodes},  for an explanation of the single letter codes in the color key.



 
\newcommand{\structcode}[2]{
    {\tt #1}&\parbox[t]{2.5in}{#2}\\
}
\begin{table}[htb]
 \hspace{1.8 in}
  \begin{tabular}{|c|l|} \hline
%    \multicolumn{1}{|c}{Code} &
%	\multicolumn{1}{|c|}{Description} \\ \hline\hline
Code & Description\\ \hline\hline
\structcode{T} {Turn}
\structcode{E} {Extended conformation}
\structcode{B} {Isolated bridge}
\structcode{H} {Alpha helix}
\structcode{G} {3-10 helix}
\structcode{I} {Pi-helix}
\structcode{C} {Coil}
\hline
\end{tabular}
	\caption{Description of secondary structure codes in the Sequence window.}
	\label{table:ug:structcodes}
    \index{secondary structure codes}
\end{table}


\subsubsection {Selecting residues from the Sequence window listing}

\index{highlight}
Click anywhere in the vertical listing with the left mouse button to highlight one residue.  Click and drag with the left mouse button
to  highlight multiple residues,  shift-click to add a single residue to the
current selection, shift-click and drag to add multiple residues to your
selection, right-click to de-select a residue.  Highlights appear as thick
yellow ``Bonds'' representations, these can be  \hyperref{changed or turned
off}{changed or turned off[\S~}{]}{ug:ui:window:sequence:offchange}.


\subsubsection {Selecting residues by clicking on the 3-D structure}
Use the {\sf Mouse} menu to enter ``Pick Atom'' mode (or press ``1'', the standard keyboard shortcut).  Click on any protein atom or nucleic acid atom, and its residue will highlight, and the sequence list will scroll to display this residue.  Shift-click works the same way, but adds to the current selection. 
Note that if the zoom factor is smaller than 1.0, the single-residue sequence highlight will be shorter in height than a full line of text. Once the Sequence window has been opened, any ``Pick'' will create or add to selections, until highlighting is  \hyperref{turned off}{turned off[\S~}{]}{ug:ui:window:sequence:offchange}.


\subsubsection {Sequence Zooming }
   \index{sequence!zooming}
	Larger molecules contain thousands of residues, too many to display in a linear text list all at once.
The sequence window can only list about 40 text lines;  to work with a molecule of more 
than 40 residues use the scroll bars to scroll through the long list, or use the {\sf Zoom} controls to fit the data from a long list into a small space.

	The {\sf Zoom} slider, and the {\sf Fit all}, {\sf Every Residue} buttons, zoom in and out of a long sequence list to allow viewing and selecting from the entire list all at once.  To represent more than 40 residues on the window, the text list  seems to ``skip'' residues, but selections, highlights and color-coded data are still active for all residues.  

By setting the {\sf Zoom} slider to a value smaller than 1.0, or by pressing the {\sf Fit all} button, more or all of the sequence information for a large molecule can be seen at once. To show a text line for every residue in the sequence (zoom factor = 1.0), click on the {\sf Every Residue} button.  The {\sf Zoom} slider can be dragged with the left mouse button (to re-scale sequence smoothly) or it can jump to a given value by clicking along the slider track with the middle button (this is useful to work more quickly with very long sequences). 

	
For a multi-thousand residue protein with {\sf Fit all} selected, hundreds of  residues 
can be selected at once, and   trends in B-value 
and structure across the entire protein sequence can be detected.
In the screen-shot above, a section of 70 residues with lower B-values than surrounding sequence is selected, by dragging a rectangle around the green stretch in the B-value column.


	Other controls include:

\begin {itemize} 
  \item {\bf Toggle display of  3-letter and 1-letter codes} -- Click on {\sf  1-letter code} to switch from 3-letter to 1-letter amino acid codes. The same button then reads {\sf 3-letter code}, click it to switch back from 1-letter to 3-letter codes.

\item {\bf Print contents of sequence window} -- 
	Select {\sf File:Print to File} to create a postscript file containing the current sequence listing and highlighting.

\end{itemize}

\subsubsection {Turn off highlighting / Change highlight style}
\label{ug:ui:window:sequence:offchange}

        
 To clear all highlights, reselect the current molecule from the {\sf Molecule} pop-up menu.
 To turn the highlight representation off completely for a given molecule, find the representation in the Graphics window which the Sequence window 
has created (appears with ``{\sf Bonds} {\sf ColorID 4}'') and set  the style to  ``none''.
To change highlighting style, set this same representation
to your preferred style and  coloring.  The selection for this
representation will still change whenever the sequence window selection changes.
Example application: specify {Multiple Frames} in the {\sf Trajectory} tab of
the highlight representation. This will display the trajectory motions of the
residues clicked on in the main VMD window, or in the Sequence window. 



\subsubsection{Caveats}
\index{sequence!caveats}

\begin{itemize}
         	
  
  \item Pause on first use:
       Since the sequence window displays secondary structure of loaded molecules, there may
       be a pause for structure calculation the first time the sequence for a protein is displayed. 

  \item Selections by chain: 
      When there are multiple segments in a chain, it is possible for several
residues to have the same residue number and chain name.  These 
residues will be highlighted/selected/deselected
together.
 
  \item B-values can be user assigned: To use the B-value column to view
arbitrary data, use the selection {\tt set beta} commands to change B-values.
To refresh the displayed B-value data, re-select the currently displayed
molecule from the {\sf Molecule} pop-up menu.  \end{itemize}



\subsection{RamaPlot}
\label{ug:ui:window:ramaplot}
\index{window!RamaPlot}
\index{Ramachandran plot}

\begin{rawhtml}
<CENTER>
\end{rawhtml}
\myfigure{ug_ramaplot}{The RamaPlot Window}{fig:ug:ramaplot}
\begin{rawhtml}
</CENTER>
\end{rawhtml}

The RamaPlot window displays a Ramachandran plot for a selected molecule.
If you animate your molecule over a range of \timesteps, RamaPlot will update
the Ramachandran plot automatically.  You can select a range of residues to
be displayed in the plot.  Clicking on a point in the Ramachandran
plot will show the trajectory of the selected residue in Ramachandran space
over all \timesteps.  Fields on the right of the window show the computed
value of phi and psi for the most recently selected residue.  Finally, you can 
create a PostScript image of the current Ramachandran plot.  RamaPlot
functionality is summarized in Fig.~\ref{fig:ug:ramaplot}.

\subsubsection{Using RamaPlot}

Start RamaPlot by typing ``{\tt ramaplot}" in the VMD text console, or by
selecting the {\sf ramaplot} menu item in the {\sf Extensions} menu.  The main
window contains a Ramachandran graph, with phi and psi running along the
horizontal and vertical axis, respectively, from -180 to 180 degrees.  The most
allowed region of Ramachandran space is colored blue; partially allowed regions
are colored green.

After loading a molecule, using the pulldown menu in the upper right part of
the window to choose a molecule.  Protein residues in the current molecule
are mapped to the Ramachandran diagram with yellow squares.  Clicking on a 
square causes the square to turn red, displays residue information in the
fields on the right side of the window, and, if trajectory data is present,
draws the location of the selected residue in Ramachandran space for all 
frames in the trajectory as empty black squares.  Clicking one of the empty
squares causes VMD to redraw the graphics display window with coordinates
from the timestep corresponding to that square.  Clicking a second time
on a red highlighted residue switches off the trajectory information
in the RamaPlot window.

When a protein contains many residues, it may be inconvenient to display
all residues at once.  Enter an atom selection in the Selection input to
choose which residues to display.  Note that the selection must contain the
alpha carbons (name CA) of the residues you want to show.  Note also that,
just like the Graphics window, the selection will not be recomputed 
if you change the animation frame.  

To print the contents of the white Ramachandran plot, select 
``Print to file..." from the RamaPlot File pulldown menu.  Enter a filename
to save the contents of the window.






