%%%%%%%%%%%%%%%%%%%%%%%%%%%%%%%%%%%%%%%%%%%%%%%%%%%%%%%%%%%%%%%%%%%%%%%%%%%%
% RCS INFORMATION:
%
%       $RCSfile: ug_tutorial.tex,v $
%       $Author: johns $        $Locker:  $                $State: Exp $
%       $Revision: 1.44 $      $Date: 2014/12/29 02:50:13 $
%
%%%%%%%%%%%%%%%%%%%%%%%%%%%%%%%%%%%%%%%%%%%%%%%%%%%%%%%%%%%%%%%%%%%%%%%%%%%%
% DESCRIPTION:
%   How to get up and running with myoglobin -- show the basics
%
%%%%%%%%%%%%%%%%%%%%%%%%%%%%%%%%%%%%%%%%%%%%%%%%%%%%%%%%%%%%%%%%%%%%%%%%%%%%


\chapter{Tutorials}
\label{chapter:tutorial:ug}

\section{Rapid Introduction to \VMD}
For those of you who don't like reading manuals, here is a quick
introduction to \VMD.  The molecules and data files used in this tutorial 
can be downloaded from the \VMD\ home page from the documentation area 
associated with this version and is clearly labeled as User's Guide 
tutorial data.  The rest of this tutorial assumes that you have 
downloaded and unpacked this data set.

To start \VMD\ type {\tt vmd} on the command line of your shell (Unix), or
start it by clicking the VMD icon in your desktop or Start menu (Apple MacOS
X and Microsoft Windows).  \VMD\ should start up with a window titled {\sf
vmd console}, a display window entitled {\sf VMD OpenGL Display}, and a main
menu entitled {\sf VMD}.  Text commands are typed in the console window,
molecules are displayed and manipulated in the graphics window, and other
interfaces and extensions are available from the menu interface.  All of the
windows can be closed or minimized, using your computer's standard windowing
controls or the \hyperref{{\tt menu} command}{{\tt menu} command
[\S}{]}{ug:ui:text:menu} in text console.  Most functions can be performed
with both the menu interface and the text console, though some of the more
sophisticated scripting features are only available as text commands.

\section{Viewing a molecule: Myoglobin}
\label{ug:tutorial:viewing}
\index{window!files}
\index{file!load}
\index{files!reading}
\index{molecule!loading}
In our quick tour of \VMD, we'll start out by demonstrating a few
of its visualization features.
To load a new molecule, select {\sf New Molecule\ldots}
from the {\sf File} menu in the Main window, this will open the 
\hyperref{{\sf Files} window}{{\sf Files} window [\S }{]}{ug:ui:window:files}.
We will load a PDB (Protein Data Bank) file containing the coordinates
of the atoms in myoglobin 
(compliments of Joel Berendzen of Los Alamos National Laboratory).  
Select the {\sf Browse\ldots} button in the files window to bring up a file
browser.  Go into the {\tt proteins/} directory of the tutorial data set
that you have downloaded from the VMD web site.
Once there, select the file {\tt mbco.pdb} in the file browser, and
press the {\sf Load} button in the molecule file browser.
button in the Files window.  
Figure \ref{fig:ug:myco} shows an example of \VMD\ displaying this protein.

\begin{rawhtml}
<CENTER>
\end{rawhtml}
\myhugefigure{ug_myco}{Sample \VMD\ session displaying myoglobin.}{Sample \VMD\ session displaying myoglobin.}{fig:ug:myco}
\begin{rawhtml}
</CENTER>
\end{rawhtml}

You can use the mouse to manipulate the structure in the display
window.  There are three basic \index{mouse!modes}
\hyperref{mouse modes}{mouse modes [\S }{]}{ug:ui:disp:modes}: 
rotation, translation, and scaling.
The mode can be changed from the Mouse menu in the main window, or by
pressing {\tt r}, {\tt t}, or {\tt s} on the keyboard while the mouse
is in the graphics window.  
While experimenting, note how the cursor changes to indicate the mouse mode.  
In rotation mode, the left mouse button controls
rotation about axes parallel to the screen, and the middle button
controls rotation about the axis perpendicular to the screen.  In
translation mode, the left mouse button controls translation parallel
to the screen, while the middle button controls translation in and
out of the screen.  Finally, in scaling mode, both the left and middle
buttons control global scaling when the mouse is moved left or right,
but the middle button causes larger changes.  

By default molecules are displayed in a ``lines'' representation, 
colored by atom type.
Suppose you would like to view the myoglobin structure with its
protein backbone represented as a tube, the heme represented as
licorice, the $SO_4$ ion and $CO$ molecule represented as van der
Waals spheres, and histidines 64 and 93 represented as CPK
models. 
First, open the
\hyperref{{\sf Graphics} window}{{\sf Graphics} window [\S }{]}{ug:ui:window:graphics}
by selecting the {\sf Representations} item in the the graphics menu
of the VMD Main window. \index{window!graphics}
Type {\tt backbone} in the {\sf Selected Atoms} text entry
\index{selection}
\index{atom!selection}
\index{coloring!methods}
area and press 'enter' to select the myoglobin backbone.  All of the
protein except for the backbone will disappear.  
Choose {\sf NewCartoon} in the drawing method chooser to display the backbone 
as a tube, and choose {\sf Structure} in the coloring method chooser 
to color the tube with the predefined secondary structure color.  
Press the {\sf Create Rep} button.  This creates a new representation 
in the browser, identical to the original one.  The new representation 
can be changed without affecting others, so clear the atom selection 
text area and enter {\tt resname HEM} to select the heme.  At this point
the heme isn't visible because it cannot be drawn as a cartoon, so choose
the `Licorice' drawing method 
\index{representation!style}
\index{drawing!method}
to make it appear.  Click on {\sf Create
New} again to make a new view, and enter {\tt resname SO4 CO} to
select the $SO_4$ ion and the $CO$ molecule, and choose the drawing
method `VDW' to render them as Van der Waal spheres.  Once again,
press the {\sf Create Rep} button and enter {\tt resid 93 64} to
select the two histidines, and render them as `CPK'.  
If you followed all that, then congratulations, you have made a nice 
image of myoglobin!  With further experimentation you should be 
well on your way to learning how to use \VMD.

\section{Rendering an Image}
\index{rendering}
  Find an interesting view of the molecule from the previous
tutorial.  Suppose you want to publish this view in a journal and want
a high quality image, or you want to make a large poster.  Taking the
image from a screen capture often results in a rather grainy image as the
size of the pixels becomes apparent, so you want something with more
resolution.  There are several programs available which can render a
high-quality raster image, based on an input script.  \VMD\ has the option
to create input scripts for many of these image processing programs,
which may then be processed to create a higher quality
image of the scene displayed by \VMD\ at the time the script was created.
See Chapter \ref{ug:topic:rendering} on rendering
for a further description of how this works.

Open the
\hyperref{{\sf Render} window}{{\sf Render} window [\S~}{]}{ug:ui:window:render}
\index{render!window}
\index{window!render}
\index{output!format}
\index{files!output}
\index{rendering!Tachyon}
and select `Tachyon' from the {\sf Render Using} menu. 
Both of the text boxes will be filled with default values which should 
not need to be changed for the purposes of this tutorial.
Press the {\sf Start Rendering} button.  
After a few moments of processing, you sould see the message
\begin{verbatim}
Info) Rendering complete.
\end{verbatim}
in the \VMD\ text console.
If everything worked correctly, you will
end up with an image file named plot.dat.tga (on MacOS X or Unix) or 
plot.dat.bmp (on Windows) in your current working directory.
This image is in either Windows BMP or Targa graphics format,
and can be read by many programs (such as {\tt display}, 
{\tt ipaste}, {\tt xv}, {\sf Gimp} or {\sf Photoshop}).

\section{A Quick Animation}
\label{ug:tutorial:animation}
\index{animation}

Another strength of \VMD\ lies in its ability to playback trajectories
resulting from molecular dynamics simulations.  A sample trajectory,
{\tt alanin.dcd} is provided in the {\tt proteins} directory included 
with VMD.  To load it, open the molecule file browser as described
previously.
\index{files!reading}
\index{molecule!loading}
Next click on the {\sf Browse} button and select the {\tt alanin.psf}
file in the file browser.  Once selected, press the {\sf Load} button to 
load the structure file.  Next, select the {\tt alanin.dcd} file and
load it as well.  This will read the DCD trajectory \timesteps into the
same molecule with the previously loaded {\tt alanin.psf} file.

In the display window you should see a simulation of an alanin residue 
in vacuo.  It isn't particularly informative, but you can easily see that 
the structure is quite unstable in an isolated environment.  After the DCD
file has loaded, animation will stop.  
To see it again or to fine- tune playback, use the
\hyperref{{\sf animation} controls}{{\sf animation} controls  [\S }{]}{ug:ui:window:animate} 
\index{window!animate}
found at the bottom of the main \VMD\ window.
Press the button that looks like {\tt >} to play the animation.  Use the 
{\sf Speed} slider at the bottom of the window to change the speed of playback.  
By rotating the molecule around, etc. you should get an idea about how
the system destabilizes over the course of the simulation.  The animation
controls are generally similar to what you'd find on a DVD or CD player.

\section{An Introduction to Atom Selection}
\index{selection}
\index{atom!selection}
In this section it is assumed that you have the myoglobin structure {\tt
mbco.pdb} loaded and the views discussed in section \ref{ug:tutorial:viewing}
created.  If this is not true, go back and repeat the process described there.

\VMD\ has a powerful atom selection method which is very
helpful when generating attractive, informative, and complex graphics.
In the previous section you used a few of these atom selection tools.
This tutorial assumes that you have already loaded the myoglobin
molecule, but it isn't necessary to recreate all the graphical
representations.

To change which atoms are used to display each representation of the
molecule shown in the display window, open the 
\hyperref{{\sf Graphics} window}{{\sf Graphics} window [\S~}{]}
{ug:ui:window:graphics} 
\index{window!graphics}
and select the representation you want to change.  
\index{representation!changing}
You can then either
edit the different fields (selection, coloring method, or drawing
method) or use the {\sf Delete} button to delete the view entirely.
Try changing or deleting some of the views.  When finished, delete all
representations for the myoglobin structure.  To get the basic line
drawing view back, clear the atom selection text entry area, enter {\tt all}
and press the {\sf Create Rep} button.

\index{atom!selection!examples}
Atoms may be selected on the basis of a property, i.e. {\tt protein}
or {\tt not protein}, {\tt water}, or {\tt nucleic backbone}.  They
may also be selected by atom name, such as {\tt atom C}, by residue
name, such as {\tt resname HEM}, or by many other identifiers.
Multiple atoms may be specified with one keyword.  For example, the
selection {\tt name C CA N O} will select the backbone atoms.  (A
similar effect may be obtained with the command {\tt protein
backbone}.)  \VMD\ can handle regular expressions, so that {\tt name
"C.*"} will select all atoms with names starting with C.  \VMD\ also
understands the boolean operators {\tt and}, {\tt or}, and {\tt not},
so the selection {\tt resname HEM and not name "N.*"} selects all
non-nitrogen atoms in the heme group of myoglobin.

Several more abstract selection criteria are available.  For instance, the
selection {\tt x $>$ 5} finds all atoms with an x coordinate greater than 5,
while {\tt mass $>$12 and mass $<$ 14} selects all atoms with mass greater
than 12 and less than 14 atomic mass units.  Many \hyperref{math
functions}{math functions [\S~}{]}{table:ug:functions} are also provided, so
the selection {\tt sqrt( sqr(x) $+$ sqr(y) $+$ sqr(z) ) $<$ 10} will select
atoms in a spherical region of radius 10 \AA\ centered about the origin of
the coordinate space.  You can pick atoms nearby a selection with the phrase
``within $<$distance$>$ of $<$selection$>$'' and all residues with the same
property as a given selection as ``same $<$property$>$ as $<$selection$>$''.

See section \ref{ug:topic:selections}
for a full description of the selection command.

\section{Comparing Two Structures}

Let's start from scratch by deleting everything: use the text
console and tye the command {\tt mol delete all} and press enter.
This deletes all loaded molecules and is often more convenient
then selecting them and deleting them all one by one.  Alternatively,
you could highlight each molecule in the molecule browser, and
use the {\sf Delete Molecule} item in the {\sf Molecule} menu to 
remove them one by one.

Begin by loading the {\tt mbco.pdb} structure with the Files window.
Turn on just the heme, CO, and histidines by using the selection commands
{\tt resname HEM CO or resid 64 93}.  The dot (probably green) in the
middle is the iron and you can verify that by picking it with the
mouse.  Do this by changing the ``Object Mode'' pull-down to ``Pick'', and
selecting ``Atoms'' for the pick mode in the Mouse menu.
The label {\tt HEM154:FE} should appear both on the display
and in the text console.\index{labels}\index{atom!picking}
\index{picking!atoms}

Change the pick mode in the Mouse menu to ``Bonds''.
To get the distance between the iron and the oxygen of the CO, click
with the left mouse button first on the iron and then on the oxygen.
\index{picking!bonds}\index{picking!distances}
The first click turned the FE label on and the second turned the O
label on and drew a line between the two atoms with the distance drawn
in the middle and a bit to the right.  
\index{distance between atoms}
\index{atoms!distance between}
The distance between the two
atoms is 2.94 \AA, as compared to 2.93 \AA\ in the paper; not bad.
However, picking the distance between the FE and the C of the CO
reveals a distance of 1.91 \AA\ as compared to 1.85 \AA\ in the paper.
The difference is that the structures in the \VMD\ distribution are
actually preliminary structures obtained before the final coordinates
were determined.

In order to experiment with more complex picking modes, 
\index{picking!modes}
\index{mouse!modes}
consider the
angle made by the O of the CO with the FE of the heme and the NE2 of
residue 93 (you can click on the atoms to find which ones are which).
Using the Mouse menu, change the pick mode to ``Angles''.
This should cause the cursor to become a red crosshair.  
Click on each of the three atoms using the left mouse button.  After
the third pick, a shallow angle will appear indicating an 8.71 degree
angle between the three atoms.


Now load the intermediate {\tt star.pdb} file which can also be found
in the {\tt proteins} directory of your distribution. Again use the
Files window to do this.  Both of the molecules will be loaded side by
side.  Go to the Graphics window and change the selection so it the same
as the first, i.e. {\tt resname HEM CO or resid 64 93}.  The two
molecules are almost atop each other, making it hard to distinguish
the two, so change the colors to simplify things.

First, in the {\sf Graphics} window, change the {\sf Coloring method} to
`Molecule'.  Use the {\sf Selected Molecule} chooser to change the
{\tt mbco.pdb} {\sf Coloring method} to `Molecule' as well.  Open the
\hyperref{{\sf Color} window}{{\sf Color} window [\S }{]}{ug:ui:window:color}
\index{color!window}
\index{window!color}
and scroll the {\sf Category} browser down until the line `Molecule'
is visible.  Click on it then click on the line which says {\tt
mbco.pdb}.  (There may be two {\tt mbco} lines if the file had been loaded
before in this session.)  Scroll the {\sf Colors} browser up to click
on `blue'.  This should change one of the molecules in the display to
blue. \index{color!assignment}

Next, click on the last line in the {\sf Names} chooser, which says
{\tt star.pdb}.  This time, choose `red' from the {\sf Colors}
chooser.
The display should be much easier to understand.  The myoglobin with
the bound CO is in blue and the intermediate state is in red.  At this
point it is easy to measure the change in position between the two
different states by using the middle mouse button to pick the same
atom in the two conformations.  

Once that is done, it is easy to point out one interesting aspect of
the way \VMD\ handles the graphics.  Go to the main window, select one of
the two molecules, and press {\sf Toggle Fixed}.  Enter translation
mode and move the other molecule around.  Notice that the number which
lists the distance between the two atoms never changes.  That's
because the mouse only affects the way the coordinates are translated
to the screen image.  It does not affect the real coordinates at all.
It is possible to change the coordinates in a molecule using the text
command interface, or by using the 
\hyperref{atom move pick modes}{atom move pick modes [\S }{]}{ug:ui:disp:pick}).
\index{molecule!fixed}
\index{molecule!translation}
\index{atom!coordinates!changing}

By the way, unfix the molecules and do a `Reset View' from the Display
menu to reset everything.  Load up the third structure, {\tt deoxy.pdb} 
and give it the same selection as the other two molecules.
However, color this one green.  Pull out Nature v. 371, Oct. 27, 1994
and turn to page 740.  With a bit of manipulation you should be able
to recreate the image that appears there.

\section{Some Nice Represenations}
\index{representation!examples}
The following views are quite nice for displaying proteins and nucleic
acids:

\begin{verbatim}
selection: all
drawing method: tube
coloring method: segname (or chain)
why?  This show the backbone of the protein and nucleic acid strands

selection: protein and (name CA or not backbone)
drawing method: lines
coloring method: segname (or chain)
why? shows where the side chains are located, but they are thin so the
  backbone is still visible and the scene is quickly drawn

selection: (numbonds = 0) and not waters
drawing method: vdw
coloring method: name
why? shows ions.  The "not waters" omits cases where a water's oxygen is
  known but not the hydrogen.

selection: not (waters or protein or nucleic)
drawing method: lines
coloring method: name
why? shows whatever is left; usually ligands and crystallizing agents
\end{verbatim}



\section{Saving your work}
\label{ug:scripts:savestate}
\index{save!vmd state}
\index{save!configuration}
\index{restore!vmd state}

After creating a set of attractive and informative representations of your
molecule, you may want to save your work so that you can regenerate the 
scene later.  There are two ways to do this in \VMD:
\begin{itemize}
\item In the main menu, press the {\sf Save State} button found in the 
{\sf File} menu; this will bring up
a browser window where you can enter a file name in which to save your work.

\item In the text console, type {\tt save\_state {\it filename}}, where
{\it filename} is the name of the file in which to save your work.

\end{itemize}

To restore your scene, you also have three choices:

\begin{itemize}

\item Use the {\sf Load State} item in the {\sf File} manu to select and
      load a previously saved \VMD\ session.

\item From the command line, start \VMD\ with the options 
{\tt vmd -e {\it filename}}, where {\it filename} was the name of the file you
saved before.  

\item After starting \VMD, from the text console, type {\tt play {\it filename}}. 

\end{itemize}

The most common source of problems is when \VMD\ can't find the files you
used to load the molecule.  If this happens, try changing to the directory
you were in when you first loaded the molecule, or edit the state file and
use the full path names where you see {\tt mol new}, 
{\tt mol addfile}, or {\tt mol load} commands.



\section{Tracking Script Command Versions of the GUI Actions}
\label{ug:scripts:logfile}
\index{save!logfile}
\index{logging tcl commands}
\index{logfile!enable from GUI}
\index{logfile!disable from GUI}

For most actions performed from the \VMD GUI, there is an equivalent
script command. \VMD can print these commands to a log file or the
console. This is a convenient way to automate file processing by first 
doing all steps interactively while logging to a file and then editing
the logfile to turn it into a Tcl script operating on multiple files.
There are two ways to do this in \VMD:
\begin{itemize}
\item In the main menu, press the {\sf Log TCL Commands to File} button 
found in the {\sf File} menu; this will bring up a browser window where you 
can enter a file name in which you can save the resulting script code.
\item In the text console, type {\tt logfile {\it filename}}, where
{\it filename} is the name of the log file.
\end{itemize}

The resulting file will contain Tcl script code that can be executed
from the \VMD command prompt. The {\sf Log TCL Commands to Console} button
or the command {\tt logfile console} will print the Tcl commands to
the console window instead. This is most useful, if you just want to
find out, which \VMD command is used to perform a specific action.

Finally, the {\tt logfile off} command or clicking on the 
{\sf Turn Off Logging} button will stop the log and close
the log file, if needed.

