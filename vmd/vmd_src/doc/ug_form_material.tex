
\subsection{Material Window}
\label{ug:ui:window:material}
\index{window!material}
\index{image!shading and material properties}

\begin{rawhtml}
<CENTER>
\end{rawhtml}
\myfigure{ug_material}{The Material Window}{fig:ug:window}
\begin{rawhtml}
</CENTER>
\end{rawhtml}

This window is used to create and modify material definitions.  The material
definitions created here will show up in the pulldown menu in the Graphics
window, allowing you to apply a material to a given representation.  

The upper left corner of the Materials window contains a browser listing all the
currently defined materials.  Below this browser is a set of five sliders which
indicate the current materials settings for the material highlighted in the
browser.  Highlighting a different material in the browser by clicking with the
mouse will update the settings of the sliders.  Conversely, moving the sliders
will change the definition of the the currently highlighted material in the
browser.  Pressing the "Default" button will restore either of the first two
materials, "Opaque" and "Transparent", to their original settings.

To create a new material, press the "Create New" button in the upper right
corner of the window.  A new material with a default name will be created and
displayed in the browser window.  This name can be changed at any time to
something more descriptive by typing in the input box to the right of the
material browser and pressing "enter" (note that the names of "Opaque" and
"Transparent" cannot be changed).  You can now edit the properties of this
material using the sliders at the bottom of the window.  All materials in the
materials browser, including those you create, will appear in the Material
pulldown menu in the Graphics window. 

To experiment with the material settings, first create a new material so
that you can edit its values.  Next, load any molecule, change its 
drawing method to VDW representation, and using the Material pulldown menu
in the Graphics window, change the representation's material to the 
material you just created.  Now, go back to the Materials window, highlight
the new material in the browser, and change some of the values in the sliders.
The effect of changing shininess should be especially dramatic.

