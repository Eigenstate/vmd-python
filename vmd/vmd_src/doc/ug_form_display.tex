%%%%%%%%%%%%%%%%%%%%%%%%%%%%%%%%%%%%%%%%%%%%%%%%%%%%%%%%%%%%%%%%%%%%%%%%%%%%
% RCS INFORMATION:
%
%       $RCSfile: ug_form_display.tex,v $
%       $Author: johns $        $Locker:  $                $State: Exp $
%       $Revision: 1.30 $      $Date: 2014/12/29 03:13:32 $
%
%%%%%%%%%%%%%%%%%%%%%%%%%%%%%%%%%%%%%%%%%%%%%%%%%%%%%%%%%%%%%%%%%%%%%%%%%%%%
% DESCRIPTION:
%  the Display window 
%
%%%%%%%%%%%%%%%%%%%%%%%%%%%%%%%%%%%%%%%%%%%%%%%%%%%%%%%%%%%%%%%%%%%%%%%%%%%%


\subsection{Display Menu and Display Settings Window}
\label{ug:ui:window:display}
\index{window!display}
\index{display!window}

\begin{rawhtml}
<CENTER>
\end{rawhtml}
\myfigure{ug_display}{The Display menu}{fig:ug:display}
\begin{rawhtml}
</CENTER>
\end{rawhtml}

The {\sf Display} menu controls many of the characteristics of the graphics
display window.  The characteristics which may be modified include:
\begin{itemize}
  \item{\bf Reset View}\index{reset view} --
This menu item can be used to force \VMD\ to reset the scene back to the
default viewing orientation and scale as is done when a molecule is 
first loaded.
  
  \item{\bf Stop Rotation}\index{stop rotation} --
This menu item stops autorotation of the scene.  The scene can be 
autorotated by quickly dragging the mouse while briefly depressing 
and releaseing the mouse button, leaving the scene spinning until it
is stopped either by this menu item or by further mouse interactions. 

  \item{\bf Perspective}\index{perspective view} --
The view of the scene can be {\sf Perspective} or {\sf Orthographic}. 
\index{orthographic view}
In the perspective view (the default), objects which are far away are
smaller than those near by.  In the orthographic view, both objects
appear at the same scale.  Note that several of the supported external
rendering programs do not support orthographic rendering.  As such, it may
be necessary to ``fake it'' by translating the scene far away from the 
camera, and apply a zoom factor.  This has the effect of significantly
reducing the perspective, while not truly an orthographic view.

  \item {\bf Antialiasing}\index{antialiasing} --
Turns antialiasing on or off.  Antialiasing helps smooth out
the jagged appearance of displayed geometry resulting from the inherently 
discrete pixels on the display device.  The antialiasing feature is 
only available on platforms which support full-screen antialiasing,
sometimes known as ``multisample antialiasing''.  Several SGI and Sun
graphics systems fully support this feature.  
On platforms lacking the multisample
capability, there may be alternate ways to perform full-screen antialiasing
by selecting an option in the display driver setup.  Windows machines most
commonly place these controls in the display driver configuration panel.

  \item {\bf Depth Cueing}\index{depth cueing} --
Turns depth cueing on or off.  Depth cueing causes distant objects to 
blend into the background color, in order to aid in 3-D depth perception.
The depth cueing settings controlled in the {Display Settings} window.
The {\bf Cue Mode} parameter controls which type of fog equation is
used.  The {\bf Linear} depth cueing mode provides a simple depth
gradient with a defined starting point and endpoint.  The 
{\bf Exp} and {\bf Exp2} depth cueing modes take a density parameter,
and generally blend into the background color much more sharply than
the linear depth cueing mode.
Scaling up the molecule will increase the amount of depth cueing
effect that is visible, since it will occupy a larger depth range.
Scaling the molecule size down decreases the depth cueing effect.  
Translating the molecule into and out of the screen will cause it
to blend into and out of the background color.

  \item {\bf Culling}\index{backface culling} --
Turns backface culling on or off.  This feature is primarily used
to accelerate rendering performance on software based implementations
of OpenGL, such as Mesa.  Backface culling actually reduces performance
on some hardware renderers, so you'll have to use your own best judgement
on whether or not it is helpful to use on your specific computer system.

  \item {\bf FPS}\index{frames per second indicator} --
This option enables or disables on-the-fly display of the achieved
\VMD\ rendering frame rate.  The frame rate is displayed in the upper
right hand corner of the graphics window when it is enabled.

  \item {\bf Lights}\index{light!toggle} --
\index{image!lighting controls}
The graphics display window can use up to four separate light sources
to add a realistic effect to displayed graphical objects.
The {\sf Lights On} browser turns these light sources on or off.  If the
number is highlighted, the light is on, and clicking on it turns the
light off.  See 
section \ref{ug:ui:disp:lights}
for more discussion regarding lights.

  \item {\bf Axes}\index{axes} --
A set of XYZ axes may be displayed at any one of five places on
the screen (each of the corners or the center) or turned off.  This is
controlled by the {\sf Axes} chooser.

  \item {\bf Background}\index{display!backgroundgradient} --
The display background can either be set to a uniform color over
the entire scene, or a vertical gradient can be set with a linearly
changing color from the top of the viewport to the bottom.

  \item {\bf Stage}\index{stage} --
The {\sf Stage} browser controls the stage, which is a checkerboard plane
that can be located in any one of six places or turned off.

  \item {\bf Stereo, Eye Sep, and Focal Length}\index{stereo!parameters} --
\label{ug:ui:window:stereo}
These controls are found in the {\sf Display Settings} window.
These controls set the stereo mode and parameters; stereo is discussed
fully in
chapter \ref{ug:topic:stereo}.
The {\sf Stereo} chooser changes the stereo mode, while the {\sf Eye Sep}
and {\sf Focal Length} controls change the eye separation distance and the
focal length, respectively.
The {\sf Stereo Eye Swap} control optionally reverses the left/right eyes
when displaying on projectors or other devices that for one reason or
another don't preserve the correct left/right eye assignments.

  \item {\bf Cachemode}\index{cachemode} --
The {\sf Cachemode} toggle controls whether or not \VMD\ uses
a display list caching mechanism to accelerate rendering of 
static geometry.  This feature can be extremely beneficial for
achieving good interactive display performance on tiled display walls,
and for remote display over a network.  Caching cannot be performed
while animating trajectories, so the performance benefit is only possible
interactive rotation and zooming of static molecular structures.

  \item {\bf Rendermode}\index{rendermode} --
The {\sf Rendermode} chooser controls which low-level rendering method
\VMD\ uses.  The {\bf Normal} rendering mode is the default VMD rendering
algorithm based on standard fixed-function OpenGL.  The {\bf GLSL} 
rendering mode uses OpenGL Programmable Shading Language to implement
real-time ray tracing of spheres, alpha-blended transparency,
and high-quality per-pixel lighting for all geometry.  On machines
with high performance graphics boards supporting programmable shading,
the {\bf GLSL} rendering mode provides quality on par with many of
the external software renderers supported by \VMD\, but at interactive 
display rates.

\item {\bf Clipping Planes (Near Clip and Far Clip)}\index{clipping planes} --
These controls are found in the {\sf Display Settings} window.
Only those parts of the scene between the near and far clipping planes 
are drawn.  
The display clipping planes also set the depth cueing start and endpoints.
Objects at the near clipping plane are distinct and crisp, 
objects at the far clipping plane will be blended into the background.
Clipping planes positions are changed with the {\sf Near Clip} 
and {\sf Far Clip} controls.  It is not possible for the near
clip to be farther away than the far clip.  When using stereo, it may
be useful to set the near clip plane much lower than the default value.
This makes the geometry ``pop out of the screen'' a bit more, 
and can be used for greater dramatic effect.

\item {\bf Screen Height (Hgt) and Distance (Dist)} --
\label{ug:ui:window:screen_size}
\index{screen parameters}
These controls are found in the {\sf Display Settings} window.
The screen height, along with the screen distance, defines the geometry
and position of the display screen relative to the viewer.  The screen
height is the vertical size of the display screen, in `world' coordinates.
Each molecule is initially scaled and translated to fit within a 2 x 2 x 2
box centered at the origin; so the screen height helps determine how large
the molecule appears initially to the viewer.  

The screen distance parameter determines the distance, in `world' coordinates, 
from the origin to the display screen.  If this is zero, the origin of the
coordinate system in which molecules (and all other graphical objects) are
drawn coincides with the center of the display.  If distance is negative
the origin is located between the viewer and the screen, if it is 
positive, the screen is closer to the viewer than the origin.
A negative value puts any stereo image in front of the screen, aiding the
three-dimensional effect; a positive value results in a stereo image that is
behind the screen, a less dramatic effect (but easier to see, for some
people) stereo effect.

\begin{rawhtml}
<CENTER>
\end{rawhtml}
\myhugefigure{screen_params}{Relationship between screen height, distance to origin, and the viewer}{Relationship between screen height
(SCRHEIGHT), screen distance to origin (SCRDIST), and the viewer}
{fig:ug:screen}
\begin{rawhtml}
</CENTER>
\end{rawhtml}

Figure \ref{fig:ug:screen} describes the relationship between the screen
height, the screen distance, and the world coordinate space.


\item {\bf Shadows}\index{Shadows} --
The shadows control enables and disables direct lighting 
shadowing when using the built-in Tachyon CPU or GPU renderers 
or when exporting the VMD molecular scene to external renderers 
that implement shadowing algorithms.
The simple direct lighting model implemented in most renderers yields
shadows that are completely dark, producing a somewhat harsh lighting
quality akin to what would be expected in a desert under full sunlight
with no clouds.  

\item {\bf Ambient Occlusion}\index{Ambient Occlusion} --
The ambient occlusion (AO) lighting control enables the use of 
so-called ambient occlusion indirect lighting when using the built-in
Tachyon CPU or GPU renderers, or external renderers that implement AO.
Images with much higher quality shading can be produced by augmenting 
direct lighting with ambient occlusion or broad angle or indirect 
lighting techniques.
Ambient occlusion lighting emulates the broad lighting effects 
similar to what would be experienced on a cloudy or overcast day with
omnidirectional light arriving on all surfaces.
The AO ambient coefficient controls the strength of the omnidirectional
lighting components.  The AO direct coefficient scales the direct 
lighting contribution associated with the directional and 
positional lights.

\item {\bf Depth of Field (DoF)}\index{Depth of Field}\index{DoF} --
The depth of field (DoF) control enables or disables emulation of 
depth of field focal blur effects associated with fast focal ratio
camera optics and close focus distances.  The depth of field 
implementation provided by the built-in Tachyon ray tracer and most
other renderers yields a plane of perfect focus at a specified distance
from the camera.  The degree of focal blurring with increasing distance
from the plane of perfect focus depends on both the simulated f/stop
and the distance between the plane of perfect focus and the camera.


\end{itemize}




