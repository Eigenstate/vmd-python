%%%%%%%%%%%%%%%%%%%%%%%%%%%%%%%%%%%%%%%%%%%%%%%%%%%%%%%%%%%%%%%%%%%%%%%%%%%
%cr                                                                       
%cr            (C) Copyright 1995 The Board of Trustees of the            
%cr                        University of Illinois                         
%cr                         All Rights Reserved                           
%cr                                                                       
%%%%%%%%%%%%%%%%%%%%%%%%%%%%%%%%%%%%%%%%%%%%%%%%%%%%%%%%%%%%%%%%%%%%%%%%%%%

%%%%%%%%%%%%%%%%%%%%%%%%%%%%%%%%%%%%%%%%%%%%%%%%%%%%%%%%%%%%%%%%%%%%%%%%%%%%
% RCS INFORMATION:
%
%       $RCSfile: vmd_macros.tex,v $
%       $Author: johns $        $Locker:  $                $State: Exp $
%       $Revision: 1.21 $      $Date: 2014/12/29 02:50:13 $
%
%%%%%%%%%%%%%%%%%%%%%%%%%%%%%%%%%%%%%%%%%%%%%%%%%%%%%%%%%%%%%%%%%%%%%%%%%%%%
% DESCRIPTION:
%
% useful macros for documentation.  This file is included by all
% main documents, i.e. programmers guide, users guide, etc.
%%%%%%%%%%%%%%%%%%%%%%%%%%%%%%%%%%%%%%%%%%%%%%%%%%%%%%%%%%%%%%%%%%%%%%%%%%%%

%
% generally useful macros
%
\newcommand{\REFAND} {\&}
\newcommand{\ETALNP}{\mbox{\it et al}}
\newcommand{\ETAL}{\mbox{\ETALNP{\it.}}}
\newcommand{\eqnref}[1] {\mbox{eq (\ref{#1})}}
%\newcommand{\mycite}[2] {\cite{#2}}
\newcommand{\mycite}[2] {}


%
% a "myfigure" macro for including graphics etc.
%

%
% Old pre-pdflatex version
%
%\newcommand{\myfigure}[3]{
%\begin{figure}[htb]
%  \begin{center}
%      \epsfig{file=pictures/#1}
%  \end{center}
%  \caption{#2}
%  \label{#3}
%  \htmlimage{scale=1.6}
%\end{figure}
%}

%
% New pdflatex-friendly version
%

% behavior: PS is scaled down a bit, HTML is 1 to 1
% this is the default macro for figures/screenshots
%% NOTE: Do not use \mynewfigure in the LaTeX sources, 
%% instead, rename it to replace the old \myfigure.
\newcommand{\myfigure}[3]{
  \begin{figure}[htb]
  \begin{center}
    \html{
      \htmlimage{scale=1.6}
    }
    \latex{
      \scalebox{0.550}{\includegraphics{pictures/#1}}
    }
    \end{center}
    \caption{#2}
    \label{#3}
  \end{figure}
}


% behavior: Both PS and HTML are rescaled to 4in in width
% use for exceptionally large figures that cause problems.
\newcommand{\myhugefigure}[4]{
  \begin{figure}[htb]
  \begin{center}
    \resizebox{4in}{!}{\includegraphics{pictures/#1}}
    \end{center}
    \caption[#2]{#3}
    \label{#4}
  \end{figure}
}


% OLD MACRO -- Keep in case of trouble...
% HTML is very ugly/barely readable
% All PDF figures have same width
%\newcommand{\myoldfigure}[3]{
%  \begin{figure}[htb]
%  \begin{center}
%    \resizebox{3in}{!}{\includegraphics{pictures/#1}}
%%   \scalebox{0.625}{\includegraphics{pictures/#1}}
%    \end{center}
%    \caption{#2}
%    \label{#3}
%  \end{figure}
%}



%
% include name and version number of program; this is generated
% by the 'make version' command in either the doc or src directory
% This will define the 'VMDNAME', 'VMDVER', and 'VMDDATE' macros,
% as well as the 'VMDAUTHORS' macro.
% Then the 'VMD' macro is used throughout the document to refer to
% the program name, while the other macros as defined in the file
% vmd_version.tex are used as given.
%

\newcommand{\VMDNAME} {vmd}
\newcommand{\VMDDATE} {August 27, 2015}
\newcommand{\VMDVER} {1.9.3a3}
\newcommand{\VMDAUTHORS} {R. Brunner, E. Caddigan, J. Cohen, A. Dalke, P. Grayson, J. Gullingsrud, D. Hardy, W. Humphrey, B. Isralewitz, S. Izrailev, A. Kohlmeyer, D. Norris, J. Saam, J. Stone, J. Ulrich, K. Vandivort}
\newcommand{\VMDVERSAUTHORS} {J. Stone, K. Vandivort}


%
% macros for style conventions when describing the program.
%

% name of class or object in program
\newcommand{\OBJ}[1] {\htmlref{{\bf #1}}{#1}}

% function arguments
\newcommand{\FA}[2] {{\rm{\bf#1}\ {\it#2}}}

% global function name
\newcommand{\FN}[3] {{\rm\bf#1}\ {\tt #2(}#3{\tt)}}

% class member function name
\newcommand{\FNO}[4] {{\rm\bf#2}\ \OBJ{#1::}{\tt#3(}#4{\tt)}}

% list item, for optional components, parameters, etc.
\newcommand{\TTLISTITEM}[1] {\item {\tt #1} \\}
\newcommand{\RMLISTITEM}[1] {\item {\rm #1} \\}
\newcommand{\BOLDLISTITEM}[1] {\item {\bf #1} \\}
\newcommand{\EMLISTITEM}[1] {\item {\em #1} \\}
\newcommand{\LISTITEM}[1] {\RMLISTITEM{#1}}

%
% other generally useful macros
%

% where to e-mail us
\newcommand{\vmdemail} {{\tt vmd@ks.uiuc.edu}}

% name of NAMD and MDCOMM programs, formatted nicely
\newcommand{\VMD}       {VMD}
\newcommand{\NAMD}      {NAMD}
\newcommand{\CESB}      {MDScope}
\newcommand{\MDScope}   {MDScope}
\newcommand{\BIOCORE}   {BioCoRE}
\newcommand{\MDTOOLS}   {MDTools}
\newcommand{\JMV}       {JMV}

% full name for CESB, i.e., what it stands for
\newcommand{\CESBNAME} {Molecular Dynamics computational environment}

% title of CESB paper
\newcommand{\CESBPAPER} {MDScope: A Visual Computing Environment for
Structural Biology}

% We aren't sure about "timesteps" vs. "frames" for now, so we define macros
% This is used to refer to "trajectory frames/timesteps" only!
\newcommand{\timestep} {frame }
\newcommand{\timesteps} {frames }
\newcommand{\Timesteps} {Frames }

%
% macros used for formatting object description pages
%

% text printed at top of object description page;
% this also has an argument mentioning in what optional component
% this object is used.
\newcommand{\OPTOBJDESCRIPTIONHEADER}[6] {
  \newpage
  \subsection{#1}
  \label{#1}
  \begin{tabular}{|ll|} \hline
    {\em Files:}			& {\tt #2} 		\\
    {\em Derived from:} 		& {#3} 			\\
    {\em Global instance (if any):} 	& {#4}			\\
    {\em Used in optional component:}	& {#5}			\\ \hline
  \end{tabular}
  \subsubsection*{Description}
  {#6}
}

% text printed at top of object description page, for a 'standard' object.
\newcommand{\OBJDESCRIPTIONHEADER}[5] {
  \newpage
  \subsection{#1}
  \label{#1}
  \begin{tabular}{|ll|} \hline
    {\em Files:}			& {\tt #2} 		\\
    {\em Derived from:} 		& {#3} 			\\
    {\em Global instance (if any):} 	& {#4}			\\
    {\em Used in optional component:}	& {Part of main \VMD\ code} \\ \hline
  \end{tabular}
  \subsubsection*{Description}
  {#5}
}

% after the header comes info on the constructor.
\newcommand{\OBJCONSTRUCTOR}[1] {
  \subsubsection*{Constructors}
  \begin{itemize}
    #1
  \end{itemize}
}

% then comes a list of any enumerations or list of names, if any
\newcommand{\OBJLISTS}[1] {
  \subsubsection*{Enumerations, lists or character name arrays}
  {#1}
}

% then comes a list of important internal data structures
\newcommand{\OBJDATA}[1] {
  \subsubsection*{Internal data structures}
  \begin{itemize}
    #1
  \end{itemize}
}

% then comes a list of the functions in this object (nonvirtual)
\newcommand{\OBJFUNCTIONS}[1] {
  \subsubsection*{Nonvirtual member functions}
  \begin{itemize}
    #1
  \end{itemize}
}

% then comes a list of the functions in this object (virtual)
\newcommand{\OBJVIRTUALFUNCTIONS}[1] {
  \subsubsection*{Virtual member functions}
  \begin{itemize}
    #1
  \end{itemize}
}

% and finally a description of how to use this object
\newcommand{\OBJUSAGE}[1] {
  \subsubsection*{Method of use}
  {#1}
}

% if desired, hints for what to change can go last
\newcommand{\OBJFUTURE}[1] {
  \subsubsection*{Suggestions for future changes/additions}
  {#1}
}

