%%%%%%%%%%%%%%%%%%%%%%%%%%%%%%%%%%%%%%%%%%%%%%%%%%%%%%%%%%%%%%%%%%%%%%%%%%%
%cr                                                                       
%cr            (C) Copyright 1995 The Board of Trustees of the            
%cr                        University of Illinois                         
%cr                         All Rights Reserved                           
%cr                                                                       
%%%%%%%%%%%%%%%%%%%%%%%%%%%%%%%%%%%%%%%%%%%%%%%%%%%%%%%%%%%%%%%%%%%%%%%%%%%

%%%%%%%%%%%%%%%%%%%%%%%%%%%%%%%%%%%%%%%%%%%%%%%%%%%%%%%%%%%%%%%%%%%%%%%%%%%%
% RCS INFORMATION:
%
%       $RCSfile: ig_install.tex,v $
%       $Author: justin $        $Locker:  $                $State: Exp $
%       $Revision: 1.9 $      $Date: 2000/12/15 23:25:55 $
%
%%%%%%%%%%%%%%%%%%%%%%%%%%%%%%%%%%%%%%%%%%%%%%%%%%%%%%%%%%%%%%%%%%%%%%%%%%%%
% DESCRIPTION:
%
% INSTALLATION GUIDE : installing VMD
%   1) Creating the executable
%   2) Setting installation parameters
%   3) Installing files
%%%%%%%%%%%%%%%%%%%%%%%%%%%%%%%%%%%%%%%%%%%%%%%%%%%%%%%%%%%%%%%%%%%%%%%%%%%%

\chapter{Installing \VMD}
\label{chapter:ig:install}

Before all the files required for \VMD\ to run properly can be
installed, the \VMD\ executable must be properly compiled. \VMD\ may
be compiled for a number of operating systems and may include any of a
number of optional features.  The standard \VMD\ distribution includes
a precompiled version of the executable, which may be used immediately
for installation.  However, the precompiled version of \VMD\ may not
meet the needs or preferences of the user, and in this case a new
version of the executable must be created as described in chapter
\ref{chapter:ig:compile}.  The precompiled executables described in
table
\ref{table:ig:execs} are provided in the {\tt src} directory, with the
given filenames, and may be used for installation if sufficient.  The
executables have been compiled with the configuration options listed
in table \ref{table:ig:execs}; the definitions of these
configuration options are given in chapter \ref{chapter:ig:compile}.

\begin{table}[htb]
  \hspace{1.1in}
  \begin{tabular}{|l|l|} \hline
    Filename & \multicolumn{1}{|c|}{Configuration Options} \\ \hline\hline
    {\tt vmd\_IRIX6}	& {\tt IRIX6 OPENGL FLTK TCL TK VRPN} \\ \hline
  \end{tabular}
  \caption{Precompiled version(s) of \VMD\ provided with the standard distribution.}
  \label{table:ig:execs}
\end{table}

\section{Setting installation parameters}
\label{section:ig:instparams}

After an executable has been created for the proper operating system
version and with the proper optional features, the first step in
installing \VMD\ is to set the values of certain installation
parameters.  These parameters are in the file {\tt
configure.parameters} in the main \VMD\ working directory.  Edit this
file and set the preferred values of the following installation
parameters (other parameters in the file should not be changed; if you
have already changes these installation parameters as part of the
steps in compiling a new version of \VMD, you may skip immediately to
section
\ref{section:ig:installfile}):
%%%%%%%%%%%%%%%%%%%%%%%%%%%%%%%%%%%%%%%%%%%%%%%%%%%%%%%%%%%%%%%%%%%%%%%%%%%
%cr                                                                       
%cr            (C) Copyright 1995 The Board of Trustees of the            
%cr                        University of Illinois                         
%cr                         All Rights Reserved                           
%cr                                                                       
%%%%%%%%%%%%%%%%%%%%%%%%%%%%%%%%%%%%%%%%%%%%%%%%%%%%%%%%%%%%%%%%%%%%%%%%%%%

%%%%%%%%%%%%%%%%%%%%%%%%%%%%%%%%%%%%%%%%%%%%%%%%%%%%%%%%%%%%%%%%%%%%%%%%%%%%
% RCS INFORMATION:
%
%       $RCSfile: vmd_paraminst.tex,v $
%       $Author: johns $        $Locker:  $                $State: Exp $
%       $Revision: 1.6 $      $Date: 2000/05/11 16:25:37 $
%
%%%%%%%%%%%%%%%%%%%%%%%%%%%%%%%%%%%%%%%%%%%%%%%%%%%%%%%%%%%%%%%%%%%%%%%%%%%%
% DESCRIPTION:
%
% multiple GUIDEs : installation parameters
% 	NOTE: should be NO blank line between last comment and begin
%%%%%%%%%%%%%%%%%%%%%%%%%%%%%%%%%%%%%%%%%%%%%%%%%%%%%%%%%%%%%%%%%%%%%%%%%%%%
\begin{itemize}
  \TTLISTITEM{INSTALLBINDIR (default=/usr/local/bin)}
The directory where scripts
and utility programs are installed.  This should be a directory which
is in the path of all users interested in running \VMD.

  \TTLISTITEM{INSTALLLIBDIR (default=/usr/local/lib/vmd)}
A directory in which \VMD\ data
files and architecture-specific executables are placed.  This
directory should {\em not} be in the path of users running \VMD.

  \TTLISTITEM{INSTALLNAME (default=vmd)}
\VMD\ uses a shell script to run the proper \VMD\ executable for the given version of Unix (the
script also sets some environment variables and starts an {\tt
xterm} running to act as the \VMD\ console).  {\tt INSTALLNAME}
is the name given to this shell script, and will be the command users
type to run \VMD.  When the program is
installed, the file {\tt bin/vmd} is then copied to the file
{\tt INSTALLBINDIR/INSTALLNAME}.

\end{itemize}





If any of the installation parameters are changed, the configure
script must be run to generate a new Makefile containing the altered
values.  Also, if a precompiled \VMD\ executable is being used, the
configure script {\em must} be run to create a Makefile which
specifies the proper operating system version. If it is necessary to
run the script again, do the following:
\begin{enumerate}
  \item Change the current directory to the main \VMD\ working
directory.
  \item Edit the file {\tt configure.options}; make sure the {\tt
MAKE} option is not included as an item in the list.  If you have
previously compiled a new version of \VMD\ as described in chapter
\ref{chapter:ig:compile}, and are now simply installing the resulting
new executable and data files, leave all other options in this file
unchanged.  If a precompiled executable is being used, change this
file to contain a single line identical to the `Configuration Options'
line in table \ref{table:ig:execs} for the relevant executable. 
  \item Type the command {\tt configure} to create a new Makefile.
\end{enumerate}

\section{Installing files}
\label{section:ig:installfile}

The penultimate step in installing \VMD\ is to run the install command. 
Once the makefile has been properly created with the required
installation parameters, change to the {\tt src} directory and type
the command:
\begin{quote} {\tt
  make install
} \end{quote}
If either of the directories {\tt INSTALLLIBDIR} or {\tt
INSTALLBINDIR} do not exist, the installation procedure will
attempt to create them; this may require the installation command to
be run with superuser access if the parent directories are in system
locations.

If you wish to have copies of the \VMD\ documentation guides installed
into the {\tt INSTALLLIBDIR} directory (in a subdirectory there named
{\tt doc}), do the following:
\begin{quote} {\tt
  make install.doc
} \end{quote}


\section{Checking perl files}
\label{section:ig:perl}

\VMD\ uses one perl script, {\tt url\_get}, for retrieving web documents
(used in the ``mol pdbload'', ``mol load webpdb'', ``mol urlload'', and ``url\_get''
commands).  The default configuration assumes the perl interpreter is
located in {\tt /usr/local/bin/perl}, but it might not be that way on
your system.  Check and possibly edit the first line of the script
\$INSTALLLIBDIR/scripts/vmd/url\_get to point to the correct location.
On some SGIs this may be in {\tt /usr/sbin/perl}.

\section{Running \VMD}
\label{section:ig:running}

Once installed, type {\tt vmd} (or whatever value for {\tt INSTALLNAME} was
used in the installation) to run the program.  This will run an {\tt xterm}
process to act as the \VMD\ console.  The Users Guide provides a complete
description of how to actually use \VMD\ to display molecules.

The environment variable {\tt VMDDIR} may be used to specify another
directory which contains \VMD\ data files and executables, if the {\tt
INSTALLLIBDIR} directory files are not adequate.  Or, special data
files may be customized for each user or for all users, as described
in chapter \ref{chapter:ig:custom}.

