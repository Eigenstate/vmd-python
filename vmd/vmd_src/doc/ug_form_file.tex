%%%%%%%%%%%%%%%%%%%%%%%%%%%%%%%%%%%%%%%%%%%%%%%%%%%%%%%%%%%%%%%%%%%%%%%%%%%%
% RCS INFORMATION:
%
%       $RCSfile: ug_form_file.tex,v $
%       $Author: johns $        $Locker:  $                $State: Exp $
%       $Revision: 1.23 $      $Date: 2012/01/10 19:30:04 $
%
%%%%%%%%%%%%%%%%%%%%%%%%%%%%%%%%%%%%%%%%%%%%%%%%%%%%%%%%%%%%%%%%%%%%%%%%%%%%
% DESCRIPTION:
%  window for reading in files
%
%%%%%%%%%%%%%%%%%%%%%%%%%%%%%%%%%%%%%%%%%%%%%%%%%%%%%%%%%%%%%%%%%%%%%%%%%%%%


\subsection{Molecule File Browser Window}
\label{ug:ui:window:files}
\index{window!molecule file browser}
\index{files!reading}

\begin{rawhtml}
<CENTER>
\end{rawhtml}
\myfigure{ug_files}{The Molecule File Browser window}{fig:ug:files}
\begin{rawhtml}
</CENTER>
\end{rawhtml}

The {\sf Files} window is used to load a file from disk into a new or 
existing \VMD\  molecule. It can be brought up by choosing {\sf New 
Molecule\ldots} from the {\sf File} menu, or by hilighting a molecule in 
the \hyperref{{\sf Main} window}{{\sf Main} window}{}{ug:ui:window:main} and 
choosing the {\sf Load Data Into Molecule\ldots} menu item. 
Once the window appears, select the file you want by using the file 
browser or by typing the filename into the text entry area.
By default \VMD\ will try to guess the type of file you are loading
by matching the filename extension with one of the file reader plugins 
in the file type list (the available file types are described 
in Chapter \ref{ug:topic:filetypes}).
If \VMD\ is unable to guess the appropriate file type or guesses
incorrectly, you must select it from the list manually.

You can control into which \VMD\ molecule you want to load your data by 
selecting it from the {\sf Load files for:} popup menu at the top of the 
window.
If the file being loaded is intended for a new molecule, select {\sf New 
Molecule} instead. If the file being loaded
contains additional coordinate frames, electron density map, or other
ancillary data for an existing molecule, choose the appropriate molecule from
the selection list at the top of the window.  
If the file being loaded contains trajectory \timesteps, you have the 
option of loading a subset of the trajectory skipping ranges or strides
of \timesteps rather than the whole thing.  You can also select for \VMD\
to load all \timesteps before continuing on, or to load them in the background
so that you may continue to interact with the menus and windows while it loads
additional \timesteps.  
If the file being loaded contains multiple volumetric data, you may 
select which data sets you would like to load.

Once you have selected the file to be loaded, the appropriate file type,
and the way it will be loaded, press the {\sf Load} button and \VMD\ will
being loading the selected file.  Any informational messages, 
errors or warnings which occur while loading the file will appear 
in the text window.


\subsubsection{Reading Trajectory Frames}
\label{ug:ui:window:edit:read}
\index{animation!read}
\index{animation!appending}
\index{files!reading}
\index{file types!input}
\index{trajectory!read}
\index{AMBER!files}
\index{CHARMM!files}
\index{NAMD!files}
\index{Gromacs!files}
\index{XPLOR!files}
\VMD\ can read in new coordinate sets from one of several file
formats such as PDB, CRD, DCD, or Gromacs files.
The new coordinate sets are appended to the end of the
stored \timesteps for the selected molecule.  
Loading coordinate data is like loading any other file, select it
with the file browser make sure the file type is set correctly for
the file being loaded, and then press the {\sf Load} button.

By default, VMD will load all of the \timesteps contained in a 
coordinate or trajectory file. 

Sometimes you may not want to read in a whole coordinate or
trajectory file.  For example, you may only want the last frame, 
or every tenth frame.  You can do this by changing the options in the 
{\sf \Timesteps} control of Files window.
\label{ug:ui:window:edit:amount}
\index{animation!\timesteps}
The {\sf \Timesteps} controls consist of three numeric input fields 
labeled {\sf First}, {\sf Last}, and {\sf Stride}.  These make it
possible to use a subset of the frames, starting at frame {\sf First}
and selecting every {\sf Stride} frames until the {\sf Last} is reached.
For instance, to select every fifth frame between frames 14 and 98,
set:

\begin{itemize}
  \item{{\sf First} to 14}
  \item{{\sf Last} to 98}
  \item{{\sf Stride} to 5}
\end{itemize}

(Remember that frame numbers in \VMD\ start at 0, so frame 0 is the
first frame.)  The value `-1' is a special number; setting {\sf First}
to -1 is the same as starting at the first frame, {\sf Last} = -1 is
the same as ending at the last frame, and {\sf Stride} = -1 is the same
as taking one step.

