%%%%%%%%%%%%%%%%%%%%%%%%%%%%%%%%%%%%%%%%%%%%%%%%%%%%%%%%%%%%%%%%%%%%%%%%%%%%
% RCS INFORMATION:
%
%       $RCSfile: ug_form_render.tex,v $
%       $Author: johns $        $Locker:  $                $State: Exp $
%       $Revision: 1.21 $      $Date: 2012/01/10 19:30:05 $
%
%%%%%%%%%%%%%%%%%%%%%%%%%%%%%%%%%%%%%%%%%%%%%%%%%%%%%%%%%%%%%%%%%%%%%%%%%%%%
% DESCRIPTION:
%  render form
%
%%%%%%%%%%%%%%%%%%%%%%%%%%%%%%%%%%%%%%%%%%%%%%%%%%%%%%%%%%%%%%%%%%%%%%%%%%%%


\subsection{Render Window}
\label{ug:ui:window:render}
\index{window!render}
\index{render!window}
\index{rendering}

The Render window is used to export the currently displayed graphics scene to
an image file or to a geometric scene description file suitable for use
by one of several external renderers, which can produce a final image.
The supported rendering packages are listed in table \ref{ug:table:render}.  
See Chapter \ref{ug:topic:rendering}
for detailed information on how rendering is performed using 
external programs, as well as information on 3-D printing and 
other uses of the exported scene description files.

\begin{rawhtml}
<CENTER>
\end{rawhtml}
\myfigure{ug_render}{The Render window}{fig:ug:render}
\begin{rawhtml}
</CENTER>
\end{rawhtml}

The rendering process works in two stages.  The first stage
exports the displayed \VMD\ scene to a text or image file
in the selected format.
The second (optional) stage renders the exported file,
potentially displaying the results when complete.  
The exported file is named in the {\sf Filename} field; 
a default name is given when a new format is selected, 
so it is best to hold off entering the filename until after
the file format is selected.  Another way to select the filename is
available by pressing the {\sf Browse}  button, which opens up a file
browser.
Pressing the {\sf Start Rendering} button writes the data file.  After
that, the {\sf Render Command} is executed.  The default command
should start the appropriate rendering program if it is available.

Some of the rendering commands have been set to call a display 
program on the rendered image when it is completed.
\VMD\ will wait for the display program to finish, which causes \VMD\ 
to freeze until the display program closes, 
so you may want to run the job in the background.  This can be done 
(on Unix) by enclosing the existing text with {\tt ()}'s and putting
an {\tt \&} at the end.  For example, the way to make the Raster3D render 
command run in the background is:
\index{rendering!in background process}
{\tt 
\begin{verbatim}
        (render < %s -sgi %s.rgb; ipaste %s.rgb)&
\end{verbatim}
}
