%%%%%%%%%%%%%%%%%%%%%%%%%%%%%%%%%%%%%%%%%%%%%%%%%%%%%%%%%%%%%%%%%%%%%%%%%%%%
% RCS INFORMATION:
%
%       $RCSfile: ug_mol_selection.tex,v $
%       $Author: johns $        $Locker:  $                $State: Exp $
%       $Revision: 1.37 $      $Date: 2014/12/29 02:51:54 $
%
%%%%%%%%%%%%%%%%%%%%%%%%%%%%%%%%%%%%%%%%%%%%%%%%%%%%%%%%%%%%%%%%%%%%%%%%%%%%
% DESCRIPTION:
%  description of the selection command
%
%%%%%%%%%%%%%%%%%%%%%%%%%%%%%%%%%%%%%%%%%%%%%%%%%%%%%%%%%%%%%%%%%%%%%%%%%%%%


\section{Selection Methods}
\label{ug:topic:selections}
\index{atom!selection}

\VMD\ has a rather powerful atom selection language available.  It is
based around the assumption that every atom has a set of associated with it 
values which can be
accessed through \index{selection!keywords}
\index{atom!selection!keywords}
keywords.  These values could be
boolean (is this a protein atom?), numeric (as in the atom
index or atomic mass), or string (the atom name).  The values can
even be referenced via a Tcl array.

To start off, here are some examples of valid selection commands in
\VMD.  Following these will be a more in depth description of how
selections work.

\index{atom!selection!examples}
\begin{verbatim}
        name CA
        resid 35
        name CA and resname ALA
        backbone
        not protein
        protein (backbone or name H)
        name 'A 1'
        name 'A *'
        name "C.*"
        mass < 5
        numbonds = 2
        abs(charge) > 1
        x < 6 and x > 3
        sqr(x-5)+sqr(y+4)+sqr(z) > sqr(5)
        within 5 of name FE
        exwithin 3 of protein
        protein within 5 of nucleic
        same resname as (protein within 5 of nucleic)
        protein sequence "C..C"
        name eq $atomname 
\end{verbatim}

\index{selection!modes}
\index{atom!selection!modes}
There are two types of selection modes.  The first is a keyword
followed by a list of either values or a range of values.  For example,
\begin{verbatim}
        name CA
\end{verbatim}
selects all atoms with the name CA (which could be a C${}_\alpha$ or a calcium);
\begin{verbatim}
        resname ALA PHE ASP
\end{verbatim}
selects all atoms in either alanine, phenylalanine, or asparagine;
\begin{verbatim}
        index 5
\end{verbatim}
selects the 6th atom (in the internal \VMD\ numbering scheme).

\VMD\ can also do range selections, similar to X-PLOR's `:' notation:
\begin{verbatim}
        mass 5 to 11.5
\end{verbatim}
selects atoms with mass between 5 and 11.5 inclusive,
\begin{verbatim}
        resname ALA to CYS TYR
\end{verbatim}
selects atoms in alanine, arginine, asparagine, aspartic acid, cystine,
and also tyrosine.

The keyword selection works by checking each term on the list
following the keyword.  The term is either a single word (eg,
\verb!name CA!) or a range (eg \verb!resid 35 to 90!).

The method for determining the range checking is determined from the
keyword data type; numeric comparisons are different than string
comparisons.  The comparison should work as expected so that ``8'' is
between ``1'' and ``11'' in a numeric context but not in a string one.
This may lead to some peculiar problems.  Some keywords, such as {\tt
segname}, can take on string values but can also be used by some
people as a number field.  Suppose someone labeled the {\tt segname}
field with the numbers 1 through 12 on the assumption that they are
numbers.  That person would be rather confused to find that
\verb!segname 1 to 11!  only returns two segments.  Also, strings
will be converted (via {\tt atof()}) to a number so if the string
isn't a number, it will be given the value of 0.  It is possible to
force a search to be done in either a string or numeric context
using the
\hyperref{relational operator}{relational operator discussed in \S }{}
         {ug:topic:selections:comparison}

Selections can be combined with the boolean operators {\tt
and} and {\tt or}, collected inside of parenthesis, and modified by
{\tt not}, as in
\begin{verbatim}
        (name CA or name CB) and mass 12 to 17
\end{verbatim}
which selects all atoms name CA or CB and have masses between 
12 and 17 amu (this could be used to distinguish a C-alpha from a
calcium).  \VMD\ has operator precedence similar to C so leaving the parentheis out of the previous expression, as in:
\begin{verbatim}
        name CA or name CB and mass 12 to 17
\end{verbatim}
actually selects all atoms named CA or those that are named CB and
have the appropriate mass.

\subsection{Definition of Keywords and Functions}
The keywords available for selecting atoms in \VMD\ are listed in tables
\ref{table:ug:keywords} and \ref{table:ug:keywords:cont} at the end of
this chapter.
If a keyword definition is followed by {\it bool}, it is either
on or off.  If followed by {\it str} it takes a value in the
string context.  If followed by {\it num} it takes a value in the
number context.


\begin{table}[htb]
\hspace{0.25in}
%  \begin{tabular}{|l|l|l|} \hline
%    \multicolumn{1}{|c}{Keyword} & 
%        \multicolumn{1}{|c}{Arg} & 
%        \multicolumn{1}{|c|}{Description} \\ \hline\hline
\begin{tabular}{|p{.9 in}| p{.4 in}| p{4.5in}| } 
\hline
%\multicolumn{3}{|l}{} \\ 
Keyword & Arg & Description \\ 
\hline \hline
all           & {\it bool}  & everything \\
none          & {\it bool}  & nothing \\
name          & {\it str}   & atom name \\
type          & {\it str}   & atom type \\
index         & {\it num}   & the atom number, starting at 0 \\
serial        & {\it num}   & the atom number, starting at 1 \\
atomicnumber  & {\it num}   & atomic number (0 if undefined) \\
element       & {\it str}   & atomic element symbol string ('X' if undefined) \\
altloc        & {\it str}   & alternate location/conformation identifier \\
chain         & {\it str}   & the one-character chain identifier \\
residue       & {\it num}   & a set of connected atoms with the same residue number \\
protein       & {\it bool}  & a residue with atoms named {\tt C, N, CA,} and {\tt O} \\
nucleic       & {\it bool}  & a residue with atoms named {\tt P, O1P, O2P} and either \\
              &             & {\tt O3', C3', C4', C5', O5'} or {\tt O3*, C3*, C4*, C5*, O5*}. \\
              &             & This definition assumes that  the base is phosphorylated, \\
              &             & an assumption which will be corrected in the future. \\
backbone      & {\it bool}  & the {\tt C}, {\tt N}, {\tt CA}, and {\tt O} atoms of a protein \\
              &             & and the equivalent atoms in a nucleic acid. \\
sidechain     & {\it bool}  & non-backbone atoms and bonds \\
water,        & {\it bool}  & all atoms with the resname {\tt H2O, HH0, OHH, HOH, } \\
waters        &             & {\tt OH2, SOL, WAT, TIP, TIP2, TIP3} or {\tt TIP4} \\
fragment      & {\it num}   & a set of connected residues \\
pfrag         & {\it num}   & a set of connected protein residues \\
nfrag         & {\it num}   & a set of connected nucleic residues \\
sequence      & {\it str}   & a sequence given by one letter names \\
numbonds      & {\it num}   & number of bonds \\
resname       & {\it str}   & residue name \\
resid         & {\it num}   & residue id \\
segname       & {\it str}   & segment name \\
x, y, z       & {\it float} & x, y, or z coordinates \\
radius        & {\it float} & atomic radius \\
mass          & {\it float} & atomic mass \\
charge        & {\it float} & atomic charge \\
beta          & {\it float} & temperature factor \\
occupancy     & {\it float} & occupancy \\ 
user          & {\it float} & time-varying user-specified value \\
at            & {\it bool}  & residues named  {\tt ADA A THY T} \\
acidic        & {\it bool}  & residues named {\tt ASP GLU} \\
acyclic       & {\it bool}  & ``protein and not cyclic'' \\
aliphatic     & {\it bool}  & residues named  {\tt ALA GLY ILE LEU VAL} \\
alpha         & {\it bool}  & atom's residue is an alpha helix \\
amino         & {\it bool}  & a residue with atoms named {\tt C, N, CA}, and {\tt O} \\
aromatic      & {\it bool}  & residues named {\tt HIS PHE TRP TYR} \\
basic         & {\it bool}  & residues named {\tt ARG HIS LYS} \\
bonded        & {\it bool}  & atoms for which numbonds {\tt >} 0 \\
buried        & {\it bool}  & residues named {\tt ALA LEU VAL ILE PHE CYS MET TRP} \\
cg            & {\it bool}  & residues named {\tt CYT C GUA G} \\
charged       & {\it bool}  & ``basic or acidic'' \\
cyclic        & {\it bool}  & residues named {\tt HIS PHE PRO TRP TYR} \\
\hline
\end{tabular}
\caption{Atom selection keywords.}
\label{table:ug:keywords}
\index{atom!selection!keywords} 
\index{selection!keywords}
\end{table}

\newpage
\begin{table}[htb]
\hspace{0.5in}
\begin{tabular}{|p{.9 in}| p{.4 in}| p{4.8in}| } 
\hline
Keyword & Arg & Description \\ 
\hline \hline
hetero         & {\it bool}  & ``not (protein or nucleic)'' \\
hydrogen       & {\it bool}  & name "[0-9]?H.*" \\
large          & {\it bool}  & ``protein and not (small or medium)'' \\
medium         & {\it bool}  & residues named {\tt VAL THR ASP ASN PRO CYS} \\
               &             & {\tt ASX PCA HYP} \\
neutral        & {\it bool}  & residues named {\tt VAL PHE GLN TYR HIS CYS} \\
               &             & {\tt MET TRP ASX GLX PCA HYP} \\
polar          & {\it bool}  & ``protein and not hydrophobic'' \\
purine         & {\it bool}  & residues named {\tt ADE A GUA G} \\
pyrimidine     & {\it bool}  & residues named {\tt CYT C THY T URI U} \\
small          & {\it bool}  & residues named {\tt ALA GLY SER} \\
surface        & {\it bool}  & ``protein and not buried'' \\
rasmol         & {\it str}   & translates Rasmol selection string to VMD \\
alpha\_helix   & {\it bool}  & atom's residue is in an alpha helix  \\
pi\_helix      & {\it bool}  & atom's residue is in a pi helix   \\
helix\_3\_10   & {\it bool}  & atom's residue is in a 3-10 helix \\
helix          & {\it bool}  & atom's residue is in an alpha or pi or 3-10 helix \\
extended\_beta & {\it bool}  & atom's residue is a beta sheet \\
bridge\_beta   & {\it bool}  & atom's residue is a beta sheet \\
sheet          & {\it bool}  & atom's residue is a beta sheet \\
turn           & {\it bool}  & atom's residue is in a turn conformation \\
coil           & {\it bool}  & atom's residue is in a coil conformation  \\
structure      & {\it str}   & single letter name for the secondary structure \\
phi, psi       & {\it float} & backbone conformational angles \\
within         & {\it str}   & selects atoms within a specified distance of \\
               &             & a selection (i.e {\tt within 5 of name FE}). \\
exwithin       & {\it str}   & exclusive within, equivalent to {\tt(within 3 of X) and not X}. \\
same           & {\it str}   & selects atoms which have the same keyword as \\
               &             & the atoms in a given selection (i.e. {\tt same segname as resid 35}) \\
ufx, ufy, ufz  & {\it num}   & force to apply in the x, y, or z coordinates \\
                \hline
\end{tabular}
\caption{Atom selection keywords (continued).}
\label{table:ug:keywords:cont}
\index{atom!selection!keywords} 
\end{table}


Table \ref{table:ug:functions} lists the built-in functions which may
be used in atom selection expressions with keywords which take on a
numeric value.

\begin{table}[htb]
\hspace{1.5in}
\begin{tabular}{|l|l|} \hline
\multicolumn{1}{|c}{Function} & 
\multicolumn{1}{|c|}{Description} \\ \hline\hline
    sqr(x)      & square of x \\
    sqrt(x)     & square root of x \\
    abs(x)      & absolute value of x \\
    floor(x)    & largest integer not greater than x \\
    ceil(x)     & smallest integer not less than x \\
    sin(x)      & sine of x \\
    cos(x)      & cosine of x \\
    tan(x)      & tangent of x \\
    atan(x)     & arctangent of x \\
    asin(x)     & arcsin of x \\
    acos(x)     & arccos of x \\
    sinh(x)     & hyperbolic sine of x \\
    cosh(x)     & hyperbolic cosine of x \\
    tanh(x)     & hyperbolic tangent of x \\
    exp(x)      & ``e to the power x'' \\
    log(x)      & natural log of x \\
    log10(x)    & log base 10 of x \\ \hline
\end{tabular}
\caption{Atom selection functions.}
\label{table:ug:functions}
\index{atom!selection!math functions}
\index{selection!math functions}
\end{table}


Table \ref{table:ug:volkeywords} lists the built-in atom selection 
keywords which may be used in atom selection expressions to query the 
values of an underlying volumetric map in the same molecule. 
These are read-only.

\begin{table}[htb]
\begin{center}
\begin{tabular}{|l|c|l|} 
  \hline
  Function & Arg & Description\\ 
  \hline\hline
  vol$N$       & \emph{float} & value of the voxel of the volumetric data of ID $N$ \\
               & & nearest to the atom \\
  interpvol$N$ & \emph{float} & interpolated value of the voxels of the volumetric \\
               & & data of ID $N$ around the atom \\ 
  \hline
\end{tabular}
\end{center}
\caption[Read-only atom selection keywords for volumetric map queries]{Read-only atom selection keywords which may be used to query the values
of an underlying volumetric map in the same molecule. The value of $N$, which
can be 0 to 7 inclusively, refers to the volID of the underlying volumetric data
(\emph{e.g.}, you could type {\tt interpvol2}).}{Read-only atom selection
keywords for querying volumetric data}
\label{table:ug:volkeywords}
\index{atom!selection!volumetric data}
\index{selection!volumetric data}
\end{table}


\subsection{Boolean Keywords}

\index{selection!keywords!boolean}
Some selections take no values.  For example, {\tt backbone}
selects the backbone atoms of the protein and nucleic acids and
{\tt protein} selects protein atoms.
Giving options to these selections is an error.
The selections can be used in the same way as other selections, as in:
\begin{verbatim}
        protein and backbone
        nucleic or protein
\end{verbatim}
In addition, if neither {\tt and} nor {\tt or} are located 
after a boolean keyword, then an implicit {\tt and} is inserted, so
that the following are valid:
\begin{verbatim}
        protein name CA         (same as: protein and name CA)
        nucleic backbone
\end{verbatim}

\subsection{Short Circuiting}
\label{ug:topic:selection:short_circuits}
\index{atom!selection!logic}
\index{selection!logic}

The boolean logic in \VMD\ does
\index{short circuit logic}{\it short circuit} evaluation on an 
element-wise basis.  For instance, given one atom, if {\tt X} is true
then {\tt X or Y} will be true regradless of the value of {\tt Y}, so there
is no need to evaluate it.  Similarly, if {\tt X} is false, then {\tt
X and Y} will also be false, so {\tt Y} again need not be evaluated.

Knowing how short circuit selections work can speed up several types
of selections.  Consider a system with a large number of waters and a
protein.  The expression \verb!protein and segname < 10! is faster
than \verb!segname < 10 and protein! since in the first selection only the
atoms which are proteins have the {\tt segname} converted to a number,
while in the second selection, all the segment names are converted.

The {\tt within} selection has its own form of short circuiting.  The
command can be interpreted as ``find the atoms of A which are withing
a given distance from B,'' and if A isn't given, search all the atoms.
The search done in \VMD\ takes a time roughly proportional to the
number of atoms in A multiplied by the number of atoms in B, so
reducing the number of atoms in A (i.e., by not testing every atoms)
make the search faster.

Using the system with a lot of water and a protein, compare the
selection \\ 
\verb!    protein within 5 of resid 1! \\
to
\verb!    (within 5 of resid 1) and protein!. \\
The first is very fast as it 
does a distance search between all the protein atoms and all the atoms
in resid 1.  However, the second selection searches through all the
atoms for those which are within 5 \AA\ of resid and then finds which
of those are protein atoms.


\subsection{Quoting with Single Quotes}
\index{quoting}
\index{atom!selection!quoting}
  \VMD\ allows two types of quoting mechanisms, single and
double quotes.  Single quotes are used to include spaces and other
non-alphanumeric characters.  Believe it or not, there are some
residue names with a space in them, so they can be referenced as, for
example, 

\begin{verbatim}
        resname 'A 1'
\end{verbatim}

  More importantly, ribose atoms can be given names like {\tt
C5'} or {\tt C5*} (depending on the age of the PDB record).  The lexer
in \VMD\ has been modified so that {\tt C5'}, {\tt O"}, and {\tt N''}
can be used without quotes, but it cannot handle an
unquoted asterisk ({\tt *} conflicts with multiplication and the parser
is not able to resolve the difference).  Some examples are:

\begin{verbatim}
        name 'O5*'
        segname 'A *'
        name O5'
\end{verbatim}

  Quotes may also be used to get around a reserved selection
word, like {\tt x}.  The selection command {\tt segname x} will give
an error because {\tt x} is another keyword.  Instead, use {\tt
segname 'x'}.  There is an escape mechanism for including single
quotes inside a single quoted string which uses a backslash
('\verb!\!') before the single quote.  This allows unusual names like
{\tt C '} to be quoted as \verb!'C \''!.

\begin{verbatim}
        segname x      <---- error; conflicts with the 'x' keyword
        segname 'x'
        name 'O5\''
\end{verbatim}

\noindent Also, double quotes (discussed in the next section) can be used, 
as in \verb!"C '"! or \verb!"C \*"!.

\subsection{Double Quotes and Regular Expressions}
\label{ug:topic:selections:regex}
\index{atom!selection!regular expression}
Double quotes around a string are used to specify a \index{regular
expression}regular expression search (compatible with Perl 5.005, using 
the Perl-compatible regular expressions library written by Philip Hazel).
If you don't know how to use them, try consulting the man pages for {\tt ed},
{\tt egrep}, {\tt vi}, or {\tt regex}.  If not, read the Perl docs, or get
any one of a number of books including the O'Reilly and Associates Sed
and Awk book.  The following examples show just a few ways that regular
expressions can be used within \VMD.

\noindent Selection of all atoms with a name starting with {\tt C}:

\begin{verbatim}
        name "C.*"
\end{verbatim}

\noindent Segment names containing a number:

\begin{verbatim}
        segname ".*[0-9]+.*"
\end{verbatim}

\noindent Multiple terms can be provided on the list of
matching keywords.  This example selects residues starting with 
an {\tt A}, the glycine residues, and residues ending with a {\tt T}.  
As with a string, a regular expression in a numeric context gets 
converted to an integer, which will always be zero:

\begin{verbatim}
        resname "A.*" GLY ".*T"
\end{verbatim}


\noindent Selections containing special characters such as $+$, $-$,
or $*$, must be escaped with the \verb$\$ character.  In order to select
atoms named \verb$Na+$, one would use the selection:
   
\begin{verbatim}
        name "Na\+"
\end{verbatim}

In brief, a regular selection allows matching to multiple
possibilities, instead of just one character.  Table
\ref{table:ug:wildcards} shows some of the methods that can be used.

\begin{table}[htb]
\hspace{0.5in}
\begin{tabular}{|c|c|l|} \hline
Symbol      & Example & \multicolumn{1}{|c|}{Definition} \\ 
\hline\hline
.           & \verb!.  , A.C !              & match any character   \\
{\tt []}    & \verb![ABCabc] , [A-Ca-c]!    & match any char in the list  \\
\verb![~]!  & \verb![~Z] , [~XYZ] , [^x-z]! & match all except the chars in the list    \\
\verb!^!    & \verb!^C , ^A.*!              & next token must be the first part of string \\
\$          & \verb![CO]G$!                 & prev token must be the last part of string \\
\verb!*!    & \verb!C* , [ab]*!             & match 0 or more copies of prev char or \\
            &                               & regular expression token \\
+           & \verb!C+ , [ab]+!             & match 1 or more copies of the prev token \\
\verb!\|!   & \verb!C\|O!                   & match either the 1st token or the 2nd \\
\verb!\(\)! & \verb!\(CA\)+!                & combines multiple tokens into one \\ \hline
\end{tabular}
\caption{Regular expression methods.}
\label{table:ug:wildcards}
\index{regular expression}
\end{table}


There are many ways to do some selections.  For example, choosing
atoms with a name of either CA or CB can be done in the following ways:

\begin{verbatim}
        name CA CB
        name "CA|CB"
        name "C[AB]"
        name "C(A|B)"
\end{verbatim}

Several caveats for those who already understand regular expressions.
\VMD\ automatically prepends ``\verb!^(!'' and appends
``\verb!)$!'' to the selection string.  This makes the selection {\tt
O} match only {\tt O} and not {\tt OG} or {\tt PRO}.  On the other
hand, putting \verb!^!  and \verb!$! into the command won't really
affect anything, selections that match on a substring must be
preceded and followed by ``.*'', as in \verb!.*O.*!, and some illegal
selections could be accepted as correct, but strange, as in
\verb!C)|(O!
, which gets converted to
\verb!^(C)|(O)$!
and matches anything starting with a C or ending with an O.

A regular expression is similar to wildcard matching in X-PLOR.  Table
\ref{table:ug:conversions} is a list of conversions from X-PLOR style
wildcards to the matching regular expression.

\begin{table}[htb]
\hspace{1.1in}
\begin{tabular}{|c|c|c|} \hline
X-PLOR Wildcard & Description & Regular Expression \\ 
\hline\hline
*               & matches any string         & .*         \\
\%              & matches a single character & .          \\
+               & matches any digit          & [0-9]      \\
\#              & matches any number         & [0-9]+     \\ 
\hline
\end{tabular}
\caption{Regular expression conversions.}
\label{table:ug:conversions}
\index{regular expression!X-PLOR conversion}
\end{table}


\subsection{Comparison selections}
\label{ug:topic:selections:comparison}
\index{selection!comparison}
\index{atom!selection!comparison}
Comparisons can be used in \VMD\ to do atom selections like {\tt mass
{\tt <} 5}, which selects atoms with mass less than 5 amu, and {\tt
name eq CA}, which is another way of choosing the CA atoms.  The
underlying idea for the comparison selection is also based on the
concept that every atom has a property as specified by a keyword.
When the keyword is given in the expression, the array (or vector) of
the corresponding values is constructed, and the size of the array is
the same as the number of atoms in the molecule.  (If a single number
or string is given instead of a keyword, the array consists of copies
of that given value.)  The operations, like addition, multiplication,
string matching, and comparison, are then applied element-wise along
the array.  This type of selection is similar to the vector statement
in X-PLOR.

Take the example {\tt mass {\tt <} 5} when applied on water, which has
an oxygen of mass 15.9994 and two hydrogens of mass 1.008.  \VMD\ sees
the keyword {\bf mass} and constructs the array [15.9994, 1.008,
1.008], then sees the ``5'' and makes the array [5, 5, 5].  It then
compares each term of the array and returns with the boolean array
[False, True, True] (since 15.9994 is not less than 5, but 1.008 is).
This final boolean array is then used to determine which atoms are
selected; in this case, the hydrogens.

More complicated comparison selections can be constructed, either from
arithmetic operations or by using some of the standard math functions
(the functions are listed in Table~\ref{table:ug:functions}).  
Probably the most often used
function will be {\tt sqr}, which squares each element of the array.
Thus, the command to select all atoms within 5 \AA\ of a point (x,y,z)
= (3,4,-5) in space is:

\begin{verbatim}
        sqr(x-3)+sqr(y-4)+sqr(z+5) <= sqr(5)
\end{verbatim}

\subsection{Comparison Operators}

There are two types of comparison operators --- numeric and string ---
which allow the user to specify the appropriate comparison function.
Suppose the segment name, which takes on a string value, contains the
names `11', and `8'.  \VMD\ cannot figure out if `8' should be less
than `11' (in the numeric sense) or greater than `11' (in the
lexographical sense).  Instead of trying to resolve this problem
through some sort of internal heuristics, \VMD\ leaves it up to the
user so that {\tt 8 < 11} but {\tt 11 lt 8}.  (Perl users should
recognize this solution.)

The numeric comparisons are the standard ones: {\tt <}, {\tt <=}, {\tt
=} or {\tt ==}, {\tt >=}, {\tt >}, and {\tt !=}.  The corresponding
string comparisons are: {\tt lt}, {\tt le}, {\tt eq}, {\tt ge}, {\tt gt}, 
and {\tt ne}.  As in perl there
is a ``match'' operator, \verb!=~!, so that

\begin{verbatim}
        'CA' =~ "C.*"
        segname =~ "VP[1-4]" (matches VP1, VP2, VP3, and VP4, present in some
                              virus structures)
\end{verbatim}

\noindent are valid.  No distinction is made between single and double quotes.

\subsection{Other selections}
\subsubsection{sequence}
\index{atom!selection!sequence}
  VMD supports selection based on the one-letter amino acid
sequence with the \verb!sequence! selection keyword.  This allows
selections of the form
\begin{verbatim}
  sequence APD
  sequence "C..C"       (might be used to pick out zinc fingers)
  sequence AATCGGAT
\end{verbatim}

Unlike the other string selection commands which take one of three
types of strings, all the strings for \verb!sequence! are taken as
regular expressions (though strings with non-alphanumerics must still
be quoted to get past the input parser).  The method works by taking
each of the protein and nucleic acid fragments (pfrag and nfrag) 
in turn and constructing the one-letter
amino acid sequence.  If a regular expression matches any of the
sequence, the atoms in the matching residues are selected.  Multiple
matches are allowed, though they cannot overlap.  As is usual with
regular expressions, the largest possible match is made, so take care
with expressions like
\verb!C.*C!.

\subsubsection{{\tt within} and {\tt same}}
\index{atom!selection!within}
\index{atom!selection!same}
Two useful types of selection mechanisms available in \VMD are:
\verb!within <number> of <selection>! 
and \verb!same <keyword> as <selection>!.  The first selects all atoms 
within the specified
distance (in \AA) from a selection, including the selection itself.
Therefore, the command:

\begin{verbatim}
        within 5 of name FE
\end{verbatim}

\noindent selects all atoms within 5 \AA\ of atoms named FE.  One common use for
this command is to limit the region of atoms shown on the screen.
Another is to find atoms that may be involved in interactions.  For
instance:

\begin{verbatim}
        protein within 5 of nucleic
\end{verbatim}

\noindent finds the protein atoms that are nearby nucleic acids.  
Some selections may be sped up
by 
\hyperref{short circuiting}{short circuiting [\S }{]}
         {ug:topic:selection:short_circuits}.
\index{short circuit logic}

A related atom selection construct is {\tt exwithin}, short for
'exclusive within'.  The atom selection {\tt (within 3 of protein) and not 
protein} is equivalent to {\tt exwithin 3 of protein}. 

  The \verb!same <keyword> as <selection>! finds all the atoms
which have the same `keyword' as the atoms in the selection.  This can
be used for selections like
\begin{verbatim}
        same fragment as resid 35
\end{verbatim}
\noindent which finds all the atoms attached to residue id 35.  Any keyword 
can be used, so selections like
\begin{verbatim}
        same resname as (protein within 5 of nucleic)
\end{verbatim}
\noindent are fine, although weird.  The perhaps the most useful keyword
for this command is {\tt residue}, so you can say {\tt same residue as
\ldots}.

\subsubsection{Finding contact residues}

\index{contact residues}
Suppose you want to view the atoms in ``A'' which are in contact with
``B''.  Use the {\tt within <distance> of <selection>} selection
command.  For purposes of demonstration, let A be protein, B be
nucleic, and define contact as an atom in A which is within 2 \AA\ of
an atom in B.  Then the selection command is

\begin{verbatim}
        protein within 2 of nucleic
\end{verbatim}

        If you want to see all the residues of A which have at least
one atom in contact with B, use

\begin{verbatim}
        same residue as (protein within 2 of nucleic)
\end{verbatim}

