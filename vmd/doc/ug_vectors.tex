\chapter{Vectors and Matrices}
\label{ug:topic:vectors}
\index{vector routines}

Tcl does not handle mathematical expressions very well.  It is slow at
evaluating expressions, and provides no facility for handling vectors
or matrices.  Since the latter two are needed for structure analysis,
we have added routines to manipulate them.  

A vector in \VMD\ is a list of numbers.  All of the vector routines
but one will work with vectors of any length; {\tt veccross} will only
use vectors of three numbers.  A matrix is a 4x4 collection of numbers
stored as a list of 4 vectors of 4 numbers, in row-major order.

Following are descriptions and examples of all the commands.  For more
examples of vectors, though without much documentation, the script
used to test the vectors implementation is located at {\tt
\$env(VMDDIR)/scripts/vmd/test-vectors.tcl}.

Since Tcl is slow at math, some of these commands have been
reimplemented in C++.  (The original definition is in the vmd script
distribution, but it is redefined later on inside \VMD).  At times,
the speedup is a factor of 40 or more.  These commands are noted by
(C++).

\section{Vectors}

\begin{itemize}
\item {\bf veczero} --
\index{vector command!veczero}
Returns the zero vector, \{0 0 0\}
\begin{verbatim}
Example:
vmd > veczero
0 0 0
\end{verbatim}

\item (C++) {\bf vecadd {\it v1 v2} [{\it v3 ... vn}]} --
Returns the vector sum of all the terms.
\index{vector command!vecadd}
\begin{verbatim}
Examples:
vmd > vecadd {1 2 3} {4 5 6} {7 8 9} {-11 -11 -11}
1 4 7
vmd > vecadd {0.1 0.2 0.4 0.8} {1 1 2 3} {3 1 4 1}
4.1 2.2 6.4 4.8
vmd > vecadd 4 5
9
\end{verbatim}

\item {\bf vecmul {\it v1 v2}} --
Returns the vector of a term-by-term multiply.
\index{vector command!vecadd}
\begin{verbatim}
Examples:
vmd > vecmul {1 2 3} {4 5 6} 
4 10 18
vmd > vecmul {0.1 0.2 0.4 0.8} {1 1 2 3}
0.1 0.2 0.8 2.4
\end{verbatim}

\item (C++) {\bf vecsub {\it v1 v2}} --
Returns the vector subtraction of the second term from the first
\index{vector command!vecsub}
\begin{verbatim}
Examples:
vmd > vecsub 6 3.2
2.8
vmd > vecsub {10 9.8 7} {0.1 0 -0.1}
9.9 9.8 7.1
vmd > vecsub {1 2 3 4 5} {6 7 8 9 10}
-5 -5 -5 -5 -5
\end{verbatim}

\item (C++) {\bf vecsum {\it v}} --
Returns the sum of the elements in v
\index{vector command!vecsum}
\begin{verbatim}
Examples:
vmd > vecsum { 1 2 3 }
6.0
\end{verbatim}

\item (C++) {\bf vecmean {\it v}} --
Returns the mean of the elements in v
\index{vector command!vecmean}
\begin{verbatim}
Examples:
vmd > vecmean { 1 2 3 }
2.0
\end{verbatim}

\item (C++) {\bf vecstddev {\it v}} --
Returns the standard deviation of the elements in v
\index{vector command!vecstddev}
\begin{verbatim}
Examples:
vmd > vecstddev { 1 2 3 4 5 6 7 8 9 10 }
2.87228131294
\end{verbatim}

\item (C++) {\bf vecscale {\it c v}} --
\item (C++) {\bf vecscale {\it v c}} --
Returns the vector of the scalar value c applied to each term of v
\index{vector command!vecscale}
\begin{verbatim}
Examples:
vmd > vecscale .2 {1 2 3}
0.2 0.4 0.6
vmd > vecscale {-5 4 -3 2} -2
10 -8 6 -4
vmd > vecscale -2 3
-6
\end{verbatim}

\item {\bf vecdot {\it v1 v2}} --
Returns the scalar dot product of the two vectors
\index{vector command!vecdot}
\begin{verbatim}
Examples:
vmd > vecdot {1 -2 3} {4 5 6}
12
vmd > vecdot {3 4} {3 4}
25
vmd > vecdot {1 2 3 4 5} {5 4 3 2 1}
35
vmd > vecdot 3 -2
-6
\end{verbatim}

\item {\bf veccross {\it v1 v2}} --
Returns the vector cross product of the two vectors.
\index{vector command!veccross}
\begin{verbatim}
Examples:
vmd > veccross {1 0 0} {0 1 0}
0 0 1
vmd > veccross {2 2 2} {-1 0 0}
0 -2 2
\end{verbatim}

\item {\bf veclength {\it v}} --
Returns the scalar length of v ($\|v\|$)
\index{vector command!veclength}
\begin{verbatim}
Examples:
vmd> veclength 5
5.0
vmd > veclength {5 12}
13.0
vmd > veclength {3 4 12}
13.0
vmd > veclength {1 -2 3 -4}
5.47723
\end{verbatim}

\item {\bf veclength2 {\it v}} --
Returns the square of the scalar length of v ($\|v\|^2$)
\index{vector command!veclength2}
\begin{verbatim}
Examples:
vmd > veclength2 5
25
vmd > veclength2 {5 12}
169
vmd > veclength2 {3 4 12}
169
vmd > veclength2 {1 -2 3 -4}
30
\end{verbatim}

\item {\bf vecnorm {\it v}} --
Returns the vector of length 1 directed along v
\index{vector command!vecnorm}
\begin{verbatim}
Examples:
vmd > vecnorm -10
-1.0
vmd > vecnorm {1 1 }
0.707109 0.707109
vmd > vecnorm {2 -3 1}
0.534522 -0.801783 0.267261
vmd > vecnorm {2 2 -2 2 -2 -2}
0.408248 0.408248 -0.408248 0.408248 -0.408248 -0.408248
\end{verbatim}

\item {\bf vecdist {\it v1 v2}} --
Returns the distance between the two vectors ($\|v_2 - v_1 \|$)
\index{vector command!vecdist}
\begin{verbatim}
Examples:
vmd > vecdist -1.5 5.5
7.0
vmd > vecdist {0 0 0} {3 4 0}
5.0
vmd > vecdist {0 1 2 3 4 5 6} {-6 -5 -4 -3 -2 -1 0}
15.8745
\end{verbatim}



\item {\bf vecinvert {\it v}} --
Returns the additive inverse of v ($-$v).
\index{vector command!vecinvert}
\begin{verbatim}
Examples:
vmd > vecinvert -11.1
11.1
vmd > vecinvert {3 -4 5}
-3 4 -5
vmd > vecinvert {0 -1 2 -3}
0 1 -2 3
\end{verbatim}

\end{itemize}

\section{Matrix routines}
\index{matrix routines!list of}

  Because matrices are rather large when expressed in text form, the
following definitions are used for the examples.


\begin{itemize}
\item {\bf transidentity} --
Returns the identity matrix.
\index{matrix routines!identity}
\begin{verbatim}
Example:
vmd > transidentity
{1.0 0.0 0.0 0.0} {0.0 1.0 0.0 0.0} {0.0 0.0 1.0 0.0} {0.0 0.0 0.0 1.0}
\end{verbatim}

\item {\bf transtranspose {\it m}} --
Returns the matrix transpose of the given matrix
\index{matrix routines!transpose}
\begin{verbatim}
Example:
vmd > transtranspose {{0 1 2 3 4} {5 6 7 8} {9 10 11 12} {13 14 15 16}}
{0 5 9 13} {1 6 10 14} {2 7 11 15} {3 8 12 16}
\end{verbatim}

\item (C++) {\bf transmult {\it m1 m2} [{\it m3 ... mn}]} --
Returns the matrix multiplication of the given matrices
\index{matrix routines!multiplication}
\begin{verbatim}
Examples:
vmd > set mat1 {{1 2 3 4} {-2 3 -4 5} {3 -4 5 -6} {4 5 -6 -7}}
vmd > set mat2 {{1 0 0 0} {0 0.7071 -0.7071 0} {0 0.7071 0.7071 0} {0 0 0 1}}
vmd > set mat3 {{0.866025 0 0 0} {0 1 0 0} {-0.5 0 0.866025 0} {0 0 0 1}}
vmd > transmult $mat1 [transidentity]
{1.0 2.0 3.0 4.0} {-2.0 3.0 -4.0 5.0} {3.0 -4.0 5.0 -6.0} 
   {4.0 5.0 -6.0 -7.0}
vmd > transmult $mat1 $mat2 $mat3
{0.512475 3.5355 0.612366 4.0} {0.7428 -0.7071 -4.28656 5.0}
   {-0.58387 0.7071 5.5113 -6.0} {7.35315 -0.7071 -6.73603 -7.0}
\end{verbatim}

\item {\bf transaxis $<$x\verb!|!y\verb!|!z$>$ 
{\it amount} [deg\verb!|!rad\verb!|!pi]} --
\index{matrix routines!rotation}
Returns the transformation matrix needed to rotate around the
specified axis by a given amount.  By default, the amount is specified
in degrees, though it can also be given in radians or factors of pi.
\begin{verbatim}
Examples:
vmd > transaxis x 90
{1.0 0.0 0.0 0.0} {0.0 -3.67321e-06 -1.0 0.0} {0.0 1.0 -3.67321e-06 0.0}
         {0.0 0.0 0.0 1.0}
vmd > transaxis y 0.25 pi
{0.707107 0.0 0.707107 0.0} {0.0 1.0  0.0 0.0} 
         {-0.707107 0.0 0.707107 0.0} {0.0 0.0  0.0 1.0}
vmd > transaxis z 3.1415927 rad
{-1.0 -2.65359e-06 0.0 0.0} {2.65359e-06 -1.0 0.0 0.0} {0.0 0.0 1.0 0.0}
         {0.0 0.0 0.0 1.0}
\end{verbatim}

\item {\bf transvec {\it v}} --
\index{matrix routines!alignment}
Returns the transformation matrix needed to bring the x axis along the
v vector.  This matrix is not unique, since a final rotation is
allowed around the vector.  The matrix is made from a rotation around
y, then one about z.
\begin{verbatim}
Examples:
vmd > transvec {0 1 0}
{-3.67321e-06 -1.0 0.0 0.0} {1.0 -3.67321e-06 0.0 0.0} {0.0 0.0 1.0 0.0}
         {0.0 0.0 0.0 1.0}
vmd > vectrans [transvec {0 0 2}] {1 0 0}
0.0 0.0 1.0
\end{verbatim}

\item {\bf transvecinv {\it v}} --
\index{matrix routines!inverse alignment}
Returns the transformation needed to bring the vector v to the x axis.
This produces the inverse matrix to transvec, and is composed of a
rotation about z then one about y.
\begin{verbatim}
Examples:
vmd > transvecinv {0 -1 0}
{-3.67321e-06 -1.0 0.0 0.0} {1.0 -3.67321e-06 0.0 0.0} {0.0 0.0 1.0 0.0}
    {0.0 0.0 0.0 1.0}
vmd > vectrans [transvecinv {-3 4 -12}] {-3 4 -12}
13.0 -1.8e-05 5.8e-05
vmd > transmult [transvec {6 -5 7}] [transvecinv {6 -5 7}]
{0.999999 2.29254e-07 -6.262e-09 0.0} {2.29254e-07 0.999999
    -4.52228e-07 0.0} {-6.262e-09 -4.52228e-07 1.0 0.0} {0.0 0.0 0.0 1.0}
\end{verbatim}

\item (C++) {\bf transoffset {\it v}} --
\index{matrix routines!translation}
Returns the transformation matrix needed to translate by the given offset
\begin{verbatim}
Examples:
vmd > transoffset {1 0 0}
{1.0 0.0 0.0 1} {0.0 1.0 0.0 0} {0.0 0.0 1.0 0} {0.0 0.0 0.0 1.0}
vmd > transoffset {-6 5 -4.3}
{1.0 0.0 0.0 -6} {0.0 1.0 0.0 5} {0.0 0.0 1.0 -4.3} {0.0 0.0 0.0 1.0}
\end{verbatim}

\item {\bf transabout {\it v amount} [deg\verb!|!rad\verb!|!pi]} --
Generates the transformation matrix needed to rotate by the given amount
counter-clockwise around axis which goes through the origin and along the
given vector.  As with transvec, the units of the amount of rotation
can be degrees, radians, or multiples of pi.
\index{example scripts!Tcl!calculation!matrix transformations}
\begin{verbatim}
Examples:
# this is a rotation about x by 180 degrees
vmd > transabout {1 0 0} 180
{1.0 0.0 0.0 0.0} {0.0 -1.0 -2.65359e-06 0.0} {0.0 2.65359e-06
        -1.0 0.0} {0.0 0.0 0.0 1.0}
# a rotation about z by 90 degrees
# (compare this to "transaxis z 90"
vmd > transabout {0 0 1} 1.5709 rad
{0.999624 -0.027414 0.0 0.0} {0.027414 0.999624 0.0 0.0} 
        {0.0 0.0 1.0 0.0} {0.0 0.0 0.0 1.0}
vmd > transabout {1 1 1} 1 pi
{-0.333335 0.666665 0.666669 0.0} {0.666668 -0.333334 0.666666
        0.0} {0.666666 0.66667 -0.333332 0.0} {0.0 0.0 0.0 1.0}
\end{verbatim}


\item {\bf trans [center \{{\it x y z}\}] 
  [origin \{{\it x y z}\}] [offset \{{\it x y z}\}]
  [axis x {\it amount} [rad\verb!|!deg\verb!|!pi]]
  [axis y {\it amount} [rad\verb!|!deg\verb!|!pi]]
  [axis z {\it amount} [rad\verb!|!deg\verb!|!pi]]
  [x {\it amount} [rad\verb!|!deg\verb!|!pi]]
  [y {\it amount} [rad\verb!|!deg\verb!|!pi]]
  [z {\it amount} [rad\verb!|!deg\verb!|!pi]]
  [axis \{{\it x y z}\}  {\it amount} [rad\verb!|!deg\verb!|!pi]]
  [bond \{{\it x1 y1 z1}\} \{{\it x2 y2 z2}\} {\it amount} 
      [rad\verb!|!deg\verb!|!pi]]
  [angle \{{\it x1 y1 z1}\} \{{\it x2 y2 z2}\} \{{\it x3 y3 z3}\} 
      {\it amount} [rad\verb!|!deg\verb!|!pi]]} --
\index{matrix routines!trans command}

This command can do almost everything the other ones can do, and then
some.  It is designed to be the main function used for generating
transformation matrices.
\end{itemize}

Using it correctly calls for understanding how it works internally.
There are three matrices: centering, rotation, and offset.  The
centering matrix determines where the center of rotation is located.
By default, this is the origin, but it can be changed to pivot about
any point.  The rotation matrix defines the rotation about that
centering point, and the offset matrix defines the final translation
after the rotation.  

  For example, to rotate around a given point, the transformations
would be 1) the centering matrix to bring that point to the origin, 2)
the rotation about the center, and 3) the final offset to return the
origin back to its original location.

  The different options for the {\tt trans} command modify the
matrices in various ways.

%\renewcommand{\labelitemi}{\labelitemii}
\newenvironment{MYitemize}{%
  \renewcommand{\labelitemi}{\bfseries --}%
      \begin{itemize}}{\end{itemize}}
\begin{MYitemize}
%\begin{itemize}
\item {\bf center \{{\it x y z}\}} --
	Sets the centering matrix so that point {x y z} is brought to the origin
\item {\bf offset \{{\it x y z}\}} --
	Sets the offset matrix so that the origin is brought to {x y z}
\item {\bf origin \{{\it x y z}\}} --
	Sets both the centering and offset matrices to {x y z}

\item {\bf axis x {\it amount} [rad\verb!|!deg\verb!|!pi]}
--
	Adds a rotation about the x axis by the given amount to the
rotation matrix

\item {\bf axis y {\it amount} [rad\verb!|!deg\verb!|!pi]}
--
	Adds a rotation about the y axis by the given amount to the
rotation matrix

\item {\bf axis z {\it amount} [rad\verb!|!deg\verb!|!pi]}
-- 
	Adds a rotation about the z axis by the given amount to the
rotation matrix

\item {\bf axis \{{\it x y z}\} {\it amount} [rad\verb!|!deg\verb!|!pi]} --
	Adds a rotation of the given amount about the given vector to
the rotation matrix

\item {\bf bond \{{\it x1 y1 z1}\} \{{\it x2 y2 z2}\} {\it amount} 
      [rad\verb!|!deg\verb!|!pi]} --
	Sets the center and offset transformations to the first point,
and defines a rotation about the bond axis by the given amount.

\item {\bf angle \{{\it x1 y1 z1}\} \{{\it x2 y2 z2}\} \{{\it x3 y3 z3}\} 
    {\it amount} [rad\verb!|!deg\verb!|!pi]} --
	Sets the center and offset transformation to the second point,
and defines a rotation about the axis perpendicular to the plane made
by the three points (the vector is computed from the cross product of
the vector connecting the first two points with that connected the
last two).
%\end{itemize}
\end{MYitemize}

%Should have some examples here ???.

\section{Multiplying vectors and matrices}

There are two commands to multiply a matrix and a vector, {\tt
vectrans} and {\tt coordtrans}.  They assume the vector is in column
form and premultiply the matrix to the vector.  If the vector contains
four numbers, the two commands are identical.  If the vector has three
elements, a fourth is added; a 0 for {\tt vectrans} and a 1 for {\tt
coordtrans}.  The difference is that vectors are not affected by
translations during transformations, while coordinates are.
\begin{itemize}
\item (C++) {\bf vectrans {\it m v}} --
\index{vector command!vectrans}
Multiple the matrix m with the vector v (length 4); returns a vector
\item {\bf coordtrans {\it m v}} --
\index{vector command!coordtrans}
Multiple the matrix m with the coordinate v (length 3); returns a vector
\end{itemize}
\begin{verbatim}
Examples:
vmd > vectrans [transaxis z 90] {1 0 0}
-3.67321e-06 1.0 0.0
vmd > vectrans [transvecinv {-3 4 -12}] {-3 4 -12}
13.0 -1.8e-05 5.8e-05
\end{verbatim}


\section{Misc. functions and values}

\index{variables!M\_PI}
  Several other terms are added to the vectors package.  The first is
the variable M\_PI, which contains the value of pi.
\begin{verbatim}
Examples:
vmd > set M_PI
3.14159265358979323846
vmd > expr 90 * ($M_PI / 180)
1.5708
\end{verbatim}

  The functions {\tt trans\_from\_rot}, {\tt trans\_to\_rot}, {\tt
trans\_from\_offset}, and {\tt trans\_to\_offset} are used to get or set a
transformation matrix from either a 3x3 rotation matrix or offset
vector.  As currently designed, these assume there is no scaling in
the matrix.  The {\tt trans\_from\_offset} is identical to {\tt
transoffset} and is present for completeness.


  The last is {\tt find\_rotation\_value varname}, which takes a
variable name and extracts from the beginning of it those terms which
describe an amount of rotation.  The rest of the data in the variable
remains, and the amount of rotation, in radians, is returned.  This is
used by those functions which need a rotation.  The valid values are:
a number, followed by one of {\tt rad}, {\tt radian}, or {\tt radians}
for a value in radians, the word {\tt pi} to give the rotation in
factors of pi, or one of {\tt deg}, {\tt degree}, or {\tt degrees} for
a value in degrees.  If no units are given, the value is assumed to be
in degrees.

Examples:
\begin{verbatim}
vmd > set a "180 deg north"
180 deg north
vmd > find_rotation_value a
3.14159
vmd > set a
north
vmd > set a "1 pi to eat"
1 pi to eat
vmd > find_rotation_value a
3.14159
vmd > set a
to eat
vmd > set a 45 
45
vmd > find_rotation_value a
0.785398
vmd > expr $M_PI * 3.0 / 2.0
4.71239
vmd > set a "4.71238 radians"
4.71238 radians
vmd > find_rotation_value a
4.71238
\end{verbatim}
