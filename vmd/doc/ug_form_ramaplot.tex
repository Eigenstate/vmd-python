
\subsection{RamaPlot}
\label{ug:ui:window:ramaplot}
\index{window!RamaPlot}
\index{Ramachandran plot}

\begin{rawhtml}
<CENTER>
\end{rawhtml}
\myfigure{ug_ramaplot}{The RamaPlot Window}{fig:ug:ramaplot}
\begin{rawhtml}
</CENTER>
\end{rawhtml}

The RamaPlot window displays a Ramachandran plot for a selected molecule.
If you animate your molecule over a range of \timesteps, RamaPlot will update
the Ramachandran plot automatically.  You can select a range of residues to
be displayed in the plot.  Clicking on a point in the Ramachandran
plot will show the trajectory of the selected residue in Ramachandran space
over all \timesteps.  Fields on the right of the window show the computed
value of phi and psi for the most recently selected residue.  Finally, you can 
create a PostScript image of the current Ramachandran plot.  RamaPlot
functionality is summarized in Fig.~\ref{fig:ug:ramaplot}.

\subsubsection{Using RamaPlot}

Start RamaPlot by typing ``{\tt ramaplot}" in the VMD text console, or by
selecting the {\sf ramaplot} menu item in the {\sf Extensions} menu.  The main
window contains a Ramachandran graph, with phi and psi running along the
horizontal and vertical axis, respectively, from -180 to 180 degrees.  The most
allowed region of Ramachandran space is colored blue; partially allowed regions
are colored green.

After loading a molecule, using the pulldown menu in the upper right part of
the window to choose a molecule.  Protein residues in the current molecule
are mapped to the Ramachandran diagram with yellow squares.  Clicking on a 
square causes the square to turn red, displays residue information in the
fields on the right side of the window, and, if trajectory data is present,
draws the location of the selected residue in Ramachandran space for all 
frames in the trajectory as empty black squares.  Clicking one of the empty
squares causes VMD to redraw the graphics display window with coordinates
from the timestep corresponding to that square.  Clicking a second time
on a red highlighted residue switches off the trajectory information
in the RamaPlot window.

When a protein contains many residues, it may be inconvenient to display
all residues at once.  Enter an atom selection in the Selection input to
choose which residues to display.  Note that the selection must contain the
alpha carbons (name CA) of the residues you want to show.  Note also that,
just like the Graphics window, the selection will not be recomputed 
if you change the animation frame.  

To print the contents of the white Ramachandran plot, select 
``Print to file..." from the RamaPlot File pulldown menu.  Enter a filename
to save the contents of the window.

