%%%%%%%%%%%%%%%%%%%%%%%%%%%%%%%%%%%%%%%%%%%%%%%%%%%%%%%%%%%%%%%%%%%%%%%%%%%%
% RCS INFORMATION:
%
%       $RCSfile: ug_joystick_display.tex,v $
%       $Author: justin $        $Locker:  $                $State: Exp $
%       $Revision: 1.2 $      $Date: 2001/12/19 18:23:16 $
%
%%%%%%%%%%%%%%%%%%%%%%%%%%%%%%%%%%%%%%%%%%%%%%%%%%%%%%%%%%%%%%%%%%%%%%%%%%%%
% DESCRIPTION:
%  The controls available from the OpenGL Display window
%
%%%%%%%%%%%%%%%%%%%%%%%%%%%%%%%%%%%%%%%%%%%%%%%%%%%%%%%%%%%%%%%%%%%%%%%%%%%%

\section{Using the Joystick in the Graphics Window}
\label{ug:joystick}
\index{joystick!using}

The Windows version of \VMD\ provides support for the 
Windows joystick driver, and will enumerate all available joystick
devices at startup time.  The joystick interface employed in VMD 
is quite simple, allowing the use of three control axes to translate,
rotate, and scale the molecule.  The joystick interface assumes a 
device with at least two buttons.  The first joystick button resets
the view in the display window, and the second button cycles through 
each of the available joystick modes.  When \VMD\ first attaches to
each of the joysticks, they are initially disabled so that miscalibrated
joysticks do not adversely affect the \VMD\ session.  Each joystick is
initially enabled by pressing its second button to switch modes.  All
joysticks are independently controlled such that multiple joysticks can
control different control axes, and multiple users could interact 
with the program with separate controls.

\newpage

