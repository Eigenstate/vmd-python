%%%%%%%%%%%%%%%%%%%%%%%%%%%%%%%%%%%%%%%%%%%%%%%%%%%%%%%%%%%%%%%%%%%%%%%%%%%%
% RCS INFORMATION:
%
%       $RCSfile: ug_mouse_display.tex,v $
%       $Author: johns $        $Locker:  $                $State: Exp $
%       $Revision: 1.30 $      $Date: 2012/01/10 19:30:06 $
%
%%%%%%%%%%%%%%%%%%%%%%%%%%%%%%%%%%%%%%%%%%%%%%%%%%%%%%%%%%%%%%%%%%%%%%%%%%%%
% DESCRIPTION:
%  The controls available from the OpenGL Display window
%
%%%%%%%%%%%%%%%%%%%%%%%%%%%%%%%%%%%%%%%%%%%%%%%%%%%%%%%%%%%%%%%%%%%%%%%%%%%%

\section{Using the Mouse in the Graphics Window}
\label{ug:ui:disp}
\index{mouse!using}

The graphics window is labeled {\sf VMD OpenGL Display} and contains a
view of the molecules and other objects which make up the scene.  
When the mouse is in the graphics display window, it may be used to perform 
the following actions such as:
\begin{itemize}
  \item Rotate, translate, or scale the displayed molecules
  \item Select, or `pick' atoms or other objects in order to 
        move them, or label them 
  \item Translate and rotate a set of atoms
  \item Apply a force (acceleration) to a set of atoms
  \item Move the lights
\end{itemize}
User-defined keyboard accelerators, or {\em hot keys}, are also available
when the mouse is in the graphics display window.  
These keys are bound to \VMD\ text commands, which are executed when the key
is pressed.  \VMD\ has many built-in default hot key commands (see 
Tables~\ref{table:ug:mouse:control}, \ref{table:ug:rotations},
\ref{table:ug:menushortcuts} and \ref{table:ug:animationhotkeys}). 
Users can add new hot keys, overriding default settings if desired.  

%%%%%%%%%%%%%%%%%%%%%%%%%%%%%%%%%%%%%%%%%%

\subsection{Mouse Modes }
\label{ug:ui:disp:modes}
\index{mouse!modes}

The mouse is in one of several {\em modes} at any time; 
the current mouse mode determines the effect of pressing
and releasing mouse buttons or the mouse wheel while the mouse
is in the graphics window.
Each mouse mode, except the lights mode (see below), 
sets the mouse cursor to a characteristic shape.  
The mouse mode is selected via the Mouse menu.

The available mouse modes are as follows:
\begin{itemize}

\item {\bf Rotate Mode}
\hspace{0.2in} (hot key 'r')
\\
\label{ug:ui:disp:rotate}
\index{rotation!using mouse}
When the mouse is in rotate mode, holding the left mouse button down 
and moving the mouse rotates the molecules about axes parallel to the
screen, in a `virtual trackball' behavior.  To get a rotation around
the axes coming out of the screen (the `z' axis), hold the middle
button down and move the mouse left or right.

You can leave \index{rotation!continuous}molecules rotating
without continuously moving the mouse.  Start the molecule moving with
the mouse, as above, then release the mouse button before you stop
moving the mouse.  With some practice it becomes easy to impart a
slight spin on the molecule, or whirl it about madly.  To stop the
rotation, either press and hold the left mouse button down until the
molecule stops moving, or select `Stop Rotation' in the Mouse menu.
Also, pressing the rotation hot key {\tt r} or any of the other mouse 
mode hot keys causes \index{rotation!stop}rotation to stop.

\item {\bf Translate Mode} 
\hspace{0.2in} (hot key 't')
\\
\label{ug:ui:disp:trans}
\index{translation!using mouse}
When the mouse is in translate mode, holding the left button
down allows you to move the molecules parallel to the screen plane 
(left, right, up, and down).  To move
the molecule towards or away from you, hold the middle button down and
move the mouse right or left, respectively.

\item {\bf Scale Mode} 
\hspace{0.2in} (hot key 's')
\\
\label{ug:ui:disp:scale}
\index{scaling!using mouse}
Pressing either the left or middle button down and moving to the right 
enlarges the molecules, and moving the mouse left shrinks them.  
The difference is that the middle button scales faster than the left button.
Scaling can also be accomplished with the mouse wheel (irrespective
of the current mode setting) on computers equipped with an appropriate
mouse.

\item {\bf Move Light} 
\\
\label{ug:ui:disp:lights}
\index{light!controlling with mouse}
\index{image!lighting controls}
\VMD\ provides four directional lights to illuminate the molecular scene.  
The lights provide diffuse lighting and specular highlights 
and help the user perceive surface shape in rendered objects.
You can use the mouse to rotate each of the light source directions
to a new position.
If the light isn't on, moving it will not affect the displayed image.
To turn a light on or off, use the {\sf Lights} item within the 
{\sf Mouse} menu.

\item {\bf Add/Remove Bonds} \\
\label{ug:ui:disp:addbonds}
\index{bonds!add or remove}
\index{mouse!add or remove bonds}
When the mouse is in add/remove bonds mode, clicking on atoms in a 
molecule will add a bond between those atoms if one is not already present,
or remove the bond between those atoms if there is already a bond.  The
two atoms must belong to the same molecule.  

\end{itemize}


%%%%%%%%%%%%%%%%%%%%%%%%%%%%%%%%%%%%%%%%%%

\subsection{Pick Modes}
\label{ug:ui:disp:pick}
\index{picking!modes}
\index{picking!atoms}
\index{picking!bonds}
\index{picking!angles}
\index{picking!dihedrals}
\index{labels!picking with mouse}
Mouse picking can be used to turn on or off various types
of labels, to query for information about an object, or to move items
around on the screen.  You can label an atom (and display the atom
name), or you can label geometric values such as the
distance between two atoms (a {\em bond} label), an angle between three
atoms (an {\em angle} label), or the dihedral angle formed by four
atoms (a {\em dihedral} label).  This is done by setting the mouse
into the proper picking mode and then selecting the relevant atoms
with the mouse.  Picking modes are selected from the {\sf Mouse} menu.

The available pick mode actions are:
\begin{itemize}

\item {\bf Center}\index{picking!center}
\hspace{0.2in} (hot key 'c')
\\ 
This mode is used to change the point about which a molecule
rotates when the molecule is rotated.  To cause a molecule to rotate
about a specific atom, select this mode and then click on that atom.
The rotation point may be restored to its default position (the center
of volume of the molecule) by executing the `Reset View' option from
the Mouse menu.  

\item {\bf Query}\index{picking!query} 
\hspace{0.2in} (hot key '0')
\\
Clicking on an item will print out the name of the item (e.g. the 
atom name) to the text console window.

\item {\bf Label $\rightarrow$ Atom}\index{picking!atoms} 
\hspace{0.2in} (hot key '1')
\\
Clicking on an atom will toggle on/off a label for the atom.  

\item {\bf Label $\rightarrow$ Bond}\index{picking!bonds} 
\hspace{0.2in} (hot key '2')
\\
Clicking on two atoms in a row will toggle on/off a bond distance
label between the two atoms (a dotted line with the distance printed
at the midpoint).  

\item {\bf Label $\rightarrow$ Angle}\index{picking!angles} 
\hspace{0.2in} (hot key '3')
\\
Clicking on three atoms in a row will toggle on/off a label showing
the angle formed by the three atoms. 

\item {\bf Label $\rightarrow$ Dihedral}\index{picking!dihedrals} 
\hspace{0.2in} (hot key '4')
\\
Clicking on four atoms in a row toggles on/off a label showing the
dihedral angle formed by the four atoms.  

\item {\bf Move $\rightarrow$ Atom}\index{picking!move atom} 
\hspace{0.2in} (hot key '5')
\\
In this mode, the position of an atom can be changed by clicking on 
the desired atom, and dragging with the mouse while the button is still
pressed.  This will change the atom coordinates.  

\item {\bf Move $\rightarrow$ Residue}\index{picking!move residue} 
\hspace{0.2in} (hot key '6')
\\
This mode may be used to move all the atoms in a selected residue at
the same time.  Select an atom in a residue, and move it to a new
position while keeping the mouse button pressed.  All the atoms in the
same residue as the selected one will be moved the same amount.  
Holding down the <shift> key and the left mouse button while moving the mouse
will rotate the atoms in the residue about the selected atom.  If the middle
mouse button is held down instead, the atoms in the residue will rotate about
a line drawn through the picked atom and parallel to a line coming directly
out of the screen.  This behavior is similar to the usual Rotate mode, except
that coordinates of atoms are changed. 

\item {\bf Move $\rightarrow$ Fragment}\index{picking!move fragment} 
\hspace{0.2in} (hot key '7')
\\
A {\em fragment} is a set of atoms all connected by a series of
covalent bonds.  This mode acts just like MoveResidue, except that the
atoms which are moved are all in the selected fragment rather than in
the selected residue.  This will change the atom coordinates.
Holding down the <shift> key and the left mouse button while moving the mouse
will rotate the atoms in the fragment about the selected atom.  If the middle
mouse button is held down instead, the atoms in the fragment will rotate about
a line drawn through the picked atom and parallel to a line coming directly
out of the screen.  This behavior is similar to the usual Rotate mode, except
that coordinates of atoms are changed. 

\item {\bf Move $\rightarrow$ Molecule}\index{picking!move molecule} 
\hspace{0.2in} (hot key '8')
\\
This mode may be used to move all the atoms in a selected molecule at
the same time.  Select an atom in a molecule, and move it to a new
position while keeping the mouse button pressed.  All the atoms in the
same molecule as the selected one will be moved the same amount.  
Holding down the <shift> key and the left mouse button while moving the mouse
will rotate the atoms in the molecule about the selected atom.  If the middle
mouse button is held down instead, the atoms in the molecule will rotate about
a line drawn through the picked atom and parallel to a line coming directly
out of the screen.  This behavior is similar to the usual Rotate mode, except
that coordinates of atoms are changed. 

\item {\bf Move $\rightarrow$ Rep}\index{picking!move highlighted rep} 
\hspace{0.2in} (hot key '9')
\\
This mode may be used to move all the atoms in a selected representation at
the same time.  You select a representation by clicking on one of the reps
in the browser window of the Graphics window.  In order to move the atoms in 
this rep, the atom you pick with the mouse must be selected by that rep.

When you have clicked on an atom in the rep, move the mouse to a new position
while  keeping the mouse button pressed.  All the atoms selected by the 
highlighted rep will be moved the same amount.  
Holding down the <shift> key and the left mouse button while moving the mouse
will rotate the atoms in the rep about the selected atom.  If the middle
mouse button is held down instead, the atoms in the rep will rotate about
a line drawn through the picked atom and parallel to a line coming directly
out of the screen.  This behavior is similar to the usual Rotate mode, except
that coordinates of atoms are changed. 

\end{itemize}

%%%%%%%%%%%%%%%%%%%%%%%%%%%%%%%%%%%%%%%%%%%%%%%%%%%%%%%%%%%%%%%%%%%%%%
%%%%%%%%%%%%%%%%%%%%%%%%%%%%%%%%%%%%%%%%%%%%%%%%%%%%%%%%%%%%%%%%%%%%%%
%%%%%%%%%%%%%%%%%%%%%%%%%%%%%%%%%%%%%%%%%%%%%%%%%%%%%%%%%%%%%%%%%%%%%%

\subsection{Hot Keys}
\label{ug:ui:hotkeys}
\index{hot keys}

When the mouse is in the graphics window, many commands are
accessible via programmable hot keys.  Hot keys allow you to do things
like change mouse modes or advance the animation by a frame by simply
pressing a key.  
There are a number of predefined hot keys,
as listed in tables \ref{table:ug:mouse:control},
\ref{table:ug:rotations}, \ref{table:ug:menushortcuts},
and \ref{table:ug:animationhotkeys}.
They can be printed out with the command {\tt user print keys}.
The commands listed are the text commands which are executed
when the hot key is pressed; these text commands are explained in
section \ref{ug:section:text}.  

\index{hot keys!customizing}
To add or modify a hot key, use the command
{\tt user add key {\it key} {\it command}}.
The {\em key} parameter must be a single character.  
If {\em command} contains more than one word, it must be enclosed in 
braces so that the subsequent command words are not ignored.  
When that key is pressed while the mouse cursor is in the 
graphics display window, the associated command will be executed.
\index{VMD!customizing}
Once you have a set of commands which are particularly
useful and familiar for you, you will want these
hot key commands automatically available every time you run \VMD.  
This can be done by placing the commands to add these items
in your {\tt .vmdrc} file, which is a file containing \VMD\ text
commands that is executed every time
\VMD\ starts up. 
\index{{\tt .vmdrc}} 
\index{startup files}
\index{files!startup}
The basic method for setting up this file is described
in section \ref{ug:section:vmdrc}.  
Once you have such a file, put the {\tt user add} commands in it.


\begin{table}[htb]
  \hspace{1in}
  \begin{tabular}{|c|l|l|} \hline
    Hot Key & \multicolumn{1}{|c}{Command} &
		\multicolumn{1}{|c|}{Purpose}\\ \hline\hline
	{\tt r, R}	& {\tt mouse mode 0 0} 	& enter rotate mode; stop rotation \\
	{\tt t, T}	& {\tt mouse mode 1 0} 	& enter translate mode \\
	{\tt s, S}	& {\tt mouse mode 2 0} 	& enter scaling mode \\
	{\tt 0}		& {\tt mouse mode 4 0}	& query item \\
	{\tt c}		& {\tt mouse mode 4 1}	& assign rotation center \\
	{\tt 1}		& {\tt mouse mode 4 2} 	& pick atom \\
	{\tt 2} 	& {\tt mouse mode 4 3}	& pick bond (2 atoms) \\
	{\tt 3}		& {\tt mouse mode 4 4}	& pick angle (3 atoms) \\
	{\tt 4}		& {\tt mouse mode 4 5} 	& pick dihedral (4 atoms) \\
	{\tt 5}		& {\tt mouse mode 4 6}	& move atom \\
	{\tt 6}		& {\tt mouse mode 4 7} 	& move residue \\
	{\tt 7}		& {\tt mouse mode 4 8} 	& move fragment \\
	{\tt 8}		& {\tt mouse mode 4 9}	& move molecule \\
	{\tt 9}		& {\tt mouse mode 4 13}	& move highlighted rep \\
	{\tt \%}	& {\tt mouse mode 4 10} & apply force on atom \\
	{\tt ${}^\wedge$}	& {\tt mouse mode 4 11}	& apply force on residue \\
	{\tt \&}	& {\tt mouse mode 4 12} & apply force on fragment \\ \hline
  \end{tabular}
  \caption{Mouse control hot keys.}
  \label{table:ug:mouse:control}
\index{hot keys!mouse control}
\index{mouse!modes}
\index{picking!hot keys}
\end{table}


\begin{table}[htb]
  \hspace{1in}
  \begin{tabular}{|c|l|l|} \hline
%    Hot Key & \multicolumn{1}{|c}{Command} &
%		\multicolumn{1}{|c|}{Purpose}\\ \hline\hline
    Hot Key & Command & Purpose\\ \hline\hline
	{\tt x}		& {\tt rock x by 1 -1} 	& spin about x axis \\
	{\tt X}		& {\tt rock x by 1 70} 	& rock about x axis \\
	{\tt y}		& {\tt rock y by 1 -1} 	& spin about y axis \\
	{\tt Y}		& {\tt rock y by 1 70} 	& rock about y axis \\
	{\tt z}		& {\tt rock z by 1 -1} 	& spin about z axis \\
	{\tt Z}		& {\tt rock z by 1 70} 	& rock about z axis \\
	{\tt j, Cntl-n}	& {\tt rotate x by 2}	& rotate $2^{\circ}$ about x \\
	{\tt k, Cntl-p}	& {\tt rotate x by -2}	& rotate $-2^{\circ}$ about x \\
	{\tt l, Cntl-f}	& {\tt rotate y by 2}	& rotate $2^{\circ}$ about y \\
	{\tt h, Cntl-b}	& {\tt rotate y by -2}	& rotate $-2^{\circ}$ about y \\
	{\tt g}		& {\tt rotate z by 2}	& rotate $2^{\circ}$ about z \\
	{\tt G}		& {\tt rotate z by -2}	& rotate $-2^{\circ}$ about z \\
	{\tt Cntl-a}	& {\tt scale by 1.1}	& enlarge 10 percent \\
	{\tt Cntl-z}	& {\tt scale by 0.9}	& shrink 10 percent \\ \hline
  \end{tabular}
  \caption{Rotation \& scaling hot keys.}
  \label{table:ug:rotations}
\index{hot keys!rotation and scaling}
\index{rotation!hot keys}
\end{table}


\begin{table}[htb]
  \hspace{1in}
  \begin{tabular}{|c|l|l|} \hline
%    Hot Key & \multicolumn{1}{|c}{Command} &
%		\multicolumn{1}{|c|}{Purpose}\\ \hline\hline
    Hot Key & Command &
		Purpose\\ \hline\hline
{\tt Alt-M}	& {\tt menu main off;menu main on}	& Show main menu  \\
{\tt Alt-f}	& {\tt menu files off;menu files on}	& Show files menu \\
{\tt Alt-g}	& {\tt menu graphics off;menu graphics on} & Show graphics menu \\
{\tt Alt-l}	& {\tt menu labels off;menu labels on}	& Show labels menu \\
{\tt Alt-r}	& {\tt menu render off;menu render on} 	& Show render menu \\
{\tt Alt-d}	& {\tt menu display off;menu display on}& Show display menu \\
{\tt Alt-c}	& {\tt menu color off;menu color on} 	& Show color menu \\
{\tt Cntl-r}	& {\tt display resetview}		& Reset display \\
{\tt Alt-q}	& {\tt quit confirm}			& Quit VMD with confirmation \\
{\tt Alt-Q}	& {\tt quit}				& Quit VMD \\
{\tt Alt-h}	& {\tt hyperref invert}			& Invert hyper text mode (NOT help) \\ \hline
  \end{tabular}
  \caption{Menu control hot keys.}
  \label{table:ug:menushortcuts}
\index{hot keys!menu control}
\index{window!hot keys}
\end{table}






\begin{table}[htb]
  \hspace{1in}
  \begin{tabular}{|c|l|l|} \hline
%    Hot Key & \multicolumn{1}{|c}{Command} &
%		\multicolumn{1}{|c|}{Purpose}\\ \hline\hline
    Hot Key & Command &
		Purpose\\ \hline\hline
	{\tt +,f,F}	& {\tt animate next}	& move to next frame\\
	{\tt -,b,B}	& {\tt animate prev}	& move to previous frame \\
	{\tt .,>}	& {\tt animate forward} & play animation forward \\
	{\tt ,}		& {\tt animate reverse} & play animation reverse \\
	{\tt <}		& {\tt animate reverse}	& play animation reverse \\
	{\tt /, ?}	& {\tt animate pause}	& stop animation \\ \hline
  \end{tabular}
  \caption{Animation hot keys.}
  \label{table:ug:animationhotkeys}
\index{hot keys!animation control}
\index{animation!hot keys}
\end{table}


\newpage

