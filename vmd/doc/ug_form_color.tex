%%%%%%%%%%%%%%%%%%%%%%%%%%%%%%%%%%%%%%%%%%%%%%%%%%%%%%%%%%%%%%%%%%%%%%%%%%%%
% RCS INFORMATION:
%
%       $RCSfile: ug_form_color.tex,v $
%       $Author: johns $        $Locker:  $                $State: Exp $
%       $Revision: 1.17 $      $Date: 2012/01/10 19:30:04 $
%
%%%%%%%%%%%%%%%%%%%%%%%%%%%%%%%%%%%%%%%%%%%%%%%%%%%%%%%%%%%%%%%%%%%%%%%%%%%%
% DESCRIPTION:
% the Color window
%
%%%%%%%%%%%%%%%%%%%%%%%%%%%%%%%%%%%%%%%%%%%%%%%%%%%%%%%%%%%%%%%%%%%%%%%%%%%%


\subsection{Color Window}
\label{ug:ui:window:color}
\index{window!color}
\index{color!window}

\begin{rawhtml}
<CENTER>
\end{rawhtml}
\myfigure{ug_color}{The Color window}{fig:ug:color}
\begin{rawhtml}
</CENTER>
\end{rawhtml}

\VMD\ maintains a database of the colors used for the molecules and the
other graphical objects in the display window.  The database consists
of several color \index{color!category}{\em categories}; each color
category contains a list of names, and each name is assigned a color.
The assignment of colors to names can be changed with this window.
There are 16 colors, as well as black (the \VMD\ color
\index{color!map}{\em map}), and this window can also be used to modify
the definitions of these 17 colors.  For more about colors, see the
chapter on
\hyperref{Coloring}{Coloring [\S }{]}{ug:topic:coloring}.

To see the names associated with a color category, click on the
category in the {\sf Category} browser located on the left side of the
window.  Click on the name to see the color to which it is mapped.  To
change the mapping, click on a new color in the browser to the right
of the {\sf Category} browser.  For instance, to change the
\index{color!background}background to white, pick `Display' in the
left browser and `Background' in the center one.  The right browser
will indicate the current color (which is initially {\tt black} for
the background).  Scroll through the right browser and select {\tt
white} to change the background.


\subsubsection{Changing the RGB Value of a Color}
\index{color!redefinition}

The {\sf Color Definitions} tab at the bottom of the Color menu lets
you change the RGB definition of the 17 palette colors.  Select a color
to edit using the browser at the bottom left corner of the menu, then
slide the three sliders to set the amount of each red, green and blue
component.  {\sf Default} restores the original color definition, and
{\sf Grayscale} toggles whether or not the three sliders will move 
together as a unit.  Color definitions are immediately updated in the
graphics window, so you can see the result of your editing right away.

\subsubsection{Color Scale}
\label{ug:ui:window:color:scale}
\index{color!scale!changing}

Several of the coloring methods in the graphics window (e.g., Beta, 
Index, Position) are used to color
a range of values, as opposed to a list of names.  The actual
coloring is determined by the \index{color!scale} \hyperref{color
scale}{color scale [\S }{]}{ug:topic:coloring:scale}.

The color scale used to assign these colors is set in the {\sf Color Scale}
tab of the color menu.  Choose one of the ten color scales from the
chooser, and adjust the {\sf Offset} and {\sf Midpoint} sliders until
the color scale shown at the bottom of the tab is as desired.  

