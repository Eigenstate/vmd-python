%%%%%%%%%%%%%%%%%%%%%%%%%%%%%%%%%%%%%%%%%%%%%%%%%%%%%%%%%%%%%%%%%%%%%%%%%%%
%cr                                                                       
%cr            (C) Copyright 1995 The Board of Trustees of the            
%cr                        University of Illinois                         
%cr                         All Rights Reserved                           
%cr                                                                       
%%%%%%%%%%%%%%%%%%%%%%%%%%%%%%%%%%%%%%%%%%%%%%%%%%%%%%%%%%%%%%%%%%%%%%%%%%%

%%%%%%%%%%%%%%%%%%%%%%%%%%%%%%%%%%%%%%%%%%%%%%%%%%%%%%%%%%%%%%%%%%%%%%%%%%%%
% RCS INFORMATION:
%
%       $RCSfile: ig.tex,v $
%       $Author: johns $        $Locker:  $                $State: Exp $
%       $Revision: 1.55 $      $Date: 2014/12/29 03:25:57 $
%
%%%%%%%%%%%%%%%%%%%%%%%%%%%%%%%%%%%%%%%%%%%%%%%%%%%%%%%%%%%%%%%%%%%%%%%%%%%%
% DESCRIPTION:
%
% INSTALLATION GUIDE : MAIN LATEX File
%   Defines name, description, and sections for the installation
% guide.
%%%%%%%%%%%%%%%%%%%%%%%%%%%%%%%%%%%%%%%%%%%%%%%%%%%%%%%%%%%%%%%%%%%%%%%%%%%%



% LaTeX 2.09
%\documentstyle[11pt,html,graphic]{article}

% LaTeX 2E
\documentclass[11pt]{article}
\usepackage{html}
\usepackage{graphics}

% define margins, etc
\topmargin      0.1in
\oddsidemargin  0in
\evensidemargin 0in
\textheight     8.80in
\textwidth      6.50in
\marginparsep   0.25cm
\headheight     0in
\headsep        0in
%\footskip      0.5in
%\footheight    0in%

%
% the document itself
%


% define macros
%%%%%%%%%%%%%%%%%%%%%%%%%%%%%%%%%%%%%%%%%%%%%%%%%%%%%%%%%%%%%%%%%%%%%%%%%%%
%cr                                                                       
%cr            (C) Copyright 1995 The Board of Trustees of the            
%cr                        University of Illinois                         
%cr                         All Rights Reserved                           
%cr                                                                       
%%%%%%%%%%%%%%%%%%%%%%%%%%%%%%%%%%%%%%%%%%%%%%%%%%%%%%%%%%%%%%%%%%%%%%%%%%%

%%%%%%%%%%%%%%%%%%%%%%%%%%%%%%%%%%%%%%%%%%%%%%%%%%%%%%%%%%%%%%%%%%%%%%%%%%%%
% RCS INFORMATION:
%
%       $RCSfile: vmd_macros.tex,v $
%       $Author: johns $        $Locker:  $                $State: Exp $
%       $Revision: 1.21 $      $Date: 2014/12/29 02:50:13 $
%
%%%%%%%%%%%%%%%%%%%%%%%%%%%%%%%%%%%%%%%%%%%%%%%%%%%%%%%%%%%%%%%%%%%%%%%%%%%%
% DESCRIPTION:
%
% useful macros for documentation.  This file is included by all
% main documents, i.e. programmers guide, users guide, etc.
%%%%%%%%%%%%%%%%%%%%%%%%%%%%%%%%%%%%%%%%%%%%%%%%%%%%%%%%%%%%%%%%%%%%%%%%%%%%

%
% generally useful macros
%
\newcommand{\REFAND} {\&}
\newcommand{\ETALNP}{\mbox{\it et al}}
\newcommand{\ETAL}{\mbox{\ETALNP{\it.}}}
\newcommand{\eqnref}[1] {\mbox{eq (\ref{#1})}}
%\newcommand{\mycite}[2] {\cite{#2}}
\newcommand{\mycite}[2] {}


%
% a "myfigure" macro for including graphics etc.
%

%
% Old pre-pdflatex version
%
%\newcommand{\myfigure}[3]{
%\begin{figure}[htb]
%  \begin{center}
%      \epsfig{file=pictures/#1}
%  \end{center}
%  \caption{#2}
%  \label{#3}
%  \htmlimage{scale=1.6}
%\end{figure}
%}

%
% New pdflatex-friendly version
%

% behavior: PS is scaled down a bit, HTML is 1 to 1
% this is the default macro for figures/screenshots
%% NOTE: Do not use \mynewfigure in the LaTeX sources, 
%% instead, rename it to replace the old \myfigure.
\newcommand{\myfigure}[3]{
  \begin{figure}[htb]
  \begin{center}
    \html{
      \htmlimage{scale=1.6}
    }
    \latex{
      \scalebox{0.550}{\includegraphics{pictures/#1}}
    }
    \end{center}
    \caption{#2}
    \label{#3}
  \end{figure}
}


% behavior: Both PS and HTML are rescaled to 4in in width
% use for exceptionally large figures that cause problems.
\newcommand{\myhugefigure}[4]{
  \begin{figure}[htb]
  \begin{center}
    \resizebox{4in}{!}{\includegraphics{pictures/#1}}
    \end{center}
    \caption[#2]{#3}
    \label{#4}
  \end{figure}
}


% OLD MACRO -- Keep in case of trouble...
% HTML is very ugly/barely readable
% All PDF figures have same width
%\newcommand{\myoldfigure}[3]{
%  \begin{figure}[htb]
%  \begin{center}
%    \resizebox{3in}{!}{\includegraphics{pictures/#1}}
%%   \scalebox{0.625}{\includegraphics{pictures/#1}}
%    \end{center}
%    \caption{#2}
%    \label{#3}
%  \end{figure}
%}



%
% include name and version number of program; this is generated
% by the 'make version' command in either the doc or src directory
% This will define the 'VMDNAME', 'VMDVER', and 'VMDDATE' macros,
% as well as the 'VMDAUTHORS' macro.
% Then the 'VMD' macro is used throughout the document to refer to
% the program name, while the other macros as defined in the file
% vmd_version.tex are used as given.
%

\newcommand{\VMDNAME} {vmd}
\newcommand{\VMDDATE} {August 27, 2015}
\newcommand{\VMDVER} {1.9.3a3}
\newcommand{\VMDAUTHORS} {R. Brunner, E. Caddigan, J. Cohen, A. Dalke, P. Grayson, J. Gullingsrud, D. Hardy, W. Humphrey, B. Isralewitz, S. Izrailev, A. Kohlmeyer, D. Norris, J. Saam, J. Stone, J. Ulrich, K. Vandivort}
\newcommand{\VMDVERSAUTHORS} {J. Stone, K. Vandivort}


%
% macros for style conventions when describing the program.
%

% name of class or object in program
\newcommand{\OBJ}[1] {\htmlref{{\bf #1}}{#1}}

% function arguments
\newcommand{\FA}[2] {{\rm{\bf#1}\ {\it#2}}}

% global function name
\newcommand{\FN}[3] {{\rm\bf#1}\ {\tt #2(}#3{\tt)}}

% class member function name
\newcommand{\FNO}[4] {{\rm\bf#2}\ \OBJ{#1::}{\tt#3(}#4{\tt)}}

% list item, for optional components, parameters, etc.
\newcommand{\TTLISTITEM}[1] {\item {\tt #1} \\}
\newcommand{\RMLISTITEM}[1] {\item {\rm #1} \\}
\newcommand{\BOLDLISTITEM}[1] {\item {\bf #1} \\}
\newcommand{\EMLISTITEM}[1] {\item {\em #1} \\}
\newcommand{\LISTITEM}[1] {\RMLISTITEM{#1}}

%
% other generally useful macros
%

% where to e-mail us
\newcommand{\vmdemail} {{\tt vmd@ks.uiuc.edu}}

% name of NAMD and MDCOMM programs, formatted nicely
\newcommand{\VMD}       {VMD}
\newcommand{\NAMD}      {NAMD}
\newcommand{\CESB}      {MDScope}
\newcommand{\MDScope}   {MDScope}
\newcommand{\BIOCORE}   {BioCoRE}
\newcommand{\MDTOOLS}   {MDTools}
\newcommand{\JMV}       {JMV}

% full name for CESB, i.e., what it stands for
\newcommand{\CESBNAME} {Molecular Dynamics computational environment}

% title of CESB paper
\newcommand{\CESBPAPER} {MDScope: A Visual Computing Environment for
Structural Biology}

% We aren't sure about "timesteps" vs. "frames" for now, so we define macros
% This is used to refer to "trajectory frames/timesteps" only!
\newcommand{\timestep} {frame }
\newcommand{\timesteps} {frames }
\newcommand{\Timesteps} {Frames }

%
% macros used for formatting object description pages
%

% text printed at top of object description page;
% this also has an argument mentioning in what optional component
% this object is used.
\newcommand{\OPTOBJDESCRIPTIONHEADER}[6] {
  \newpage
  \subsection{#1}
  \label{#1}
  \begin{tabular}{|ll|} \hline
    {\em Files:}			& {\tt #2} 		\\
    {\em Derived from:} 		& {#3} 			\\
    {\em Global instance (if any):} 	& {#4}			\\
    {\em Used in optional component:}	& {#5}			\\ \hline
  \end{tabular}
  \subsubsection*{Description}
  {#6}
}

% text printed at top of object description page, for a 'standard' object.
\newcommand{\OBJDESCRIPTIONHEADER}[5] {
  \newpage
  \subsection{#1}
  \label{#1}
  \begin{tabular}{|ll|} \hline
    {\em Files:}			& {\tt #2} 		\\
    {\em Derived from:} 		& {#3} 			\\
    {\em Global instance (if any):} 	& {#4}			\\
    {\em Used in optional component:}	& {Part of main \VMD\ code} \\ \hline
  \end{tabular}
  \subsubsection*{Description}
  {#5}
}

% after the header comes info on the constructor.
\newcommand{\OBJCONSTRUCTOR}[1] {
  \subsubsection*{Constructors}
  \begin{itemize}
    #1
  \end{itemize}
}

% then comes a list of any enumerations or list of names, if any
\newcommand{\OBJLISTS}[1] {
  \subsubsection*{Enumerations, lists or character name arrays}
  {#1}
}

% then comes a list of important internal data structures
\newcommand{\OBJDATA}[1] {
  \subsubsection*{Internal data structures}
  \begin{itemize}
    #1
  \end{itemize}
}

% then comes a list of the functions in this object (nonvirtual)
\newcommand{\OBJFUNCTIONS}[1] {
  \subsubsection*{Nonvirtual member functions}
  \begin{itemize}
    #1
  \end{itemize}
}

% then comes a list of the functions in this object (virtual)
\newcommand{\OBJVIRTUALFUNCTIONS}[1] {
  \subsubsection*{Virtual member functions}
  \begin{itemize}
    #1
  \end{itemize}
}

% and finally a description of how to use this object
\newcommand{\OBJUSAGE}[1] {
  \subsubsection*{Method of use}
  {#1}
}

% if desired, hints for what to change can go last
\newcommand{\OBJFUTURE}[1] {
  \subsubsection*{Suggestions for future changes/additions}
  {#1}
}



\newcommand{\DOCTITLE} {Installation Guide}

\newcommand{\DOCDESC} {
This document describes how to install one of the precompiled releases
of VMD and contains links to information on compilation of VMD from the
source code release.}

\begin{document}

\thispagestyle{empty}

\vspace*{0.3in}

\begin{center}
  \rule{6in}{0.04in}                            \\      \vspace{0.25in}
  {\Huge \VMD\ \DOCTITLE}                       \\      \vspace{0.25in}
  \rule{6in}{0.04in}                            \\      \vspace{0.25in}
  {\Large Version \VMDVER}                      \\      \vspace{0.20in}
  \VMDDATE                                      \\      \vspace{0.20in}
  \rule{6in}{0.04in}                            \\      \vspace{0.25in}

%  \htmladdimg{./vmd.gif}

  {\Large NIH Biomedical Technology Research Center for Macromolecular Modeling and Bioinformatics}  \\      \vspace{0.20in}
  {\large       \htmladdnormallinkfoot{Theoretical and Computational Biophysics Group}{http://www.ks.uiuc.edu/}  \\
                Beckman Institute for Advanced Science and Technology \\
                University of Illinois at Urbana-Champaign            \\
                405 N. Mathews                                        \\
                Urbana, IL  61801                                     \\
                \vspace{0.1in} \htmladdnormallink{{\tt http://www.ks.uiuc.edu/Research/vmd/}}{http://www.ks.uiuc.edu/Research/vmd/} \\
  }
  \vspace{0.5in}
  {\Large \bf Description}
  \vspace{0.1in}
\end{center}

{\DOCDESC}
\vspace{0.1in} \\
VMD development is supported by the National Institutes of Health
under grant numbers NIH~9P41GM104601 and 5R01GM098243-02.

\newpage
%%%%%%%%%%%%%%%%%%%%%%%%%%%%%%%%%%%%%%%%%%%%%%%%%%%%%%%%%%%%%%%%%%%%%%%%%%%
%cr                                                                       
%cr            (C) Copyright 1995-2009 The Board of Trustees of the            
%cr                        University of Illinois                         
%cr                         All Rights Reserved                           
%cr                                                                       
%%%%%%%%%%%%%%%%%%%%%%%%%%%%%%%%%%%%%%%%%%%%%%%%%%%%%%%%%%%%%%%%%%%%%%%%%%%

%%%%%%%%%%%%%%%%%%%%%%%%%%%%%%%%%%%%%%%%%%%%%%%%%%%%%%%%%%%%%%%%%%%%%%%%%%%%
% RCS INFORMATION:
%
%       $RCSfile: vmd_ref.tex,v $
%       $Author: johns $        $Locker:  $                $State: Exp $
%       $Revision: 1.74 $      $Date: 2014/12/29 03:25:57 $
%
%%%%%%%%%%%%%%%%%%%%%%%%%%%%%%%%%%%%%%%%%%%%%%%%%%%%%%%%%%%%%%%%%%%%%%%%%%%%
% DESCRIPTION:
%
% referencing this work, and credits and acknowledgments
%%%%%%%%%%%%%%%%%%%%%%%%%%%%%%%%%%%%%%%%%%%%%%%%%%%%%%%%%%%%%%%%%%%%%%%%%%%%

\section{Registering \VMD}

\VMD\ is made available free of charge for all interested 
end-users of the software (but please see the Copyright and 
Disclaimer notices).  Please check the current \VMD\ license 
agreement for details.  Registration is part of our software download
procedure.  Once you've filled out the forms on the \VMD\ download
area and have read and agreed to the license, you are finished with
the registration process.

\section{Citation Reference}
The authors request that any published work or images created using \VMD\ 
include the following reference:

\begin{description}
  \item{Humphrey, W., Dalke, A. and Schulten, K.,} ``VMD - Visual Molecular
Dynamics'' {\em J. Molec. Graphics} {\bf 1996}, {\em 14.1}, 33-38.
\end{description}

\VMD\ has been developed by the 
Theoretical and Computational Biophysics Group at the
Beckman Institute for Advanced Science and Technology of the
University of Illinois at Urbana-Champaign.
This work is supported by the National Institutes of Health under
grant numbers NIH~9P41GM104601 and 5R01GM098243-02.

\section{Acknowledgments}

The authors would particularly like to thank those individuals 
who have contributed suggestions and improvements, particularly 
those contributing new features.  Special thanks go to 
Joshua Anderson, Anton Arkhipov, Andrew Dalke, Michael Bach, 
Alexander Balaeff, Ilya Balabin, Robert Brunner,
Eamon Caddigan, Jordi Cohen, Simon Cross,
Markus Dittrich, John Eargle, 
Peter Freddolino, Todd Furlong, Luis Gracia, 
Paul Grayson, Justin Gullingsrud, James Gumbart,
David Hardy, Konrad Hinsen,
Barry Isralewitz, Sergei Izrailev, Robert Johnson,
Axel Kohlmeyer, Michael Krone, Michelle Kuttell, Benjamin Levine,
John Mongan, Jim Phillips, Elijah Roberts, Jan Saam, 
Charles Schwieters, Marcos Sotomayor, Alexander Spaar,
John E. Stone, Johan Strumpfer,
Alexey Titov, Leonardo Trabuco, Dan Wright, and Kirby Vandivort.
The entire \VMD\ user community now benefits from your contributions.

The authors would like to thank individuals
who have indirectly helped with development 
by making suggestions, pushing for new features, and trying out buggy code.  
Thanks go to 
Aleksei Aksimentiev, Daniel Barsky, Axel Berg, 
Tom Bishop, Robert Brunner, Ivo Hofacker, Mu Gao, 
James Gumbart, Xiche Hu, Tim Isgro, Dorina Kosztin, Ioan Kosztin, 
Joe Landman, Ilya Logunov, Clare Macrae, Amy Shih, 
Lukasz Salwinski, Stephen Searle, Charles Schwieters, 
Ari Shinozaki, Svilen Tzonev, Emad Tajkhorshid, Michael Tiemann, 
Elizabeth Villa, Raymond de Vries, Simon Warfield,
Willy Wriggers, Dong Xu, and Feng Zhou.  

Many external libraries and packages are used in \VMD, and the 
program would not be as capable without them.  
The authors of \VMD wish to thank
the authors of FLTK;
the authors of Tcl and Tk; 
the authors of Python; 
the authors of VRPN;
Jon Leech for uniform point distributions;
Amitabh Varshney for SURF;
Dmitrij Frishman for developing STRIDE; 
Jack Lund for the url\_get perl script; 
Brad Grantham for the ACTC triangle consolidation library;
John E. Stone for the Tachyon ray tracer, WorkForce threading and timer routines, hash table code, and Spaceball drivers;
\index{rendering!Tachyon}
\index{spaceball!driver}
and 
Ethan Merrit for one of the ribbon drawing algorithms. 


\section{Copyright and Disclaimer Notices}
\index{VMD!copyright}
\index{copyright}
\begin{center}
VMD is  Copyright \copyright\ 1995-2015 
Theoretical and Computational Biophysics Group and the \\
Board of Trustees of the University of Illinois
\end{center}

\noindent
Portions of this code are copyright \copyright\ 1997-1998 Andrew Dalke.

\bigskip

The terms for using, copying, modifying, and distributing VMD are
specified by the VMD License.  The license agreement is distributed
with VMD in the file LICENSE. If for any reason you do not have this
file in your distribution, it can be downloaded from: \\ 
{\tt http://www.ks.uiuc.edu/Research/vmd/current/LICENSE.html} 
\\
Some of the code and executables used by \VMD\ have their own 
usage restrictions:  
\begin{itemize}

\LISTITEM{ACTC}\index{ACTC}
ACTC, the triangle consolidation library used in some versions of VMD,
is Copyright (C) 2000, Brad Grantham and Applied Conjecture,
all rights reserved.

Redistribution and use in source and binary forms, with or without
modification, are permitted provided that the following conditions
are met:
\\
  1. Redistributions of source code must retain the above copyright
     notice, this list of conditions and the following disclaimer.
\\
  2. Redistributions in binary form must reproduce the above copyright
     notice, this list of conditions and the following disclaimer in the
     documentation and/or other materials provided with the distribution.
\\
  3. All advertising materials mentioning features or use of this software
     must display the following acknowledgment:
       This product includes software developed by Brad Grantham and
       Applied Conjecture.
\\
  4. Neither the name Brad Grantham nor Applied Conjecture
     may be used to endorse or promote products derived from this software
     without specific prior written permission.
\\
  5. Notification must be made to Brad Grantham about inclusion of this
     software in a product including the author of the product and the name
     and purpose of the product.  Notification can be made using email
     to Brad Grantham's current address (grantham@plunk.org as of September
     20th, 2000) or current U.S. mail address.

\LISTITEM{Python}\index{Python}
Python is made available subject to the terms and conditions in CNRI's
License Agreement. This Agreement together with Python may be
obtained from a proxy server on the Internet using the following
URL: {\tt http://hdl.handle.net/1895.22/1012}

\LISTITEM{PCRE}\index{PCRE}
The Perl Compatible Regular Expressions (PCRE) library used in VMD
was written by Philip Hazel and is Copyright (c) 1997-1999 
University of Cambridge.  
\\
Permission is granted to anyone to use this software for any purpose on any
computer system, and to redistribute it freely, subject to the following
restrictions:
\\
1. This software is distributed in the hope that it will be useful,
   but WITHOUT ANY WARRANTY; without even the implied warranty of
   MERCHANTABILITY or FITNESS FOR A PARTICULAR PURPOSE.
\\
2. The origin of this software must not be misrepresented, either by
   explicit claim or by omission.
\\
3. Altered versions must be plainly marked as such, and must not be
   misrepresented as being the original software.
\\
4. If PCRE is embedded in any software that is released under the GNU
   General Purpose License (GPL), then the terms of that license shall
   supersede any condition above with which it is incompatible.

\LISTITEM{STRIDE}\index{stride}
STRIDE, the program used for secondary structure calculation, is
free to both academic and commercial sites provided that STRIDE will
not be a part of a package sold for money.  The use of STRIDE in
commercial packages is not allowed without a prior written
commercial license agreement.  See
{\tt http://www.embl-heidelberg.de/argos/stride/stride\_info.html}

\LISTITEM{SURF}\index{surf}
The source code for SURF is copyrighted by the original author,
Amitabh Varshney, and the University of North Carolina at Chapel Hill.
Permission to use, copy, modify, and distribute this software and its
documentation for educational, research, and non-profit purposes is
hereby granted, provided this notice, all the source files, and the
name(s) of the original author(s) appear in all such copies.
\\
BECAUSE THE CODE IS PROVIDED FREE OF CHARGE, IT IS PROVIDED "AS IS" AND
WITHOUT WARRANTY OF ANY KIND, EITHER EXPRESSED OR IMPLIED.
\\
This software was developed and is made available for public use with
the support of the National Institutes of Health, National Center for
Research Resources under grant RR02170.

\LISTITEM{Tachyon}\index{Tachyon}
The Tachyon multiprocessor ray tracing system and derivative code built
into VMD is Copyright (c) 1994-2015 by John E. Stone.
See the Tachyon distribution for redistribution and licensing information.
\index{rendering!Tachyon}
% Permission is granted to use Tachyon freely with VMD. -- JES 2006

\LISTITEM{Desmond and Maestro plugins by D. E. Shaw Research}
Copyright 2009, D. E. Shaw Research, LLC
All rights reserved.
\\
Redistribution and use in source and binary forms, with or without
modification, are permitted provided that the following conditions are
met:
\\
  Redistributions of source code must retain the above copyright
  notice, this list of conditions, and the following disclaimer.
\\
  Redistributions in binary form must reproduce the above copyright
  notice, this list of conditions, and the following disclaimer in the
  documentation and/or other materials provided with the distribution.
\\
  Neither the name of D. E. Shaw Research, LLC nor the names of its
  contributors may be used to endorse or promote products derived from
  this software without specific prior written permission.
\\
THIS SOFTWARE IS PROVIDED BY THE COPYRIGHT HOLDERS AND CONTRIBUTORS
"AS IS" AND ANY EXPRESS OR IMPLIED WARRANTIES, INCLUDING, BUT NOT
LIMITED TO, THE IMPLIED WARRANTIES OF MERCHANTABILITY AND FITNESS FOR
A PARTICULAR PURPOSE ARE DISCLAIMED. IN NO EVENT SHALL THE COPYRIGHT
OWNER OR CONTRIBUTORS BE LIABLE FOR ANY DIRECT, INDIRECT, INCIDENTAL,
SPECIAL, EXEMPLARY, OR CONSEQUENTIAL DAMAGES (INCLUDING, BUT NOT
LIMITED TO, PROCUREMENT OF SUBSTITUTE GOODS OR SERVICES; LOSS OF USE,
DATA, OR PROFITS; OR BUSINESS INTERRUPTION) HOWEVER CAUSED AND ON ANY
THEORY OF LIABILITY, WHETHER IN CONTRACT, STRICT LIABILITY, OR TORT
(INCLUDING NEGLIGENCE OR OTHERWISE) ARISING IN ANY WAY OUT OF THE USE
OF THIS SOFTWARE, EVEN IF ADVISED OF THE POSSIBILITY OF SUCH DAMAGE.


\end{itemize}



\newpage
\section{Obtaining VMD Source and Binary Distributions}
The VMD source code and binary distributions can be obtained 
after registering at the VMD web page.  
Download the appropriate distribution file with your web browser.
Windows binary distributions are self extracting, so once the
distribution file is downloaded, proceed to the installation directions
below.

For source distributions and Unix binary distributions, uncompress and 
untar the file.  This will produce a subdirectory named {\tt vmd-\VMDVER}.  
Unless otherwise specified, all references to VMD code will be from this 
subdirectory, so {\tt cd} there.

\section {Installing a Pre-Compiled Version of VMD}
To install the pre-compiled Windows version of VMD,
simply run the self-extracting executable, and it will start the
VMD Windows installer program, which includes built-in help.
This process is automated and should be familiar to most
Windows users.  When installing VMD be sure that you have administrator
privileges.

To install the pre-compiled MacOS X bundle version of VMD, open the
VMD disk image and drag the VMD application into an appropriate 
directory.  Once the VMD application has been placed appropriately
it should be ready for immediate use as no other installation
steps are required.

To install the pre-compiled Unix version of VMD, then
only three steps remain to be done after you uncompress
and untar the distribution. 
\begin{itemize}
\item Edit the {\tt configure} script.  If necessary, 
change the following values:
{\tt
\begin{verbatim}
     $install_bin_dir
\end{verbatim}
}
\begin{verbatim}
	  This is the location of the startup script 'vmd'.  It should
	  be located in the path of users interested in running VMD.
\end{verbatim}
{\tt
\begin {verbatim}
     $install_library_dir
\end{verbatim}
}
\begin{verbatim}
	  This is the location of all other VMD files.  This includes
	  the binary and helper scripts.  It should not be in the path.
\end{verbatim}
\item Next generate the Makefile based on these configuration
variables.  This is done by running {\tt ./configure} .
\item After configuration is complete, {\tt cd} to the {\tt src}
directory and type {\tt make install}.  This will put the code in
the two directories listed above.  After this, you just type {\tt vmd}
to begin, provided that {\tt vmd} is in your path.
\end{itemize}


%%%%%%%%%%%%%%%%%%%%%%%%%%%%%%%%%%%%%%%%%%%%%%%%%%%%%%%%%%%%%%%%%%%%%%%%%%%
%cr                                                                       
%cr            (C) Copyright 1995 The Board of Trustees of the            
%cr                        University of Illinois                         
%cr                         All Rights Reserved                           
%cr                                                                       
%%%%%%%%%%%%%%%%%%%%%%%%%%%%%%%%%%%%%%%%%%%%%%%%%%%%%%%%%%%%%%%%%%%%%%%%%%%

%%%%%%%%%%%%%%%%%%%%%%%%%%%%%%%%%%%%%%%%%%%%%%%%%%%%%%%%%%%%%%%%%%%%%%%%%%%%
% RCS INFORMATION:
%
%       $RCSfile: ig_custom.tex,v $
%       $Author: jordi $        $Locker:  $                $State: Exp $
%       $Revision: 1.20 $      $Date: 2003/04/09 21:51:52 $
%
%%%%%%%%%%%%%%%%%%%%%%%%%%%%%%%%%%%%%%%%%%%%%%%%%%%%%%%%%%%%%%%%%%%%%%%%%%%%
% DESCRIPTION:
%
% INSTALLATION GUIDE : customizing VMD
%   1) .vmdrc file
%   2) .vmdsensors file
%%%%%%%%%%%%%%%%%%%%%%%%%%%%%%%%%%%%%%%%%%%%%%%%%%%%%%%%%%%%%%%%%%%%%%%%%%%%

%\chapter{Customizing \VMD}
%\label{chapter:ig:custom}

\section{Customizing \VMD\ Startup}

The Unix version of \VMD\ reads in several data files (if they exist) 
when it starts up.
These files control the initial appearance and behavior of \VMD\ at
the start, and may be customized to suit each users particular tastes.
Default versions of these files are placed in the {\tt
INSTALLLIBDIR} directory (usually {\tt /usr/local/lib/vmd}). 
While each user may specify to use different versions of these files,
unless this is done the commands and values in the default files are
used.  In this way, an administrator may customize the default
behavior of \VMD\ for all users, while allowing each user the option
to change the default behavior however they choose.  This chapter
describes each of these data files.

Several configurable parameters may also be set in a number of ways, including by command-line options or by environment variables.  The order of precedence of these methods is as follows (highest precedence to lowest):
\begin{enumerate}
  \item Command-line options (see the Users Guide).
  \item Environment variable settings (see the Users Guide).
  \item Built-in defaults, as specified by compilation configurable parameters.
        These are used only if no other values are specified by the other 
        methods mentioned in this list.
\end{enumerate}

%\begin{centering}
%  \displayepsf[htb] (pictures/screen_params.eps scaled 800) {
%    \caption{Relationship between screen height, screen distance to origin,
%			and the viewer.}
%    \label{fig:ig:screen}
%  }
%\end{centering}

\section{The {\tt .vmdrc} and {\tt vmd.rc} files}

After initialization is complete, \VMD\ reads the {\em startup} file.  This file contains text commands
for \VMD\ to execute, just as if they had been entered at the \VMD\
text console command prompt.  The file can contain any number of
commands, including blank lines and comment lines (which begin with
the {\tt \#} character).  If an error is encountered while reading
this file, the command in error is skipped and processing of the file
continues.

The base filename for this startup file is {\tt .vmdrc} by default
on Unix systems and {\tt vmd.rc} on Windows;
this is determined by the configuration parameter {\tt STARTUPFILENAME}.
\VMD\ searches for this file in a number of locations,
and reads in the {\em first} version of the file it finds.  The order
of searching for the file is:
\begin{enumerate}
  \item {\tt ./STARTUPFILENAME}
  \item {\tt \$HOME/STARTUPFILENAME}
  \item {\tt INSTALLLIBDIR/STARTUPFILENAME}
\end{enumerate}
See the Users Guide for a list of all \VMD\ text commands.

\section{The {\tt .vmdsensors} file}

If \VMD\ is compiled with the VRPN option, it will look for files that specify how to access the external spatial
tracking devices. These files are read whenever \VMD\ is told to
initialize a specific external device.  The Tracker library will load 
the first file it finds in the following search order:
\begin{enumerate}
  \item {\tt \$HOME/.vmdsensors}
  \item The {\tt \$VMDSENSOR}  environment variable.
  \item {\tt INSTALLLIBDIR/.vmdsensors}
\end{enumerate}
%NOT TRUE RIGHT NOW: In the event of duplicated information, the device specifications found in 
%{\tt \$HOME/.vmdsensors} have precedence over those in {\tt \$VMDSENSOR}.

This last file ({\tt INSTALLLIBDIR/.vmdsensors}) contains extensive 
comments on how to configure the sensor description files
properly. If the VRPN option is omitted when compiling \VMD, 
this file is not used.


\section{What to Do If It Doesn't Work}

  If you are running a VMD binary which has been built with a
native OpenGL implementation (i.e. not Mesa), you should make sure
that you have the vendor provided OpenGL runtime libraries and the
X server extensions correctly installed.  SGI systems normally have
the OpenGL runtime support installed on them.  Sun, HP, and IBM systems
often do not come with OpenGL support by default.  If you don't have the 
OpenGL runtime libraries for these systems, they can be downloaded for free
from the Sun, HP, and IBM web sites respectively.  Each of the vendor's OpenGL 
implementations generally include "install check" programs which verify
the correct installation and operation of the OpenGL libraries.

\begin{itemize}
\item Sun's OpenGL site is at {\tt http://www.sun.com/software/graphics/opengl/}
\item HP's OpenGL site is at {\tt http://www.hp.com/unixwork/products/opengl.html} 
\item IBM's OpenGL site is at {\tt http://www.austin.ibm.com/software/OpenGL/}
\end{itemize}
We suggest that you check that you are doing everything correctly,
and if it still doesn't work, report the problem by e-mail to
{\tt vmd@ks.uiuc.edu}. We'll try to help you.

 \section {Compiling Your Own Version of VMD}
   If for some reason you want to recompile VMD, then you will need to read 
 the rest of this document.  Most users will want to use the binary 
 distributions we provide since they have been thorougly tested prior to
 release.  It may be necessary for you to compile your own version of VMD
 in cases where we do not provide a binary for your platform, or
 when the provided binaries will not run correctly with a particular
 version of your operating system.  
 Full compilation instructions for VMD are found in the 
 online VMD Programmer's Guide:
   {\tt http://www.ks.uiuc.edu/Research/vmd/doxygen/}

\end{document}



