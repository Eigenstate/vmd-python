%%%%%%%%%%%%%%%%%%%%%%%%%%%%%%%%%%%%%%%%%%%%%%%%%%%%%%%%%%%%%%%%%%%%%%%%%%%%
% RCS INFORMATION:
%
%       $RCSfile: ug_rendering.tex,v $
%       $Author: johns $        $Locker:  $                $State: Exp $
%       $Revision: 1.44 $      $Date: 2014/12/29 03:18:55 $
%
%%%%%%%%%%%%%%%%%%%%%%%%%%%%%%%%%%%%%%%%%%%%%%%%%%%%%%%%%%%%%%%%%%%%%%%%%%%%
% DESCRIPTION:
%  rendering
%
%%%%%%%%%%%%%%%%%%%%%%%%%%%%%%%%%%%%%%%%%%%%%%%%%%%%%%%%%%%%%%%%%%%%%%%%%%%%
\chapter{Scene Export and Rendering}
\label{ug:topic:rendering}
\index{rendering}

One of the most common tasks performed by users of VMD is producing
images which can be loaded into other programs or used in printed
documents, posters, slides, and transparencies.  The Render window 
provides a simple mechanism for generating image files from
snapshots of the VMD graphics window and through the use of external
rendering and ray tracing programs. 

\section{Screen Capture Using Snapshot}

The simplest way to produce raster image files in \VMD\ is to use
the ``Snapshot'' feature.  The snapshot feature captures the contents
of the VMD graphics window, and saves them to a raster image file.
On Unix systems, the captured image is written to a 24-bit color 
Truevision ``Targa'' file.  On Windows systems, the captured image is written 
to a 24-bit color Windows Bitmap, or ``BMP'' file.
To use the {\tt snapshot} feature, simply open the 
\hyperref{{\sf Render}}{{\sf Render}[\S~}{]}{ug:ui:window:render} window 
and choose the {\tt snapshot} option.  
\VMD\ will capture the contents of the graphics window, and attempt to 
save the resulting image to the filename given in the {\sf Render} window.
You may find that it is important not to have other windows or 
cursors in front of the VMD graphics display when using snapshot, 
since the resulting images may include obscuring windows or cursors.  
This is a platform-dependent behavior, so you will need to determine 
if your system does this or not.

\section{Higher Quality Rendering}

Sometimes images produced by screen capture aren't good enough; 
you may want a very large, high quality picture, or a picture 
with shadows, reflections, or high quality rendering of transparent surfaces.  
While \VMD\ generally produces nice looking images in its graphics window,
it was designed to generate its images very rapidly to maximize
interactivity, which precludes the use of photorealistic rendering 
techniques that would slow down the operation of whole program.
Instead of producing high quality images directly, \VMD\ writes 
scene description files which can be used as input to several popular 
scanline rendering and ray tracing programs.  
Tables \ref{ug:table:render} lists the currently 
supported output formats, and where appropriate rendering software 
may be obtained.

\begin{table}[htb]
\begin{tabular}{|l|l|} 
  \hline
  Name & Description \\ 
  \hline\hline
  ART$^1$                  & Simple VORT ray tracer \\
  Gelato                   & NVIDIA Gelato PYG Format \\
  PostScript               & Simple Vector PostScript Output \\
  POV3$^2$                 & POV-Ray 3.x ray tracer \\
  Radiance$^3$             & Radiosity ray tracer \\
  Raster3D$^4$             & Fast raster file generator \\
  Rayshade$^5$             & Rayshade ray tracer \\
  RenderMan                & PIXAR RenderMan RIB Format, render with Aqsis, \\
                           & Pixie, PRMan, RenderDotC \\
  STL                      & Stereolithography format, triangles only \\
  Tachyon$^6$              & Fast, high quality parallel ray tracer \\
  TachyonInternal$^6$      & Faster, built-in Tachyon engine that generates images directly \\
                           & from VMD internal data structures with no intermediate step \\
  TachyonLOptiXInternal    & Built-in GPU-accelerated version of Tachyon \\
                           & for NVIDIA GPUs (available on supported platforms) \\
  TachyonLOptiXInteractive & Interactive, built-in, GPU-accelerated version of Tachyon \\
                           & for NVIDIA GPUs (available on supported platforms) \\
  VRML-1                   & Virtual Reality Markup Language V1.0 \\
  VRML-2                   & Virtual Reality Markup Language V2.0 \\
  Wavefront                & Wavefront .OBJ/.MTL scene format, loads into 3DS~Max, \\
                           & Blender, Maya, and others \\
  X3D$^7$                  & X3D declarative scene format \\
  X3DOM$^8$                & Open source HTML5-based X3D viewing system \\
\hline

\end{tabular}

  $^1$Available from \htmladdnormallink{{\tt http://bund.com.au/\~{}dgh/eric/}} 
    {http://bund.com.au/~dgh/eric/} along with the rest of VORT package \\
  $^2$See \htmladdnormallink{{\tt http://www.povray.org/}}
    {http://www.povray.org/} \\
  $^3$See \htmladdnormallink{{\tt http://radsite.lbl.gov/radiance/HOME.html}}
    {http://radsite.lbl.gov/radiance/HOME.html} \\
  $^4$See \htmladdnormallink{{\tt http://www.bmsc.washington.edu/raster3d/}}
    {http://www.bmsc.washington.edu/raster3d/} \\
  $^5$See \htmladdnormallink{{\tt http://graphics.stanford.edu/\~{}cek/rayshade/rayshade.html}}
    {http://graphics.stanford.edu/~cek/rayshade/rayshade.html} \\
  $^6$See \htmladdnormallink{{\tt http://www.photonlimited.com/\~{}johns/tachyon/}}
    {http://www.photonlimited.com/~johns/tachyon/} \\
  $^7$See \htmladdnormallink{{\tt http://www.web3d.org/x3d/}}
    {http://www.web3d.org/x3d} \\
  $^8$See \htmladdnormallink{{\tt http://www.x3dom.org/}}
    {http://www.x3dom.org/} \\
\caption{Miscellaneous Rendering Options}
\label{ug:table:render}
\index{rendering!list of supported renderers}
\index{rendering!methods}
\index{rendering!ART}
\index{rendering!Gelato}
\index{rendering!PostScript}
\index{rendering!POV-Ray}
\index{rendering!Radiance}
\index{rendering!Raster3D}
\index{rendering!Rayshade}
\index{rendering!RenderMan}
\index{rendering!STL}
\index{rendering!VRML-1}
\index{rendering!VRML-2}
\index{rendering!Tachyon}
\index{rendering!TachyonInternal}
\index{rendering!TachyonLOptiXInternal}
\index{rendering!TachyonLOptiXInteractive}
\index{rendering!Wavefront}
\index{rendering!X3D}
\index{rendering!X3DOM}
\index{STL files}
\index{VRML files}
\index{X3D files}
\end{table}

Making the raster image is usually a two step process, 
e.g. with the exception of the built-in TachyonInternal 
and TachyonLOptiXInteractive renderers.  First you must make
a scene description file suitable for the chosen rendering program, and then
execute the program using the new file as input to produce the raster
image output.  The external rendering programs typically
support different output file formats, which may need to be converted to
something more appropriate for you.  It is impossible to predict what
that might be, so we'll describe how to convert the different file types to
Targa and let you use the tools listed in Table~\ref{ug:table:render}
to get what you need.
Raster3D, Tachyon, and POV-Ray can produce Targa files, so you don't need 
to do anything but specify this output format.  
Rayshade creates RLE image files, which can be converted using ImageMagick.
Radiance generates an .oct file, which can be converted with the 
{\tt rview} and {\tt rpict} commands included in the Radiance distribution.

The free program {\tt display} from ImageMagick -- see  \htmladdnormallink{{\tt
http://www.imagemagick.org/}} {http://www.imagemagick.org/} -- should be able
to read and convert between all of these formats.

We suggest using Tachyon or Raster3D as they are 
generally the fastest programs.  These programs are easy to 
understand, and are fast even when rendering very complex molecules.

The generated scene files are plain text so they are very easy to modify.  
This is most often done to create a larger raster
file, though some have other global options which you may wish to
change.  For instance, by default the Raster3D file turns shadows on.
We suggest you consult the relevant renderer's documentation to determine 
what can be modified in the file.

To actually render the current image into an output file, first set up
the graphics in \VMD\ just as you wish the output to appear.  Then,
either use the \hyperref{Render window}{Render window 
[\S~}{]}{ug:ui:window:render}, or the following text command, to create
the input file and start the rendering program going:
\vspace{0.1in}
\\
{\bf render {\em method} {\em filename} [{\em render command}]}
\vspace{0.1in}
\\
{\em method} is one of the names listed in the first column of table
\ref{ug:table:render}, and {\em filename} is the name of the file which
will contain the resulting image processing program script.  Any text
following this will be used as a command to be run to process the file.
If {\tt \%s} appear in the command string, they will be replaced with the
name of the script file.

\section{Caveats}
\index{rendering!caveats and considerations}

When \VMD\ creates the output file it will try to match the current
view and screen size.  For the most part it does a good a job but
there can be some problems.  The colors in the final raster image can 
sometimes look different from what is seen in the VMD graphics window.
This is because the external rendering programs use different shading
equations and algorithms from what VMD uses.
\\
Potential rendering discrepencies include:
\begin{itemize}
  \item Geometry may look slightly different; 
        in \VMD\ curved surfaces are polygonalized and drawn using a
        number of polygonal facets, curved surfaces may be rendered entirely
        smoothly in the final output (which is generally looked upon as an
        improvement!)
  \item The rendered object colors or intensities may be slightly different
        due to different colormaps, gamma values, or lighting models;
        This is particularly true with the material properties used for
        performing complex shading.  VMD's real-time rendering of these
        material properties is often simplistic or limited compared to
        full-fledged photorealistic renderers, so there can potentially
        be big differences between implementations of transparency,
        specular highlights, etc.
  \item Many of the external renderers do not support true orthographic
        rendering.  This can be ``faked'' by translating the camera
        very far away from the molecule, followed by zooming the camera
        so that the image size is acceptable again.  This will significantly
        decrease the perspective effect, but is not a true orthographic
        projection.
  \item The rendering commands do not currently support stereo output,
        so even if the display is currently in stereo mode, a non-stereo
        perspective will be used for the rendering program input script;
        Rendering in stereo is accomplished by setting the display mode
        to ``left'', then rendering an image, followed by ``right'', and
        rendering again.  This will yield a stereo pair to the best of
        VMD's ability with the external rendering program.
  \item The near and far clipping planes are ignored by all external renderers;
  \item Text is generally not available as a graphics primitive in the
        renderer scene languages, so label text will not
        appear, although the lines of bond, angle, etc. labels will be
        drawn.  The only exception is in Postscript output, which supports
        text output.
  \item Dotted spheres are not drawn with dots.
  \item The background color may be black, as not all output formats support
        a background color other than black;
\end{itemize}


\section{One Step Printing}
A frequently asked question is ``How can I quickly get a printout of
the VMD Display?'' There are several one step solutions to this
problem, a few are listed below:
\index{postscript}
\begin{itemize}
\item Choose the {\tt snapshot} option and convert the resulting
      image to your desired image format using ImageMagick or similar tools.
\item Choose the {\tt TachyonInternal} option and convert the resulting
      image to your desired image format using ImageMagick or similar tools.
\end{itemize}


\section{Making Stereo Images}
\index{rendering!stereo}
\index{example scripts!Tcl!drawing!rendering stereo pairs}
Stereoscopic images can be rendered with a simple sequence 
of text commands, cycling between the left and right monoscopic 
stereo modes and exporting one scene for each eye:
\index{TachyonInternal}
{\tt
\begin{verbatim}
        display stereo left
        render TachyonInternal left.tga
        display stereo right
        render TachyonInternal right.tga
\end{verbatim}
}

External renderers don't always support the ability to draw stereo images.  
In principle, it is possible to write the scene to the file twice with the
appropriate transformations applied to make the view correct for each
eye, but then the shadows would be incorrect.
Instead, we suggest making one image of the current scene, then shift
the molecules to the left (or right) to make the other image.  
%% (Note that neither the stage nor the axes will move, so they will not be in
%% stereo.)  
The text commands for this are something like:
\index{Raster3D}
{\tt
\begin{verbatim}
        display stereo off
        render Raster3D left.r3d
        trans by -.1 0 0
        render Raster3D right.r3d
\end{verbatim}
}
\noindent The two files must then be rendered to produce the rgb file.
As it turns out, this method makes it easy to produce stereo images of
ordinary Raster3D files.  Since \VMD\ can read the Raster3D format,
all you have to do is read the file and then execute the commands listed
above.  The text commands for generating left or right views also have
equivalents in the GUI under the {\tt Stereo} option of the Display window.


\section{Making a Movie}
\index{movies}
\index{animation!movie}
It is possible to make movies with \VMD, through the use of Tcl or Python
scripts, or with the ``vmdmovie'' extension included with \VMD.
Several movie making scripts are provided in the \htmladdnormallink{\VMD\ 
script library}{http://www.ks.uiuc.edu/Research/vmd/script_library/}
on the \VMD\ home page.  These scripts can be used as-is, or they can
be customized to perform complex animation tasks beyond the scope of
this user guide.  In general, movies are created by driving
render commands with a script, producing a sequence of individual
image files.  When the script has completed rendering all of the 
individual frames, the images are ready for import into an
animation package, or can be converted to one of several popular
compressed movie formats by further processing.  The ``vmdmovie'' 
extension provided with \VMD\ completely automates the movie creation
process, though it requires a number of software packages be installed
in order to do the job.  Please see the separate documentation on the 
movie scripts and ``vmdmovie'' in the \VMD\ script library.


