%%%%%%%%%%%%%%%%%%%%%%%%%%%%%%%%%%%%%%%%%%%%%%%%%%%%%%%%%%%%%%%%%%%%%%%%%%%
%cr                                                                       
%cr            (C) Copyright 1995 The Board of Trustees of the            
%cr                        University of Illinois                         
%cr                         All Rights Reserved                           
%cr                                                                       
%%%%%%%%%%%%%%%%%%%%%%%%%%%%%%%%%%%%%%%%%%%%%%%%%%%%%%%%%%%%%%%%%%%%%%%%%%%

%%%%%%%%%%%%%%%%%%%%%%%%%%%%%%%%%%%%%%%%%%%%%%%%%%%%%%%%%%%%%%%%%%%%%%%%%%%%
% RCS INFORMATION:
%
%       $RCSfile: ig_intro.tex,v $
%       $Author: johns $        $Locker:  $                $State: Exp $
%       $Revision: 1.9 $      $Date: 2000/05/16 18:52:35 $
%
%%%%%%%%%%%%%%%%%%%%%%%%%%%%%%%%%%%%%%%%%%%%%%%%%%%%%%%%%%%%%%%%%%%%%%%%%%%%
% DESCRIPTION:
%
% INSTALLATION GUIDE : introduction chapter
%   1) General description of document and purpose
%   2) Organization of document
%   3) How to use this guide
%   4) For more information ...
%   5) Contacting the authors
%   6) Referencing this work
%%%%%%%%%%%%%%%%%%%%%%%%%%%%%%%%%%%%%%%%%%%%%%%%%%%%%%%%%%%%%%%%%%%%%%%%%%%%

\chapter{Introduction}
\label{chapter:ig:intro}

% general intro text
\VMD\ is a molecular visualization and analysis program designed to
be used for interactive display of molecular systems, particularly
biopolymers such as proteins, nucleic acids, and biological
assemblies such as membrane lipid bilayers.  As described in the
\VMD\ Programmer's and User's Guides, \VMD\ has several goals:
\begin{itemize}
  \item General molecular visualization;
  \item Visualization of dynamic molecular data;
  \item Analysis of molecular structures;
  \item Display and control of molecular dynamics simulations;
  \item Support for several input and display (output) devices,
       including output image files;
  \item Modifiable and extensible program structure.
\end{itemize}

This manual, along with the Users
Guide and Programmers Guide, documents the use of \VMD.

% how the guide is organized
\section{Guide organization}

This \DOCTITLE\ details how to compile and install \VMD.  You have the
option of simply installing a precompiled version of
\VMD\ provided in the standard distribution, or the option of recompiling \VMD\
with a different set of features and options.  Depending on the method
preferred, different chapters of this Guide will be relevant.  This
manual is organized into the following chapters following this
introduction:

\begin{itemize}
  \BOLDLISTITEM{Obtaining and unpacking \VMD\ (chapter
\ref{chapter:ig:unpack})} The first step in installing \VMD;
this chapter should be read first.

  \BOLDLISTITEM{Compiling \VMD\ (chapter \ref{chapter:ig:compile})}
Describes how to compile \VMD.  The program may be compiled for a
variety of operating system versions, and may optionally include
or exclude a number of features.  Chapter \ref{chapter:ig:compile}
lists all the operating systems for which \VMD\ may be compiled,
describes all the optional program features which may be included
in the compiled executable, and mentions all the configurable
parameters which may be preferentially customized.  The
\VMD\ executables provided with the standard distribution are compiled
with a default set of features, as described in chapter
\ref{chapter:ig:install}.

  \BOLDLISTITEM{Installing \VMD\ (chapter \ref{chapter:ig:install})} 
Describes how to install the \VMD\ executables and data files, once
\VMD\ has been properly compiled.  The standard distribution includes
precompiled versions of \VMD\ for several different Unix operating
systems, as listed in table \ref{table:ig:execs}.  You may install
one of these precompiled versions as is, or you may refer to
chapter \ref{chapter:ig:compile} to first compile a new version of \VMD\
before installing it.

  \BOLDLISTITEM{Customizing \VMD\ (chapter \ref{chapter:ig:custom})} 
Mentions how to edit the \VMD\ configuration data files in order to
customize the appearance and behavior of \VMD\ for a particular
environment.  This is done after \VMD\ has been successfully compiled
and installed.
\end{itemize}

\section{How to use this guide}
\label{section:ig:howtouse}

There are two ways to install \VMD:
\begin{enumerate}
  \item Use a provided precompiled version of \VMD; to do this, use
chapter \ref{chapter:ig:unpack} and chapter \ref{chapter:ig:install}. 
Currently, precompiled versions of \VMD\ are supplied for the following
operating systems (see table \ref{table:ig:execs}):
  \begin{itemize}
    \item SGI IRIX 5.X or IRIX 6.X.
  \end{itemize}
For other operating systems, the second method must be used. 
  \item Compile \VMD\ with a selected set of features and
parameters.  This requires first unpacking the distribution (chapter
\ref{chapter:ig:unpack}), compiling the program (chapter
\ref{chapter:ig:compile}), and then installing the executables and
data files (chapter \ref{chapter:ig:install}).  Chapter
\ref{chapter:ig:compile} lists the operating systems for which
\VMD\ is supported, and all possible program features and
user-selectable parameters.  You are also strongly advised to
get and install the FLTK or XForms, Tcl, and Tcl-X packages.
\end{enumerate}

Finally, after installation is complete, several factors may be customized
about how \VMD\ appears to users upon startup.
This is done by following the instructions in chapter \ref{chapter:ig:custom}.

% for more information ...
%%%%%%%%%%%%%%%%%%%%%%%%%%%%%%%%%%%%%%%%%%%%%%%%%%%%%%%%%%%%%%%%%%%%%%%%%%%
%cr                                                                       
%cr            (C) Copyright 1995 The Board of Trustees of the            
%cr                        University of Illinois                         
%cr                         All Rights Reserved                           
%cr                                                                       
%%%%%%%%%%%%%%%%%%%%%%%%%%%%%%%%%%%%%%%%%%%%%%%%%%%%%%%%%%%%%%%%%%%%%%%%%%%

%%%%%%%%%%%%%%%%%%%%%%%%%%%%%%%%%%%%%%%%%%%%%%%%%%%%%%%%%%%%%%%%%%%%%%%%%%%%
% RCS INFORMATION:
%
%       $RCSfile: vmd_moreinfo.tex,v $
%       $Author: johns $        $Locker:  $                $State: Exp $
%       $Revision: 1.25 $      $Date: 2006/04/17 02:12:16 $
%
%%%%%%%%%%%%%%%%%%%%%%%%%%%%%%%%%%%%%%%%%%%%%%%%%%%%%%%%%%%%%%%%%%%%%%%%%%%%
% DESCRIPTION:
%
% how to get more information about VMD and MDScope
%%%%%%%%%%%%%%%%%%%%%%%%%%%%%%%%%%%%%%%%%%%%%%%%%%%%%%%%%%%%%%%%%%%%%%%%%%%%

\section{For information on our other software}

\VMD\ is part of a suite of tools developed by the 
Theoretical and Computational Biophysics group at the University of Illinois. 

\begin{itemize}
\LISTITEM{\BIOCORE}\index{BioCoRE} 
BioCoRE is a web-based collaborative environment for structural biology
which provides tools to allow collaboration between researchers down the
hall or around the world.  Anyone with access to the internet and a
standard web browser can join BioCoRE and create or be added to research
projects, and information about a particular project is shared among all
members of that project.  More information is available at the 
\htmladdnormallinkfoot{BioCoRE home page}{http://www.ks.uiuc.edu/Research/biocore}

\LISTITEM{\NAMD}\index{NAMD} 
A parallel, object-oriented molecular
dynamics code designed for high-performance simulation of large
biomolecular systems.
\NAMD\ uses the CHARMM\mycite{}{} force field and file formats compatible
with both CHARMM and X-PLOR\mycite{}{}.
\NAMD\ supports both periodic and non-periodic boundaries with
efficient full electrostatics, multiple timestepping, constant
pressure and temperature ensemble simulation methods.
More information is available at the
\htmladdnormallinkfoot{NAMD home page}{http://www.ks.uiuc.edu/Researach/namd}

%% \LISTITEM{\JMV}\index{JMV}
%% JMV is an easy-to-use Java-based molecular viewer which uses Java3D to 
%% provide hardware accelerated rendering of molecules.  
%% JMV is designed as component software,
%% and can be used within other programs, or as a standalone application.  
%% More information is available at the
%% \htmladdnormallinkfoot{JMV home page}{http://www.ks.uiuc.edu/Research/jmv}

\LISTITEM{\MDTOOLS}\index{MDTools}
MDTools is a collection of programs, scripts, and utilities provided 
for researchers to make various modeling and simulation tasks easier. 
More information is available at the
\htmladdnormallinkfoot{MDTools home page}{http://www.ks.uiuc.edu/Development/MDTools}

%% \LISTITEM{Structural Biology Software Database}\index{Structural Biology Software Database} 
%% The database contains programs which are of interest to researchers in 
%% structural biology, quantum chemistry, and bioinformatics.  
%% The database entries are contributed by users, and are moderated to provide 
%% a high quality resource.  
%% More information is available at the
%% \htmladdnormallinkfoot{Structural Biology Software Database home page}{http://www.ks.uiuc.edu/Development/biosoftdb/}

\end{itemize}

For more information on our software efforts, see the 
\htmladdnormallinkfoot{Theoretical and Computational Biophysics Group home page}{http://www.ks.uiuc.edu/}. 



\label{section:ig:moreinfo}

% contacting the authors
%%%%%%%%%%%%%%%%%%%%%%%%%%%%%%%%%%%%%%%%%%%%%%%%%%%%%%%%%%%%%%%%%%%%%%%%%%%
%cr                                                                       
%cr            (C) Copyright 1995 The Board of Trustees of the            
%cr                        University of Illinois                         
%cr                         All Rights Reserved                           
%cr                                                                       
%%%%%%%%%%%%%%%%%%%%%%%%%%%%%%%%%%%%%%%%%%%%%%%%%%%%%%%%%%%%%%%%%%%%%%%%%%%

%%%%%%%%%%%%%%%%%%%%%%%%%%%%%%%%%%%%%%%%%%%%%%%%%%%%%%%%%%%%%%%%%%%%%%%%%%%%
% RCS INFORMATION:
%
%       $RCSfile: vmd_authors.tex,v $
%       $Author: johns $        $Locker:  $                $State: Exp $
%       $Revision: 1.14 $      $Date: 2009/04/29 16:06:53 $
%
%%%%%%%%%%%%%%%%%%%%%%%%%%%%%%%%%%%%%%%%%%%%%%%%%%%%%%%%%%%%%%%%%%%%%%%%%%%%
% DESCRIPTION:
%
% contacting the authors
%%%%%%%%%%%%%%%%%%%%%%%%%%%%%%%%%%%%%%%%%%%%%%%%%%%%%%%%%%%%%%%%%%%%%%%%%%%%

%%
%% used to use the \VMDAUTHORS macro, but this includes way too many people.
%%
\section{Contacting the authors}

The current developer of \VMD\ is John E. Stone.
The list of individuals that made signficant contributions to this 
version of VMD in the form of patches, bug fixes, and completely new plugins 
includes
Anton Arkhipov, Michael Bach, Robert Brunner,
Jordi Cohen, Simon Cross, Markus Dittrich, John Eargle, 
Peter Freddolino, Luis Gracia, Justin Gullingsrud, 
David Hardy, Konrad Hinsen, James Gumbart,
Robert Johnson, Axel Kohlmeyer, Michell Kuttel, 
John Mongan, Jim Phillips, Elijah Roberts, Jan Saam, 
Alexander Spaar, Marcos Sotomayor, 
Leonardo Trabuco, Dan Wright, and Kirby Vandivort.

We are very interested in and grateful for any user comments and
reports of program bugs or inaccuracies.  If you have any suggestions,
bug reports, or general comments about \VMD, please send them to
us at \vmdemail.


\label{section:ig:authors}

% credits, and referencing this work
%%%%%%%%%%%%%%%%%%%%%%%%%%%%%%%%%%%%%%%%%%%%%%%%%%%%%%%%%%%%%%%%%%%%%%%%%%%
%cr                                                                       
%cr            (C) Copyright 1995-2009 The Board of Trustees of the            
%cr                        University of Illinois                         
%cr                         All Rights Reserved                           
%cr                                                                       
%%%%%%%%%%%%%%%%%%%%%%%%%%%%%%%%%%%%%%%%%%%%%%%%%%%%%%%%%%%%%%%%%%%%%%%%%%%

%%%%%%%%%%%%%%%%%%%%%%%%%%%%%%%%%%%%%%%%%%%%%%%%%%%%%%%%%%%%%%%%%%%%%%%%%%%%
% RCS INFORMATION:
%
%       $RCSfile: vmd_ref.tex,v $
%       $Author: johns $        $Locker:  $                $State: Exp $
%       $Revision: 1.74 $      $Date: 2014/12/29 03:25:57 $
%
%%%%%%%%%%%%%%%%%%%%%%%%%%%%%%%%%%%%%%%%%%%%%%%%%%%%%%%%%%%%%%%%%%%%%%%%%%%%
% DESCRIPTION:
%
% referencing this work, and credits and acknowledgments
%%%%%%%%%%%%%%%%%%%%%%%%%%%%%%%%%%%%%%%%%%%%%%%%%%%%%%%%%%%%%%%%%%%%%%%%%%%%

\section{Registering \VMD}

\VMD\ is made available free of charge for all interested 
end-users of the software (but please see the Copyright and 
Disclaimer notices).  Please check the current \VMD\ license 
agreement for details.  Registration is part of our software download
procedure.  Once you've filled out the forms on the \VMD\ download
area and have read and agreed to the license, you are finished with
the registration process.

\section{Citation Reference}
The authors request that any published work or images created using \VMD\ 
include the following reference:

\begin{description}
  \item{Humphrey, W., Dalke, A. and Schulten, K.,} ``VMD - Visual Molecular
Dynamics'' {\em J. Molec. Graphics} {\bf 1996}, {\em 14.1}, 33-38.
\end{description}

\VMD\ has been developed by the 
Theoretical and Computational Biophysics Group at the
Beckman Institute for Advanced Science and Technology of the
University of Illinois at Urbana-Champaign.
This work is supported by the National Institutes of Health under
grant numbers NIH~9P41GM104601 and 5R01GM098243-02.

\section{Acknowledgments}

The authors would particularly like to thank those individuals 
who have contributed suggestions and improvements, particularly 
those contributing new features.  Special thanks go to 
Joshua Anderson, Anton Arkhipov, Andrew Dalke, Michael Bach, 
Alexander Balaeff, Ilya Balabin, Robert Brunner,
Eamon Caddigan, Jordi Cohen, Simon Cross,
Markus Dittrich, John Eargle, 
Peter Freddolino, Todd Furlong, Luis Gracia, 
Paul Grayson, Justin Gullingsrud, James Gumbart,
David Hardy, Konrad Hinsen,
Barry Isralewitz, Sergei Izrailev, Robert Johnson,
Axel Kohlmeyer, Michael Krone, Michelle Kuttell, Benjamin Levine,
John Mongan, Jim Phillips, Elijah Roberts, Jan Saam, 
Charles Schwieters, Marcos Sotomayor, Alexander Spaar,
John E. Stone, Johan Strumpfer,
Alexey Titov, Leonardo Trabuco, Dan Wright, and Kirby Vandivort.
The entire \VMD\ user community now benefits from your contributions.

The authors would like to thank individuals
who have indirectly helped with development 
by making suggestions, pushing for new features, and trying out buggy code.  
Thanks go to 
Aleksei Aksimentiev, Daniel Barsky, Axel Berg, 
Tom Bishop, Robert Brunner, Ivo Hofacker, Mu Gao, 
James Gumbart, Xiche Hu, Tim Isgro, Dorina Kosztin, Ioan Kosztin, 
Joe Landman, Ilya Logunov, Clare Macrae, Amy Shih, 
Lukasz Salwinski, Stephen Searle, Charles Schwieters, 
Ari Shinozaki, Svilen Tzonev, Emad Tajkhorshid, Michael Tiemann, 
Elizabeth Villa, Raymond de Vries, Simon Warfield,
Willy Wriggers, Dong Xu, and Feng Zhou.  

Many external libraries and packages are used in \VMD, and the 
program would not be as capable without them.  
The authors of \VMD wish to thank
the authors of FLTK;
the authors of Tcl and Tk; 
the authors of Python; 
the authors of VRPN;
Jon Leech for uniform point distributions;
Amitabh Varshney for SURF;
Dmitrij Frishman for developing STRIDE; 
Jack Lund for the url\_get perl script; 
Brad Grantham for the ACTC triangle consolidation library;
John E. Stone for the Tachyon ray tracer, WorkForce threading and timer routines, hash table code, and Spaceball drivers;
\index{rendering!Tachyon}
\index{spaceball!driver}
and 
Ethan Merrit for one of the ribbon drawing algorithms. 


\section{Copyright and Disclaimer Notices}
\index{VMD!copyright}
\index{copyright}
\begin{center}
VMD is  Copyright \copyright\ 1995-2015 
Theoretical and Computational Biophysics Group and the \\
Board of Trustees of the University of Illinois
\end{center}

\noindent
Portions of this code are copyright \copyright\ 1997-1998 Andrew Dalke.

\bigskip

The terms for using, copying, modifying, and distributing VMD are
specified by the VMD License.  The license agreement is distributed
with VMD in the file LICENSE. If for any reason you do not have this
file in your distribution, it can be downloaded from: \\ 
{\tt http://www.ks.uiuc.edu/Research/vmd/current/LICENSE.html} 
\\
Some of the code and executables used by \VMD\ have their own 
usage restrictions:  
\begin{itemize}

\LISTITEM{ACTC}\index{ACTC}
ACTC, the triangle consolidation library used in some versions of VMD,
is Copyright (C) 2000, Brad Grantham and Applied Conjecture,
all rights reserved.

Redistribution and use in source and binary forms, with or without
modification, are permitted provided that the following conditions
are met:
\\
  1. Redistributions of source code must retain the above copyright
     notice, this list of conditions and the following disclaimer.
\\
  2. Redistributions in binary form must reproduce the above copyright
     notice, this list of conditions and the following disclaimer in the
     documentation and/or other materials provided with the distribution.
\\
  3. All advertising materials mentioning features or use of this software
     must display the following acknowledgment:
       This product includes software developed by Brad Grantham and
       Applied Conjecture.
\\
  4. Neither the name Brad Grantham nor Applied Conjecture
     may be used to endorse or promote products derived from this software
     without specific prior written permission.
\\
  5. Notification must be made to Brad Grantham about inclusion of this
     software in a product including the author of the product and the name
     and purpose of the product.  Notification can be made using email
     to Brad Grantham's current address (grantham@plunk.org as of September
     20th, 2000) or current U.S. mail address.

\LISTITEM{Python}\index{Python}
Python is made available subject to the terms and conditions in CNRI's
License Agreement. This Agreement together with Python may be
obtained from a proxy server on the Internet using the following
URL: {\tt http://hdl.handle.net/1895.22/1012}

\LISTITEM{PCRE}\index{PCRE}
The Perl Compatible Regular Expressions (PCRE) library used in VMD
was written by Philip Hazel and is Copyright (c) 1997-1999 
University of Cambridge.  
\\
Permission is granted to anyone to use this software for any purpose on any
computer system, and to redistribute it freely, subject to the following
restrictions:
\\
1. This software is distributed in the hope that it will be useful,
   but WITHOUT ANY WARRANTY; without even the implied warranty of
   MERCHANTABILITY or FITNESS FOR A PARTICULAR PURPOSE.
\\
2. The origin of this software must not be misrepresented, either by
   explicit claim or by omission.
\\
3. Altered versions must be plainly marked as such, and must not be
   misrepresented as being the original software.
\\
4. If PCRE is embedded in any software that is released under the GNU
   General Purpose License (GPL), then the terms of that license shall
   supersede any condition above with which it is incompatible.

\LISTITEM{STRIDE}\index{stride}
STRIDE, the program used for secondary structure calculation, is
free to both academic and commercial sites provided that STRIDE will
not be a part of a package sold for money.  The use of STRIDE in
commercial packages is not allowed without a prior written
commercial license agreement.  See
{\tt http://www.embl-heidelberg.de/argos/stride/stride\_info.html}

\LISTITEM{SURF}\index{surf}
The source code for SURF is copyrighted by the original author,
Amitabh Varshney, and the University of North Carolina at Chapel Hill.
Permission to use, copy, modify, and distribute this software and its
documentation for educational, research, and non-profit purposes is
hereby granted, provided this notice, all the source files, and the
name(s) of the original author(s) appear in all such copies.
\\
BECAUSE THE CODE IS PROVIDED FREE OF CHARGE, IT IS PROVIDED "AS IS" AND
WITHOUT WARRANTY OF ANY KIND, EITHER EXPRESSED OR IMPLIED.
\\
This software was developed and is made available for public use with
the support of the National Institutes of Health, National Center for
Research Resources under grant RR02170.

\LISTITEM{Tachyon}\index{Tachyon}
The Tachyon multiprocessor ray tracing system and derivative code built
into VMD is Copyright (c) 1994-2015 by John E. Stone.
See the Tachyon distribution for redistribution and licensing information.
\index{rendering!Tachyon}
% Permission is granted to use Tachyon freely with VMD. -- JES 2006

\LISTITEM{Desmond and Maestro plugins by D. E. Shaw Research}
Copyright 2009, D. E. Shaw Research, LLC
All rights reserved.
\\
Redistribution and use in source and binary forms, with or without
modification, are permitted provided that the following conditions are
met:
\\
  Redistributions of source code must retain the above copyright
  notice, this list of conditions, and the following disclaimer.
\\
  Redistributions in binary form must reproduce the above copyright
  notice, this list of conditions, and the following disclaimer in the
  documentation and/or other materials provided with the distribution.
\\
  Neither the name of D. E. Shaw Research, LLC nor the names of its
  contributors may be used to endorse or promote products derived from
  this software without specific prior written permission.
\\
THIS SOFTWARE IS PROVIDED BY THE COPYRIGHT HOLDERS AND CONTRIBUTORS
"AS IS" AND ANY EXPRESS OR IMPLIED WARRANTIES, INCLUDING, BUT NOT
LIMITED TO, THE IMPLIED WARRANTIES OF MERCHANTABILITY AND FITNESS FOR
A PARTICULAR PURPOSE ARE DISCLAIMED. IN NO EVENT SHALL THE COPYRIGHT
OWNER OR CONTRIBUTORS BE LIABLE FOR ANY DIRECT, INDIRECT, INCIDENTAL,
SPECIAL, EXEMPLARY, OR CONSEQUENTIAL DAMAGES (INCLUDING, BUT NOT
LIMITED TO, PROCUREMENT OF SUBSTITUTE GOODS OR SERVICES; LOSS OF USE,
DATA, OR PROFITS; OR BUSINESS INTERRUPTION) HOWEVER CAUSED AND ON ANY
THEORY OF LIABILITY, WHETHER IN CONTRACT, STRICT LIABILITY, OR TORT
(INCLUDING NEGLIGENCE OR OTHERWISE) ARISING IN ANY WAY OUT OF THE USE
OF THIS SOFTWARE, EVEN IF ADVISED OF THE POSSIBILITY OF SUCH DAMAGE.


\end{itemize}


\label{section:ig:ref}

