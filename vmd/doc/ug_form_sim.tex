\subsection{IMD Connect Simulation Window}
\label{ug:ui:window:sim}
\index{window!IMD simulation}

\VMD\ has the ability to work with a molecular dynamics program
running on another computer, to interact with and display the results 
of a simulation as they are calculated.
A major feature in \VMD\ is the ability to add
perturbative steering forces to a running simulation, which are
incorporated directly into the dynamics calculation;
we refer to this capability as Interactive Molecular Dynamics (IMD).  
\index{imd!requirements}
In order to run and IMD simulation it is necessary to have a molecular 
dynamics program that supports the IMD communication protocol.  
To date, two such programs exist; NAMD, developed at University of Illinois,
and Protomol, developed at Notre Dame.
The rest of the discussion in this chapter assumes you are using \NAMD.
See \htmladdnormallinkfoot{the \NAMD\ home page}
{http://www.ks.uiuc.edu/Research/namd/} for information on obtaining \NAMD.

\subsubsection{Interactive Molecular Dynamics}
\index{IMD}
\index{Interactive Molecular Dynamics}
IMD works by establishing a TCP connection between \VMD\ and the 
molecular dynamics simulation program.  NAMD, or whichever MD program is 
being used, acts as the server. 
In order to prepare \NAMD to accept \VMD's IMD connection request, 
\NAMD must be configured to listen for incoming connections on a network port.
Once NAMD has started up, may wait for the user to connect through 
that port.  When \VMD connects to \NAMD succesfully, the simulation commences.
Before connecting to the remote simulation, the \VMD user must first load a 
molecule corresponding to the system being simulated.  The structure file
should correspond to the same structure file used by \NAMD.
Once the molecule is loaded and NAMD has been started and is listening for a 
connection, you are ready to connect to the simulation and start receiving 
coordinates.  To establish a connection, open the Simulation window, enter 
the hostname
on which NAMD is running and the port on which NAMD is listening for incoming
connections, then press the Connect button to establish the connection.
If NAMD is running on several distributed nodes, VMD must connect to the root 
node on which NAMD initially started out.  

\subsubsection{IMD Using the Simulation window}
The {\sf Simulation} window allows you to control the behavior of a 
molecular dynamics
simulations which has been previously connected to through use of the
Remote window.  This window contains controls to change parameters for the
simulation and to affect how \VMD\ displays the results of the simulation.
The window also contains informative displays, which show the current status
of the simulation connection, and such things as the current energy,
temperature, and timestep of the molecular system being simulated.

At the top of the window are two entry fields and a button for establishing
a connection to a running MD simulation.  Enter both the hostname on which 
the simulation is running, and the port on which the simulation is listening,
then press the Connect button to establish the connection.  See the text 
console for possible error messages and status updates.
Below the connection display is a browser used to set some connection 
parameters.  These include:
\index{remote!connection!modifiable parameters}
\begin{itemize}
  \item {\bf Transfer Rate}: How often a timestep is transferred from the
remote simulation program to \VMD.  By default, this is 1, which means
every calculated timestep is sent.  If this is set to some value N,
then only every Nth step will send from the remote computer, 
thereby decreasing the amount of network processing and rendering 
that needs to be done.

  \item {\bf Keep Rate}: How often \VMD\ saves the timestep in its
animation list, instead of just discarding it after displaying it.  By
default, this is 0, which means that \VMD\ does not save any timesteps.
When this is 0, then when \VMD\ receives a new timestep it {\em replaces} the
last timestep in the animation list with the new timestep, instead of
{\em appending} it.  When it is set to some number N larger than 0, then
every Nth timestep received from the remote simulation will be appended to
the molecule. 
\end{itemize}

Parameters may be changed by entering text into the appropriate entry field
and pressing {\tt <return>}.
When a new value is entered, a command is sent to the remote simulation 
to change it.  There may be some delay between when the simulation gets 
commands, acts on them, and the results propagate back to \VMD.
Connection state is shown in the center of the window.
The simulation status text area displays energy values for the 
system being simulated (kinetic, electrostatic, etc.),
as well as the current timestep and the temperature.  
It is updated each time a new coordinate set (timestep) 
is received by \VMD.
\index{remote!connection!killing}
The {\sf Stop Sim} button will terminate the remote simulation, but
will not delete the molecule in \VMD.
\index{remote!connection!detaching}
The {\sf Detach Sim} button will sever the connection
between \VMD\ and \NAMD, 
but will allow the simulation to continue running.

