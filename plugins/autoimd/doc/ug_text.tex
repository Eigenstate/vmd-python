
\section{Introduction to AutoIMD}

\descr {AutoIMD is a means of quickly starting a NAMD molecular dynamics 
simulations from atomic coordinates in VMD. The simulation can then be visualized in
real-time on the VMD graphics screen. With the appropriate devices (such
as a VRPN tracker), the user can also interact with a running simulation
and apply forces to individual atoms or residues. Such an interaction is 
extremely useful when building and modeling systems, and can also be used
to gain precious insights by tinkering with key residues and atoms.}

\descr {While the \htmladdnormallink {\emph{Interactive\ Molecular Dynamics}} {http://www.ks.uiuc.edu/Research/vmd/imd/} (IMD) technology that enables 
all of this is an intrinsic part of VMD, AutoIMD makes it much more
 accessible and useful. In general terms, AutoIMD adds the following capabilities:}
\vspace{-1\baselineskip}
\begin{itemize}
\item The user can instantaneously get a simulation running for a subset of his/her system which is specified through a VMD atom selection. 
\item Instead of a full-blown simulation, AutoIMD can also be used to quickly 
minimize the energies of parts of a system ({\em e.g.}: on which external 
manipulations have been performed, for example).
\item It provides a graphical user interface for setting basic simulation parameters as well as for specifying which machine or cluster to run the simulation on.
\end{itemize}

\screenshot{autoimd-diagram}{An example AutoIMD session. The molten zone is in the center, surrounded by the fixed zone and then by the excluded zone. A pointer is being used to interact with the system in real-time.}{fig:autoimd-diagram}

\descr {Fig.~\ref{fig:autoimd-diagram} is an example of an AutoIMD session.  When using
AutoIMD, you can divide up your system into three distinct regions,
shown in the figure:\footnote{ The ``molten zone'' method used here
was first discussed in the context of IMD in J.~Prins {et~al.}
{A~virtual environment for steered molecular dynamics.} {\sl Future
Generation Computer Systems} {\bf 15}, 485--495 (1999).}}
\vspace{-1\baselineskip}
\begin{itemize}
\item The {\em molten zone} is the region where the atoms are allowed
to move freely, typically surrounded on all sides by the fixed zone.
This is the part of the system that you are interested in interacting
with (typically using a {\em pointer} as shown). You can refer to this
region through the {\tt imdmolten} atomselect macro.
\item The {\em fixed zone} is included in the simulation, but its
atoms are held fixed. Forces from atoms in the fixed zone will
affect the moving atoms and constrain their motion. You can refer to this
region through the {\tt imdfixed} macro.
\item The {\em excluded zone} is removed from your system before
the simulation is started and are ignored by the simulation (in order to speed it up).
You can refer to this region through the {\tt imdexcluded} macro.
\end{itemize}


\section{AutoIMD Requirements}

\descr {To get started using AutoIMD with your system, you will need the following:}
\vspace{-1\baselineskip}
\begin{enumerate}
\item A computer running VMD. Computers running MS Windows are supported as of VMD 1.8.4.
\item An installed copy of NAMD, preferably in the default path {\tt /usr/local/bin} (if not, you will get a prompt). 
If you wish to run your simulations on a local cluster, you 
need to perform some additional steps which are detailed later, in section
 \ref{par:customize}.
\item If you wish to use a haptic device, you must also set up VMD
 accordingly (see the VMD documentation).
\item Atomic coordinates (from a PDB, DCD file, etc.) and a PSF file
 describing your system.
\item A CHARMM parameter file for your simulation if your system requires non-standard parameters. The standard parameters are included in AutoIMD, but can also be downloaded here: \\ \htmladdnormallink{http://www.pharmacy.umaryland.edu/faculty/amackere/force\_fields.htm}{http://www.pharmacy.umaryland.edu/faculty/amackere/force\_fields.htm}.
\end{enumerate}


%\section{An example system}

%\descr {You may also want to get {\tt autoimd-example.tar.gz}, a set of PSF
%and PDB files for the aquaglyceroporin GlpF.  These are example files
%that you can use to try out AutoIMD. To read more about the system,
%see the IMD-GlpF page at \htmladdnormallink{{\tt http://www.ks.uiuc.edu/Research/smd\_imd/imd-glpf/}}{http://www.ks.uiuc.edu/Research/smd_imd/imd-glpf/}.}

\section{How to Run an AutoIMD session}

\nitem {Start VMD, load your system (make sure to include a PSF file), and start AutoIMD
from the {\gui Extensions~$\to$ autoimd} menu item (in the VMD main window). The AutoIMD window 
(Fig.~\ref{fig:autoimd-GUI}) should appear.}

\screenshot{autoimd-GUI}{The AutoIMD main window.}{fig:autoimd-GUI}


\nitem {\emph{[OPTIONAL]} Select the {\gui Settings$\to$Simulation Parameters...} menu item from 
the AutoIMD window. The dialog box shown in Fig.~\ref{fig:simsettings-GUI} should pop up. Use this dialog box to tell AutoIMD the location of your scratch directory.
You may also specify an alternate set of CHARMM parameter files, if your system contains non-standard residues.}

\screenshot{simsettings-GUI}{The AutoIMD Simulation Parameters dialog box.}{fig:simsettings-GUI}



\nitem {Specify the molten zone by entering a VMD atom selection into the text box.}

\descr {\emph{NOTE: Because
of the way that AutoIMD currently works, you should avoid referencing atoms by their {\tt index} 
or {\tt residue} number (since they will change in the simulated system), instead, refer to 
atoms using their {\tt name}, {\tt resid} and {\tt segname} or {\tt chain}, \emph{etc}.}} 

\nitem {Optionally, you can also change the fixed zone that is used to hold the molten zone into place, although the default should work. 
Note that the fixed zone that you specify might later be adjusted by AutoIMD to include complete
residues instead of residue fragments, this is normal.}
 

\nitem {Pick a server and adjust the number of processors. If you want to run the simulation on the same
computer that VMD is running on, you would pick ``Local" and ignore the processors field.}

\nitem {Click the {\gui Submit} button to start your NAMD simulation.}

\descr {\emph{NOTE: If this is your first time running AutoIMD, it will prompt you to create 
a scratch directory in which AutoIMD will store its temporary files. 
The scratch directory needs to be accessible on both
your local machine and the host on which NAMD will run. Do not store your own files in this directory.}}

\descr {\emph{NOTE: If you get an error message saying ``Unable to
    open psf file xxx.psf.", you need to reload your PSF file
    on top of your current molecule one more time and click  Submit again.}} 
  
\nitem {Click {\gui Connect} to connect to the simulation with IMD.
You should see the atoms start to move. }

\descr {\emph{NOTE: It might take a few seconds for the NAMD
    simulation to get started. During this time, VMD will not be able
    to connect and you will see some error messages of the type: 
      Error connecting to host on port xxxx. This is
    normal, and these errors can usually be ignored. VMD will connect as soon
    as NAMD is ready.}} 
    
\descr {\emph{NOTE: If after waiting a long time, AutoIMD still does not connect, your
simulation may have had trouble starting properly. Make sure that your simulation is actually running.
 If not, your NAMD job may have aborted. Check the NAMD output in the  autoimd.log file in your scratch directory to 
find the reason (NAMD will say there why it crashed, etc.). Once this is fixed, you can click on Discard and then Submit, to try again.}} 
    
\nitem {Interact with your system for a while.  You can use the {\gui Mouse~$\to$ Force} menu items, or a 3D tracker to pull on your atoms. Alternatively, just can also just watch your system equilibrate. At any
point during your simulation, you can save a PDB snapshot of your system by
choosing the {\gui File~$\to$ Save Full PDB As...} menu item.}

\nitem {When you are done, click {\gui Finish}.  The coordinates
of your atoms in your originally loaded molecule will updated, and the
simulation will stop. {\gui Discard} also stops the simulation, but
does not update the coordinates and you will be brought back to your initial state. 
Use this button it if anything goes wrong.}

\section{AutoIMD Adjustable Parameters}
\label{par:params}

AutoIMD allows you to set a number of adjustable parameters, through the Simulation Parameters window ({\gui Settings~$\to$ Simulation~Parameters} menu item). These options are described here:

\paragraph {scratch directory} The directory in which AutoIMD will store its temporary files needed to run the NAMD simulation. Do not store your own files it it.

\paragraph {CHARMM params} The list of CHARMM parameters files used by NAMD. You will need to modify this if your system contains non-standard residues for which you have created parameters.

\paragraph {temperature} - The temperature at which the simulation dynamics will be run, in degrees Kelvin.

\paragraph {NAMD config template} - AutoIMD runs a NAMD simulation using a pre-defined template which contains all the instructions to be followed by NAMD. If you wish to create a custom simulation, e.g. use different NAMD settings and protocols than what AutoIMD offers, you should copy the template provided by AutoIMD, edit it, and specify it here.

\paragraph {initial minimization} The number of minimization steps performed before the equilibration starts.

\paragraph {DCD save frequency} The frequency at which simulation steps are saved to disk. These will be located in a DCD file called autoimd.dcd in the scratch directory. To load and view this trajectory in VMD, you will also need to load the autoimd.psf file in the same directory.

\paragraph {VMD keep frequency} The frequency at which simulation steps are kept in VMD's memory. You can view these by using the trajectory animation slider.

\paragraph {IMD communication rate} If this is set to something greater than 1, NAMD will not send every computed step to VMD, but will skip steps and only send coordinates to VMD at a set interval.



\section{Making your settings permanent and adding new servers}
\label{par:customize}

\descr {If you use AutoIMD a lot, you might wish that could set your default settings 
in a startup script and not have to worry about typing them in each time. Almost
everything in AutoIMD is customizable.}

\descr {To set your default scratch directory and CHARMM parameter file, you can include 
the following lines (provided as an example only) in your {\tt .vmdrc} startup file. Be sure to use absolute paths!}
\vspace{-1\baselineskip}
\begin{verbatim}
    package require autoimd

    autoimd set scratchdir  /home/user/scratchdir
    autoimd set parfiles    {/home/par_paramfile1.inp  /home/par_paramfile2.inp  ...}
    [...]
\end{verbatim}

\descr {For more advanced customization, it is recommended that you copy the file 
{\tt autoimd-settings.tcl} that is distributed with the AutoIMD package, to your local
directory. You can then edit it to suit your needs (it is self-documented). Have it be run at
 startup by adding the following line to your {\tt .vmdrc} file (be sure to use the right path):}
\vspace{-1\baselineskip}
\begin{verbatim}
    source autoimd-settings.tcl
\end{verbatim}


\descr {If you have access to a local cluster for running NAMD, this file is also where you would tell
 AutoIMD how to access it.}

\section{Customizing the representations used during AutoIMD}

\descr {AutoIMD uses multiple representations to help you keep track of the
molten, fixed, and excluded zones in your simulation.  When a
simulation is submitted, two molecules are used at once: one
containing the original, complete structure and one containing just
the atoms needed for IMD.  To avoid showing the same atoms twice,
AutoIMD displays only the excluded and fixed zone of the original molecule. 
To accomplish this, the atomselection macros {\tt imdexclude},
{\tt imdfixed}, and {\tt imdmolten} are defined to correspond to
the three regions of the simulation. You can take advantage of them to 
select desired parts of your simulations.}

\descr {You can customize both molecules' representations with the VMD
Graphics Form, just as in a normal VMD session. AutoIMD saves the
representation used for the IMD molecule when ending a simulation and
reloads it when a new simulation is submitted.  The default IMD representation
can be permanently customized through AutoIMD's {\tt imdrep} variable. 
Here is an example of how it can be used:}
\vspace{-1\baselineskip}
\begin{verbatim}
    autoimd set imdrep {
        mol representation Bonds 0.300000 6.000000
        mol color Name
        mol selection "imdmolten and not hydrogen and not water"
        mol material Opaque
        mol addrep $imdmol

        mol representation VDW 1.000000 8.000000
        mol color Name
        mol selection "imdmolten and numbonds 0"
        mol material Opaque
        mol addrep $imdmol

        mol representation Bonds 0.300000 6.000000
        mol color Name
        mol selection "imdmolten and water"
        mol material Opaque
        mol addrep $imdmol

        mol representation Tube 0.400000 6.000000
        mol color ColorID 12
        mol selection "imdfrozen"
        mol material Opaque
        mol addrep $imdmol
    }
\end{verbatim}


\descr {To view the current IMD representation, you would type (at the VMD prompt):}
\vspace{-1\baselineskip}
\begin{verbatim}
    puts $AutoIMD::imdrep
\end{verbatim}


\section{Getting Additional Help}

\descr {For additional help, as well as to provide feedback, contact us as {\tt vmd\@ks.uiuc.edu}.}

